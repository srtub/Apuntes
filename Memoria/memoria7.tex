

\documentclass[a4paper,11pt]{article} 
\usepackage[spanish]{babel}           
\usepackage[utf8]{inputenc}           

\usepackage[T1]{fontenc}   		   % Fonte por defecto.
\usepackage{graphicx, subfigure}    		   % Engadir imaxes.
\usepackage{color}      		   % Uso de cores.
\usepackage{anysize}     		   % Modificar o tamaño dos marxes.
\usepackage{multicol, multirow}    % Escribir a doble, triple...columna.
\usepackage{bm}          		   % Letras gregas en negriña.
\usepackage{textcomp}    		   % Símbolos, poden consultarse na rede.
\usepackage{eurosym}     		   % Símbolo € (\euro).
\usepackage{amsthm}                % Paquete da AMS para escribir teoremas.
\usepackage{amsmath,amsfonts}      %Paquetes específicos de símbolos.
\usepackage{lineno}                % Numerar as liñas. 

\marginsize{2cm}{2cm}{1.5cm}{1.5cm} % MARXES: Esq, der, sup, inf.
\parindent=0mm                        % Sangría. 
\parskip=2mm                          % Espazo entre párrafos.
\renewcommand{\baselinestretch}{1}    % Interliñado.
\renewcommand{\spanishtablename}{Táboa} 


\title{Tema 7: Memoria de testemuñas}
\date{}


\begin{document}  

\maketitle

\section{Principais factores que inflúen na exactitude da testemuña}
\begin{itemize}
	\item \underline{Testemuño sospeitoso}: Relaciónase co fenómeno de <<cegueira ó cambio>>, que é 
	a tendencia a non notar cambios aparentemente obvios nun obxecto. Aínda que en xeral cremos que 
	procesamos toda a información visual dunha situación de forma razoablemente completa, o certo é 
	que detectamos moito máis eficazmente os cambios nos obxectos que observamos de forma directa. 
	
	Podería pensarse que as testemuñas que presencian eventos dramáticos e novidosos deberían estar
	moi atentos, pero non ten por que ser así. Hai varios factores en contra da testemuña que poden
	distorsionar a súa memoria: normalmente, non espera que o evento se produza; probablemente irá 
	pensando nas súas propias cousas; e o que ve, a veces, é moi breve.
	\item \underline{Recordo do que se espera ver}: O recordo dos eventos pode verse influenciado
	polo que esperamos ver. Bartlett (1932) explícao argumentando que os nosos esquemas de
	coñecemento almacenados na MLP nos levan a formar certas expectativas das situacións. O recordo
	implica un proceso de reconstrución no que se emprega toda a información relevante, incluída a
	dos propios esquemas. Por este motivo, moitas veces a información recordada non procede da
	observación, senon da información almacenada nos esquemas.
	
	\item \underline{Preguntas capciosas}: Loftus e Palmer (1974) afirman que os recordos das 
	testemuñas son fráxiles, e que poden distorsionarse tras a observación do evento. Nun estudo,
	amosan ós participantes un vídeo dun accidente de tráfico múltiple. Logo pregúntanlles cal 
	sería, aproximadamente, a velocidade dos coches ó chocar. Noutros casos, substitúen a palabra 
	<<chocar>> por <<colisionar>>, <<darse>>, <<contactar>> ou <<estrellarse>>. Atopan que as 
	estimacións da velocidade varían en función do verbo empregado, sendo máis altas con 
	<<estrellarse>> e máis baixas con <<contactar>>. Unha semana máis tarde pregúntanlles se viron 
	cristales rotos na escena. Aínda que non os había, o 32\% dos participantes cos que se empregara 
	o verbo <<estrellarse>> afirmou que si, fronte a tan só o 14\% daqueles cos que se empregara 
	<<chocar>>.
	
	Como vemos, o recordo das testemuñas influénciase facilmente mediante información enganosa 
	presentada logo da observación dun evento. Este é o efecto da \textit{interferencia 
	retroactiva}, que se define como a alteración dun recordo debido á aprendizaxe doutro material 
	durante o intervalo de retención. Tamén pode darse a \textit{interferencia proactiva} cando as 
	experiencias previas dos suxeitos son relevantes para as preguntas que se lles fan, e en 
	consecuencia distorsionan as súas respostas.
	
	Crese que a información enganosa distorsiona o recordo mediante un proceso de monitorización da
	fonte. Cando se activa a búsqueda na memoria, cunha pregunta por exemplo, actívanse recordos
	procedentes de varias fontes. O suxeito é o encargado de discernir a información contida nestas
	para dar a súa resposta. Porén, podería producirse unha atribución errónea da fonte, co que se
	recuperaría información sobre un recordo erróneo.
	
	Loftus (1992) tamén propón que pode producirse unha ``aceptación da información enganosa'' por
	parte das testemuñas, que posteriormente os levaría a considerar esta información como parte do 
	seu recordo. Este proceso de aceptación sería máis común a medida que aumentara o tempo 
	transcorrido desde o evento.
	
	\item \underline{Diferenzas individuais}: A idade das testemuñas é moi importante. Sábese que os
	nenos máis maiores son mellores testemuñas que os máis pequenos, así como o son os adultos 
	xóvenes en comparación cos anciáns. As persoas maiores contan con dúas desventaxas: son máis 
	facilmente influenciables mediante información enganosa e adoitan mesturar os recordos, no 
	sentido de que recordan información xenuína pero poden olvidar o contexto no que a adquiriron.
	
	\item \underline{Confianza da testemuña}: Crese que a confianza que a testemuña ten no seu 
	recordo non é, en xeral, un bó predictor da exactitude de dito recordo. Isto ocorre porque as 
	testemuñas non saben se a súa habilidade para recordar un evento é mellor ou peor que a dos 
	demais, e en consecuencia non posúen unha base sólida para determinar o seu nivel de confianza. 
	A confianza tamén pode aumentar debido á retroalimentación confirmatoria, que ocorre cando un 
	axente externo indica á testemuña que o seu recordo é correcto.
	
	\item \underline{Influenza da ansiedade e da violencia}: En canto á violencia, en xeral, provoca
	unha mellora no que se consideran aspectos centrais dun evento, pero empeora o recordo de
	aspectos periféricos. Pode darse un <<efecto de focalización na arma>>, que se da cando a
	presenza dunha arma impide o recordo de detalles sobre o agresor e o entorno. Este efecto é 
	raro, en tanto que debería ser máis sinxelo fixarse na cara de alguén que posiblemente cometera 
	un delito. Hai dúas posibles explicacións para que suceda isto: primeiro, que a arma constitúe 
	unha importante ameaza; e segundo, que a arma chame a atención porque é inesperada na maioría de 
	contextos nas que é vista polas testemuñas.
	
	En canto á ansiedade e o estrés, un nivel elevado perxudica a exactitude na identificación de
	testemuñas e reduce o recordo de detalles.
\end{itemize}

\section{O recordo de caras}
Crese que non implica exactamente os mesmos procesos que a identificación de obxectos, sobre todo porque, en xeral, os pacientes prosopagnósicos (``cegos para as caras'') son eficaces no recoñecemento dos mesmos. Ademais, a maioría destes pacientes presenta danos na área fusiforme facial do cerebro, que se activa considerablemente durante a identificación de caras.

Patterson e Baddeley (1977) identificaron algúns factores que determinan o bó ou mal recordo de caras. Categorizaron fotografías de persoas non familiares para os participantes segundo as súas características físicas e psicolóxicas, e disfrazaron a algunhas delas. As persoas aparecían de frente ou de perfil.

Os participantes recoñecían mellor as caras que se categorizaran segundo dimensións psicolóxicas que segundo as físicas, o que indica que non se obtén vantaxes no recordo ó analizar as caras segundo os rasgos que as compoñen. En canto ós disfraces, cada vez que se engadía ou se retiraba un elemento diminuía a probabilidade de recoñecemento da cara.

Por outra banda, o procesamento de caras é diferente do de outros obxectos: a análise que facemos das caras é holística ou global, e o procesamento de obxectos é detallado. É dicir, procesamos a estrutura global da cara e prestamos pouca atención ós detalles, mentres que as partes dun obxecto se procesan unha a unha. Unha boa forma de considerar o boas que son as persoas identificando caras é presentándoas de forma invertida, xa que aínda que estas seguen sendo recoñecibles, é moito máis difícil identificar correctamente a súa estrutura global cando as vemos do revés (\textit{exemplo:}ilusión Tatcher).

Un dato curioso é que moitas veces as testemuñas son quen de recordar as caras pero non o contexto no que as viron. Isto ten implicacións moi importantes, coma no caso da <<transferencia inconsciente>>, que se da cando se recoñece unha cara como vista anteriormente e se lle atribúe, de forma incorrecta, a responsabilidade dun acto delictivo.

A maioría dos datos suxiren, tamén, que o recoñecemento de caras dunha testemuña empeora considerablemente se se proporciona unha descripción verbal da cara en cuestión antes de someter á testemuña a unha roda de recoñecemento. É o que se coñece como <<ensombrecemento verbal>>.

Por último falaremos do <<efecto da outra raza>>, que implica un recoñecemento máis preciso das caras da mesma raza que das de outras. Este efecto explícase mediante dúas hipóteses: a da pericia, que explica que como a maioría das persoas teñen máis experiencia na distinción de caras da mesma raza, resúltalles máis sinxelo identificar a raza propia; e a sociocognitiva, que explica que realizamos un procesamento moito máis minucioso das caras dos individuos cos que nos identificamos que daqueles cos que non nos identificamos.

\section{Procedementos policiais con testemuñas}
\begin{itemize}
	\item \textbf{Rodas de recoñecemento:} Nelas preséntase ó sospeitoso xunto con varios 
	individuos, cuxas características globais son semellantes, e pregúntaselle á testemuña se 
	recoñece a alguén como o delincuente. Sábese que o rendemento das testemuñas non é infalible 
	nestas rodas, e para mellorar a súa execución tómanse varias precaucións: advírtese ás 
	testemuñas de que o culpable non ten por que estar presente na roda, e empréganse rodas 
	secuenciais nas que o suxeito ve as caras de unha en unha.
	\item \textbf{Entrevista ás testemuñas:} Actualmente existen certos métodos que fan que as
	entrevistas sexan máis axeitadas:
	\begin{itemize}
		\item[-] Fanse preguntas con final aberto para que as testemuñas poidan extenderse coas
		respostas.
		\item[-] Evítanse as interrupcións por parte dos entrevistadores, xa que estas perxudican a
		concentración das testemuñas e fan que sexa máis difícil recuperar información relevante.
		\item[-] Non se plantexan as preguntas nunha orde determinada, senon que se teneñ en conta 
		as respostas da testemuña para ir formulando as preguntas.
	\end{itemize}
	
	Unha entrevista particularmente importante é a cognitiva, que se basea en catro regras xerais de
	recuperación:
	\begin{enumerate}
		\item Restablécense mentalmente o entorno e o contacto persoal experimentados durante o
		delito.
		\item Anímase a contar tódolos detalles da situación.
		\item Faise que as testemuñas describan o incidente en varias ordes distintas (de atrás cara
		adiante).
		\item Téntase contar o sucesos desde puntos de vista distintos (os de outras testemuñas).
	\end{enumerate}
	
	Esta entrevista adoita ser moi eficaz porque fai uso de todo o coñecemento existente sobre a
	memoria humana. As dúas primeiras regras seguen o principio da especificidade da codificación de
	Tulving (1979), que di que os suxeitos recordan máis cando o contexto no que se observou o 
	delito se parece ó contexto no que se recupera. A terceira e a cuarta baséanse na suposta 
	complexidade dos trazos de memoria, a partir dos cales pode recuperarse información diversa 
	sobre a experiencia orixinal.
	
	Con todo, a entrevista cognitiva é menos efectiva se se emprega tras certo intervalo temporal
	desde a observación do evento. Ademais, é máis útil para incrementar o recordo de detalles
	periféricos en lugar dos centrais. Por último, as regras da entrevista empréganse de forma
	conxunta, polo que non se sabe con certeza que aspectos son os máis responsables da súa 
	eficacia.
\end{itemize}

\section{Do laboratorio ó tribunal}
Existen claras diferenzas entre as experiencias das testemuñas no laboratorio e na vida real: 
\begin{itemize}
	\item O evento é observado polas testemuñas no laboratorio, pero na vida real é moito máis
	probable que sexa a víctima quen aporte as probas.
	\item A práctica no laboratorio causa menos ansiedade e estrés que a observación dun delito 
	real.
	\item As testemuñas de laboratorio adoitan observar o evento desde unha única perspectiva, pero 
	na vida real é probable que se movan pola escea. 
	\item No recordo de caras, as testemuñas teñen só uns segundos para estudar cada cara; na vida
	real é moi posible que se dispoña incluso de minutos. 
	\item As consecuencias dunha identificación errónea por parte das testemuñas de laboratorio son
	practicamente nulas, pero na vida real poden costarlle a vida a alguén.
\end{itemize}

Por estes e outros motivos existe a preocupación de que as declaracións de expertos en testemuños nos xuízos poidan facer que os xurados sexan demasiado escépticos sobre a validez do que digan as testemuñas.

En primeiro lugar, os descubrimentos sobre memoria de testemuñas adoitan ser inconsistentes, así que os expertos non poden dar unha mensaxe clara. En segundo lugar, a exactitude da memoria das testemuñas depende tanto dos aspectos situacionais como das súas características individuais, que normalmente non se teñen en conta. E por último, moitos dos datos sobre o testemuño das testemuñas son sinxelos, pero a importancia de cada factor depende sempre da situación (o efecto de focalización da arma ten sentido nun contexto de segundos, non de minutos); estes factores máis complexos non adoitan terse en conta.

Noutra liña de investigación, suxírese que os testemuños dos expertos si son útiles. Crese que os recordos de testemuñas reais son máis imprecisos que os dos de laboratorio, o que implica que as diferenzas de memoria entre testemuñas de laboratorio e reais están subestimadas. Así, a opinión dos expertos en testemuños podería solucionar varios problemas. En xeral, os xurados esaxeran a precisión do recordo das testemuñas, e a maioría de xurados son insensibles ós factores que inflúen nesta precisión. Unha forma de evitar a insensibilidade pode ser explicación dun experto, que expoña os descubrimentos científicos existentes sobre eses temas.


\end{document}