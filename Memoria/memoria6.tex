

\documentclass[a4paper,11pt]{article} 
\usepackage[spanish]{babel}           
\usepackage[utf8]{inputenc}           

\usepackage[T1]{fontenc}   		   % Fonte por defecto.
\usepackage{graphicx, subfigure}    		   % Engadir imaxes.
\usepackage{color}      		   % Uso de cores.
\usepackage{anysize}     		   % Modificar o tamaño dos marxes.
\usepackage{multicol, multirow}    % Escribir a doble, triple...columna.
\usepackage{bm}          		   % Letras gregas en negriña.
\usepackage{textcomp}    		   % Símbolos, poden consultarse na rede.
\usepackage{eurosym}     		   % Símbolo € (\euro).
\usepackage{amsthm}                % Paquete da AMS para escribir teoremas.
\usepackage{amsmath,amsfonts}      %Paquetes específicos de símbolos.
\usepackage{lineno}                % Numerar as liñas. 

\marginsize{2cm}{2cm}{1.5cm}{1.5cm} % MARXES: Esq, der, sup, inf.
\parindent=0mm                        % Sangría. 
\parskip=2mm                          % Espazo entre párrafos.
\renewcommand{\baselinestretch}{1}    % Interliñado.
\renewcommand{\spanishtablename}{Táboa} 


\title{Tema 6: Memoria autobiográfica}
\date{}


\begin{document}  

\maketitle

A memoria autobiográfica (tamén chamada <<memoria persoal>>) recolle os recordos que temos sobre nós mesmos, é dicir, sobre o noso eu, e sobre os acontecementos e experiencias concretas dos sucesos ó longo das nosas vidas, así como as emocións e pensamentos provocados por estas. É o resultado da interacción do eu co mundo e combina o noso coñecemento sobre o mundo externo co noso coñecemento de nós mesmos, dando como resultado a conciencia de identidade persoal e a capacidade individual para revivir o pasado, interpretar o presente e planificar o futuro.

Na memoria autobiográfica están implicados os sistemas de memoria episódica e semántica, e a recuperación dos seus contidos realízase de forma consciente.

\section{Por que precisamos a memoria autobiográfica?}
Considéranse catro funcións principais da memoria autobiográfica:

\begin{itemize}
	\item \textbf{Función directiva:} Orientada á resolución de conflitos. A partir dos recordos
	autobiográficos, os seres humanos constrúen modelos mentais do mundo no que viven, que lles
	serven de guía acerca do que se pode e se debe ou do que non se pode e non se debe facer. Os
	recordos persoais inclúen, moi a miúdo, directrices sobre que facer, como actuar ou que pensar
	en situacións futuras.
	\item \textbf{Función social ou comunicativa:} Compartir o recordo das nosas experiencias 
	persoais con outros favorece as nosas relacións cos demais, porque fai que nos sintamos 
	respaldados por eles. Considérase que as persoas compartimos os nosos recordos autobiográficos
	cos demais con tres obxectivos: para iniciar, manter e desenvolver boas relacións cos nosos
	semellantes; para ensinar e/ou informar a outros (explicar, dar consellos, etc.); e para amosar
	e provocar empatía.
	\item \textbf{Función representativa do eu:} A memoria autobiográfica permítenos organizar o
	coñecemento sobre nós mesmos. O coñecemento do propio eu no pasado e a súa proxección no futuro
	persoal son cruciais para o desenvolvemento, a integridade, o axuste e a contigüidade do eu.
	
	Esta función relaciónase coa terapia de reminiscencia, que consiste na reconstrución dos 
	recordos vitais a partir de fotografías e obxectos do noso pasado. 
	\item \textbf{Función de afrontamento das adversidades:} O recordo de experiencias positivas 
	pode contribuir a animarnos en momentos duros ou conflitivos.
\end{itemize}

Aínda que estas funcións son plausibles, nacen principalmente da especulación. Algúns estudos sobre o cuestionario TALE (\textit{Thinking About Life Experiences}) indican que as funcións se solapan entre elas. Quizais sexa posible que a memoria autobiográfica posúa diferentes funcións, pero é dubidoso que poidan explicarse separando unhas de outras na vida real.

\section{Diferenzas entre memoria semántica, memoria episódica e memoria autobiográfica}
\begin{itemize}
	\item \underline{Memoria semántica}: Sistema encargado de adquirir, reter e empregar feitos e
	conceptos xerais sobre o mundo. Non contén información sobre o tempo nin o espazo e non fai
	referencia ó Eu, isto é, non implica unha conciencia de pasado. A recuperación de información
	neste sistema só implica \textit{conciencia noética} ou de saber, pero non de revivir.
	\item \underline{Memoria episódica}: Sistema encargado de adquirir, reter e empregar a
	información relativa ós sucesos persoais pasados, ocorridos nun lugar e nun momento específicos.
	Neste sistema, a información organízase arredor dun evento concreto e a súa recuperación implica
	viaxar mentalmente no tempo e revivir as experiencias mediante a \textit{conciencia 
	autonoética} (experiencia consciente de un mesmo como unha entidade continua a través do tempo).
	\item \underline{Memoria autobiográfica}: Sistema encargado de almacenar os recordos persoais
	significativos na vida dunha persoa. Considérase un subsistema da memoria episódica, en tanto
	que todo o que recolle este sistema pertence tamén á memoria episódica, pero non tódolos
	recordos episódicos son autobiográficos. Para que o recordo dun evento sexa almacenado neste 
	sistema debe ser significativo na historia vital do individuo, e ir acompañado polas emocións,
	sentimentos, pensamentos, etc., provocados por esa experiencia. 
	
	Dentro deste sistema podemos diferenciar entre <<recordos autobiográficos>>, aqueles que
	implican unha representación mental do evento recordado, cun contexto espazo-temporal e
	emocional; e <<feitos autobiográficos>>, que son eventos inferidos da autobiografía: sábese que
	ocorreron, pero son tal lonxanos que é imposible recordalos. Os compoñentes semántico e
	episódico da memoria autobiográfica correspóndense cos feitos e cos recordos autobiográficos, 
	respectivamente.
\end{itemize}

\section{Métodos de estudo}
Nos estudos de memoria autobiográfica, o investigador non pode controlar a situación nin os estímulos. Isto fai que sexa difícil explicar como se adquire este tipo de memoria e como se produce o olvido. 

\subsection{Diarios}
Permiten un control obxectivo dos recordos dos suxeitos porque dan a coñecer o recordo inicial. 

Un estudo clásico con diario é o de Wagenaar (1986), quen acumula recordos durante máis de seis anos anotando dous eventos diarios. Elixe catro claves de recuperación para cada evento: \textit{que} evento é, \textit{quen} intervén nel e \textit{onde} e \textit{cando} ocorre. Puntúa tamén respecto á saliencia, ó nivel de implicación emocional e á sensación producida polo evento (agradable/desagradable).

Na fase de proba, selecciona os eventos ó azar xunto con unha, dúas ou tres claves de recuperación, presentadas de forma aleatoria. Descobre que as claves do \textit{que}, \textit{quen} e \textit{onde} son igual de efectivas á hora de lembrar o recordo, mentres que a do \textit{cando} é moito menos eficaz. 

O problema dos diarios é que os procesos de selección de recordos (que normalmente serán os máis memorables) e de repaso a través das claves favorecen o mantemento dos eventos na memoria. Ademais, son moi custosos polo tempo e esforzo que requiren.

\subsection{Método das palabras clave}
Proporciónaselle unha palabra ós participantes para que lembren un episodio autobiográfico a partir dela. Este método está adaptado tanto para o recordo de periodos vitais (infancia) coma de eventos específicos. Aínda que é moi sinxelo e pouco controlable, emprégase moito e aporta resultados moi produtivos.

Algo característico dos recordos é que non se distribúen de forma homoxénea ó longo da vida. Destacamos dous fenómenos:
\begin{itemize}
	\item \underline{Amnesia infantil}: Tendencia a lembrar poucos episodios autobiográficos
	pertencentes ós cinco primeiros anos de vida. Os recordos desta etapa, se existen, son eventos
	illados carentes de contexto espacial e/ou temporal. Conteñen moi pouca información, e esta é
	sobre todo emocional. 
	
	Dúas teorías explican este fenómeno: a do <<eu cognitivo>>, que explica que o neno non 
	desenvolve a súa memoria autobiográfica ata que se recoñece a si mesmo como ser individual (o 
	que ocorre arredor dos dous anos); e a do <<desenvolvemento sociocultural>>, que explica que 
	tanto a linguaxe como a cultura son condicións necesarias para a aparición da memoria 
	autobiográfica porque permiten que o neno fale dos seus recordos.
	
	\item \underline{Pico de reminiscencia}: Tendencia dos individuos maiores de 40 anos a lembrar
	máis experiencias autobiográficas situadas entre os 15 e os 30 anos. É posible que se deba a que
	neste periodo temporal adoitan suceder moitos eventos importantes na vida das persoas 
	(universidade, matrimonio, fillos, etc.), e a que ditos eventos posúen unha gran carga 
	emocional. Ademais, nas últimas etapas da vida abundan máis os recordos de feitos felices que os 
	de eventos negativos.
\end{itemize}

O método das palabras clave é moi empregado para a construción do relato vital ou historia de vida, que é unha descrición coherente e ordeada da nosa vida que as persoas creamos para nós mesmas. Os acontecementos incluidos no relato tenden a ser importantes para nós, codifícanse máis profundamente e recupéranse con maior facilidade.

O problema dos métodos retrospectivos é que normalmente non existe un rexistro obxectivo co que comparar os recordos dos suxeitos.

\section{O sistema de memoria do eu (Conway, 2005)}
Conway define a memoria autobiográfica como un sistema que mantén o coñecemento sobre o \textit{eu experencial}, o ``min''. O contido da memoria sempre se refire a el, pero non sempre produce recordos específicos sobre el; as experiencias de recordo só se dan cando o coñecemento autobiográfico mantén o acceso ós recordos episódicos asociados (podo lembrarme de ter ido de vacacións, pero non ter recordos específicos sobre as mesmas).

Os recordos autobiográficos son transitorios, e constrúense a partir da interacción entre dous elementos:
\begin{itemize}
	\item \textbf{Base de coñecemento autobiográfico:} Contén dous tipos de representacións, o 
	coñecemento autobiográfico e a memoria episódica. 
	\begin{itemize}
		\item \underline{Coñecemento autobiográfico}: Organízase xerarquicamente en tres niveis de
		abstracción diferentes:
		\begin{itemize}
			\item \textbf{Historias de vida:} Conteñen coñecemento xeral sobre o mundo e coñecemento
			avaliativo sobre o individuo. Son narracións integradoras do eu que proporcionan ó
			individuo unha vida actual cun mínimo de unidade e propósito psicosocial, permitíndolle
			organizar os seus recordos e o coñecemento abstracto do seu pasado nunha visión
			biográfica coherente. Reflexan os valores e normas culturais da sociedade e conteñen
			imaxes do eu que o separan en diferentes \textit{eus}.
			\item \textbf{Períodos vitais:} Representacións que encapsulan na memoria períodos
			concretos e longos de tempo que se miden en anos ou décadas, e que inclúen lugares,
			persoas, actividades, sentimentos e obxectivos comúns. Conteñen coñecemento avaliativo 
			(positivo e negativo) do proceso de consecución de obxectivos.
			\item \textbf{Acontecementos xerais:} Poden ser únicos (primeiro día na universidade),
			repetidos (quedada semanal cos amigos) ou prolongados (viaxe a Irlanda). Mídense en
			días, semanas ou meses, e organízanse de diferentes formas.
		\end{itemize}
		\item \underline{Memoria episódica}: Inclúe representacións mentais con características e
		organización propias, que se manifestan en circuitos cerebrais concretos. Os recordos
		episódicos posúen certas propiedades que os diferenzan doutras representacións de memoria:
		\begin{enumerate}
			\item Conteñen rexistros esquematizados do procesamento sensorio-perceptivo-conceptual-					afectivo.
			\item Reteñen patróns de activación/inhibición durante longos períodos de tempo.
			\item A miúdo represéntanse en forma de imaxes visuais.
			\item Sempre teñen unha perspectiva (de campo ou do observador).
			\item Representan porcións de experiencia de curta duración.
			\item Abarcan unha dimensión temporal cunha orde de ocorrencia.
			\item Están suxeitos a un olvido rápido.
			\item Van acompañados de conciencia autonoética.
			\item Proporcionan especificidade ós recordos autobiográficos.
		\end{enumerate}
	\end{itemize}
	
	A base tamén almacena episodios sensoperceptivos dos eventos, detalles sensoriais que nos
	confirman que os nosos recordos son auténticos e non fabulacións. 
	\item \textbf{Eu de traballo ou operativo:} Abarca un gran conxunto de obxectivos activos 
	(ideas, metas) e autoimaxes. Modula o acceso á MLP e á súa vez está influenciado por ela. 
	Codifica a información real (o que vivimos) e fantástica (o que podería ter ocorrido). A súa   
	función principal é manter a coherencia para xerar unha memoria consistente cos obxectivos
	presentes, coa autoimaxe actual e coas crenzas do individuo. Cando desaparece a coherencia poden
	surxir problemas que normalmente desembocan en fabulacións e delirios.
\end{itemize}

Conway relaciona os detalles autobiográficos coa conciencia autonoética, que é esencial para diferenciar os recordos reais das fabulacións. O acceso a ela é relativamente lento (uns segundos) en comparación con outros tipos de memoria.

A teoría de Conway é útil porque combina todo o que se coñece sobre a memoria autobiográfica, e suxire novas preguntas que quizais conduzan a un avance na disciplina.

\section{Diferentes tipos de recordos}
\subsection{Recordos voluntarios e involuntarios}
A condición elemental para a formación dun recordo autobiográfico é que a memoria episódica se conecte co coñecemento autobiográfico. No proceso de formación dos recordos interveñen tres compoñentes do modelo de Conway: o eu operativo, o coñecemento autobiográfico e a memoria episódica.

Os recordos autobiográficos poden formarse a partir de dúas vías: recuperación xenerativa e recuperación directa. A principal diferenza entre ambas é que a primeira é activada polo eu operativo, e a segunda, por unha clave de recuperación.
\begin{itemize}
	\item \textbf{Recuperación xenerativa:} Da lugar á construción de recordos voluntarios. É un 
	proceso iterativo de tres fases (búsqueda, avaliación e elaboración). Comeza sempre coa 
	detección dunha clave (externa ou interna) que activa unha parte da base de coñecemento 
	autobiográfico. Logo iníciase un proceso de búsqueda que proporciona un resultado, e este é 
	avaliado segundo certos criterios. Se o coñecemento recuperado é consistente con ditos 
	criterios, remata o proceso; senon, iníciase de novo todo o ciclo. O proceso iníciase cada vez 
	cunha clave diferente, ou coa clave anterior modificada. Polo tanto, a recuperación xenerativa
	é un proceso cíclico que implica a localización e recuperación dos recordos por aproximacións
	sucesivas.
	\item \textbf{Recuperación directa:} Da lugar á construción de recordos involuntarios. É
	iniciado por unha clave suficientemente distintiva como para activar un proceso de construción
	automática dunha representación mnemónica a través da propagación da activación por unha rede
	asociativa de memoria autobiográfica.
	
	Neisser (1967) interpreta os recordos involuntarios coa súa <<hipótese da reaparición>>, que
	explica que nalgunhas circunstancias poden crearse recordos que logo reaparecen exactamente da 
	mesma forma (recordos de destello, \textit{flashbacks} do TEPT). Este tipo de recordo difire 
	moito da visión reconstrutiva da memoria normal.
	
	Por outra banda, Berntsen e Rubin (2008) propoñen que os recordos involuntarios seguen o mesmo 
	curso que os recordos intrusivos en xeral, e que se producen en toda a poboación, non só nos 
	pacientes de TEPT. Propoñen que son máis accesibles porque:
	\begin{enumerate}
		\item Son recentes.
		\item Son moi emocionantes.
		\item É máis probable que se desenvolvan para eventos positivos.
		\item Amosan un pico de reminiscencia, sobre todo en persoas de máis idade.
	\end{enumerate}
	
	As investigacións conclúen que os \textit{flashbacks} e demais recordos intrusivos non requiren 
	dun tipo especial de memoria: posúen as mesmas características que os recordos recurrentes 
	normais e presentan os mesmos principios básicos de tódolos tipos de memoria autobiográfica.
\end{itemize}

\subsection{Recordos definidores do eu}
Son recordos que están fortemente ligados ó eu a través dunha profunda significación motivacional e emocional para a súa propia vida. Son recordos que nos definen, e nos din a nós mesmos e ós demais quen somos. Posúen catro características básicas:
\begin{enumerate}
	\item \underline{Vividez e intensidade afectiva}: Estes recordos teñen unha forte cualidade
	sensorial, xeralmente visual, e tenden a aparecen na conciencia coa claridade dunha experiencia
	actual.
	\item \underline{Niveis altos de repaso}: As persoas recuperamos constantemente estes recordos,
	porque funcionan como fontes informativas e serven de apoio e de guía en momentos difíciles.
	\item \underline{Vinculación con recordos semellantes}: Como son recordos que capturan aspectos 
	característicos e significativos da autocomprensión dos individuos, é moi probable que se
	conecten a redes de recordos relacionados que comparten obxectivos, intereses, logros e
	respostas afectivas semellantes.
	\item \underline{Intereses permanentes ou conflitos non resoltos}: Son recordos moi intensos
	porque reflexan áreas centrais e antigas de intereses ou conflitos da personalidade, temas
	intemporais que lle dan sentido á identidade dos individuos.
\end{enumerate}

\subsection{Recordos de destello ou memorias fotográficas}
Brown e Kulik (1977) acuñan este termo para referirse a recordos claros, detallados e persistentes de situacións impactantes ou extremas. Propoñen a existencia dun proceso separado (mecanismo de impresión inmediata) que nas condicións apropiadas realiza rexistros de memoria moi precisos, levando á reprodución case fotográfica dos eventos e os seus contextos.

Estes recordos caraterízanse por ser moi duradeiros, precisos, concretos e vivos. Con todo, crese que a súa exactitude depende da medida en que a persoa se vira afectada polo evento. Cuestiónase se realmente dependen dun proceso de memoria separado, porque existen varias explicacións para que estes recordos sexan máis nítidos que os demais:

\begin{enumerate}
	\item Son moi distintivos; é difícil confundilos con outros eventos, o cal non ocorre para a 			maioría de recordos cotiáns.
	\item Como nos impactan tanto, tendemos a falar deles e a velos nos medios de comunicación, o 
	cal contribúe ó seu repaso.
	\item Tenden a ser importantes, e poden chegar a cambiar algo nas nosas vidas.
	\item Posúen un compoñente de activación emocional. As fortes emocións provocadas polos eventos 
	aumentan a capacidade de recordar minuciosamente os seus detalles.
\end{enumerate}

\subsection{Factores sociais e emocionais}
As persoas tendemos a construir os recordos. Por este motivo, a nosa memoria (tanto xeral como autobiográfica) pode verse influenciada polas nosas esperanzas e necesidades: a tendencia da especie humana a ser o centro de atención probablemente inflúe nos nosos recordos, axudándonos a manter a autoestima. Olvidamos de forma selectiva os fracasos, centrándonos máis no éxito e nos eloxios. 

Aínda que esta tendencia poida parecer patolóxica, crese que a súa función é útil. Obsérvase que os pacientes depresivos, quen recordan diferencialmente máis episodios negativos, presentan unha tendencia a describir os recordos autobiográficos de forma moito menos detallada que as persoas non depresivas. Crese que este é un mecanismo de defensa que evita que tódolos recordos sexan tan ``penosos''.

\subsection{Recordos recuperados}
A teoría freudiana propón que o \textit{ego} se defende da ansiedade mediante a represión, bloqueando os recordos negativos e potencialmente ameazantes. Aínda que esta teoría se desestima en moitos ámbitos, neste sentido influíu en moitas situacións clínicas, como é o caso do abuso infantil.

Algúns terapeutas afirman que os abusos na infancia poden desencadear problemas psicolóxicos e emocionais na idade adulta. Suxiren que a cura para estes traumas é sacar á luz o recordo reprimido, aínda que na maioría dos casos nin sequera se sabe se o abuso foi real. Recomendan fixarse nos síntomas: baixa autoestima, pensamentos suicidas ou autodestrutivos, depresión, disfunción sexual... O problema é que estes síntomas, por desgracia, non son raros, polo que non poderían confirmar dito abuso.

\subsection{Síndrome dos falsos recordos}
É o termo que se aplica a aqueles casos (sobre todo de abuso infantil) nos que se induce a un suxeito a que recorde e chegue a convencerse dun evento que en realidade non ocorreu. Isto é relativamente sinxelo cando as persoas son suxestionables. Tanto é así que un promedio do 37\% das persoas que participan en investigacións sobre falsos recordos déixanse persuadir, e incluso xeran recordos bastante vívidos e detallados.

A memoria é potencialmente maleable, sobre todo en situacións de alta suxestibilidade, como pode ser na relación entre terapeuta e paciente. O problema destaca sobre todo nos casos de abuso infantil.

\subsection{Trastorno de estrés postraumático}
O TEPT é o conxunto de síntomas que poden derivar de situacións de estrés extremo (violación, accidentes de tráfico, guerras...). Os que o padecen adoitan sufrir \textit{flashbacks} da situación estresante, pesadelos e un estado de ansiedade xeral.

En canto ós \textit{flashbacks}, Brewin (2001) suxire que poden darse grazas a unha <<memoria accesible situacionalmente>>, que evocaría elevados niveis de detalle, pero que non poderían traerse á mente de forma consciente. 

Non se sabe que mecanismo está detrás do TEPT, pero crese que podería ser o condicionamento clásico: un estímulo contextual da situación estresante actuaría como EC posteriormente, evocando a emoción á que foi asociado, e esta á súa vez evocaría o recordo. Neste sentido, algúns tratamentos de TEPT empregan a extinción da resposta de medo facendo que os pacientes revivan mentalmente a escea, en condicións controladas. 

En moitos casos, dado un nivel de estrés equivalente, algunhas persoas desenvolven TEPT e outras non. Consideramos dúas teorías explicativas deste fenómeno:
\begin{itemize}
	\item \underline{Resposta do SNA}: Ante unha situación ameazadora, a amígdala indícalle ó SNA 
	que libere adrenalina e cortisol, hormonas que poñen en alerta ó organismo. Cando pasa o perigo, 
	as glándulas adrenais reciben un sinal para deixar de producir as hormonas e deixar que o 
	organismo volte, gradualmente, ó estado normal.
	
	Crese que nos pacientes con TEPT este proceso correctivo se reduce, o que desencadea períodos de
	estrés máis prolongados. Crese que o tratamento con propanolol, que acelera este proceso, 
	podería reducir o impacto dos recordos asociados.
	\item \underline{Volume hipocampal}: Os pacientes con TEPT posúen un hipocampo algo menor. Crese
	que un hipocampo reducido fai que os pacientes sexan máis vulnerables a sufrir este trastorno,
	posiblemente pola dificultade desta estrutura para recuperarse tras un gran aumento de
	adrenalina asociado ó estrés extremo.
\end{itemize}

\section{Amnesia psicóxena}
Os síntomas da amnesia psicóxena adoitan incluir a perda case total dos recordos. Normalmente consérvanse o coñecemento semántico xeral e a intelixencia, así como algúns trazos de memoria autobiográfica, pero a nova aprendizaxe vese moi afectada.

\begin{itemize}
	\item \textbf{Fuga:} Perda repentina de memoria autobiográfica, xeralmente acompañada por 
	deambulacións, que dura entre horas e días. Cando se recupera a memoria, o paciente non recorda 
	o período de fuga. Principais características:
	\begin{enumerate}
		\item Adoita vir precedida de estrés, e é máis común en situacións de guerra.
		\item É congruente cun estado de depresión.
		\item Adoita existir unha historia de amnesia de base xenética.
		\item É difícil descartar que exista un móbil oculto para sufrila (moitos pacientes 
		fínxena).
	\end{enumerate}
	\item \textbf{Amnesia focal retrógrada:} Perda do acceso ós recordos adquiridos antes dun 
	trauma. Aparece normalmente en pacientes que tamén sofren amnesia anterógrada. O patrón oposto 
	(amnesia retrógrada sen amnesia anterógrada) non é común, pero existe: os pacientes son 
	incapaces de lembrar o seu pasado de forma explícita, pero poden aprendelo grazas ós demais e, 
	posteriormente, recordar o aprendido sobre si mesmos.
	\item \textbf{Amnesia específica dunha situación:} A amnesia é máis común cando se asocia a 
	situacións extremas e violentas, e aumenta conforme se incrementa a violencia do evento. Tamén 
	se da comunmente cando o alcohol é un factor da situación, probablemente porque esta substancia 
	causa fallos na codificación e recuperación dependentes do estado. É dicir, o que se codifica 
	ebrio recupérase mellor no mesmo estado. 
	\item \textbf{Trastorno de personalidade múltiple:} Consiste na exhibición, por parte dun 			
	paciente, de dúas ou máis personalidades. Kopelman indica que é un trastorno raro, en tanto que 
	non se distribúe uniformemente no mundo. Crese que isto pode deberse a que moitos síntomas deste 
	trastorno se ``poñen de moda'', e se fomentan nos pacientes mediante o reforzo de condutas (do 
	mesmo modo que ocorría antigamente coa catatonia na esquizofrenia, por exemplo).

	As diferentes e múltiples personalidades dun suxeito poden ser ou non mutuamente conscientes. 
	Cando non o son, pode demostrarse unha memoria implícita común (presentando unhas palabras a unha 
	personalidade e pedindo unha compleción de raíces a outra, por exemplo), pero o recordo explícito 
	entre personalidades parece estar ausente.

	Non se sabe claramente que produce este trastorno. Kopelman opina que é posible que os pacientes 
	poidan estar probando un novo modo de vida.
\end{itemize}

\section{Déficits con base orgánica na memoria autobiográfica}
Resultado dun dano cerebral específico. Difiren da amnesia psicóxena en que a identidade persoal non adoita perderse, mentres que os problemas de orientación no tempo e no espazo son moi comúns. O resultado pode ser, por exemplo, o continuo plantexamento das mesmas preguntas.

Aínda que a memoria autobiográfica é unha área de investigación relativamente recente, os estudos de neuroimaxe comezan a revelar as súas bases anatómicas.

\section{Memoria autobiográfica e cerebro}
\subsection{Estudos neuropsicolóxicos}
\begin{itemize}
	\item \textbf{Distinción semántico-episódica:} A amnesia retrógrada orgánica adoita causar a
	perda tanto de recordos episódicos específicos como de recordos autobiográficos semánticos, pero
	non sempre; a veces só afecta a un de ambos.
	\item \textbf{Fabulación:} Dase cando a información autobiográfica é falsa pero non
	intencionalmente enganosa. Tende a ser temporal e, a miúdo, incoherente. Diferenciamos dous 
	tipos:
	\begin{itemize}
		\item \underline{Fabulación inducida}: Resultado dun intento do paciente amnésico de encher
		lagoas de coñecemento, coa intención de evitar situacións embarazosas.
		\item \underline{Fabulación espontánea}: Está ligada ó dano no lóbulo frontal, coma no caso
		de pacientes con síndrome disexecutivo. A lesión interfire coa memoria autobiográfica de 
		dous xeitos:
		\begin{enumerate}
			\item[1.] Xera dificultades para establecer claves de recuperación apropiadas.
			\item[2.] Provoca que información inverosímil poida ser xerada e aceptada por estes
			pacientes.
		\end{enumerate}
	\end{itemize}
	\item \textbf{Delirios:} Son crenzas falsas dun paciente sobre o mundo e sobre si mesmo, que
	resultan inverosímiles para os demais. Danse con máis frecuencia en pacientes esquizofrénicos. A 
	súa base orgánica non está clara, e non hai evidencias de que os pacientes con delirios sexan 
	menos competentes. 
	
	Os delirios poden ser de natureza fantástica, aínda que a súa natureza é habitualmente paranoica 
	(persoas que cren que as súas mentes e accións son controladas externamente, por exemplo). Uns 
	poucos son positivos (xente que se cre deus). En xeral parecen ser formas de explicar 
	experiencias extraordinarias, que levan ós pacientes a crear versións modificadas do seu mundo, 
	de xeito que ditas experiencias adquiran sentido.
\end{itemize}

\subsection{Base anatómica da memoria autobiográfica}
Unha análise xeral recente establece unha distinción entre aspectos episódicos e semánticos da memoria autobiográfica, e destaca unha tendencia xeral a que a emoción cambie o equilibrio da activación entre os hemisferios dereito e esquerdo.

En xeral, a recuperación autobiográfica leva a unha maior activación da amígdala, relacionada coa emoción; do hipocampo, relacionado coa memoria episódica; e do xiro frontal inferior dereito, relacionado co procesamento autorreferencial.

\end{document} %Non pode haber nada escrito despois desta instrución.
