

\documentclass[a4paper,11pt]{article} 
\usepackage[spanish]{babel}           
\usepackage[utf8]{inputenc}           

\usepackage[T1]{fontenc}   		   % Fonte por defecto.
\usepackage{graphicx, subfigure}    		   % Engadir imaxes.
\usepackage{color}      		   % Uso de cores.
\usepackage{anysize}     		   % Modificar o tamaño dos marxes.
\usepackage{multicol, multirow}    % Escribir a doble, triple...columna.
\usepackage{bm}          		   % Letras gregas en negriña.
\usepackage{textcomp}    		   % Símbolos, poden consultarse na rede.
\usepackage{eurosym}     		   % Símbolo € (\euro).
\usepackage{amsthm}                % Paquete da AMS para escribir teoremas.
\usepackage{amsmath,amsfonts}      %Paquetes específicos de símbolos.
\usepackage{lineno}                % Numerar as liñas. 

\marginsize{2cm}{2cm}{1.5cm}{1.5cm} % MARXES: Esq, der, sup, inf.
\parindent=0mm                        % Sangría. 
\parskip=2mm                          % Espazo entre párrafos.
\renewcommand{\baselinestretch}{1}    % Interliñado.
\renewcommand{\spanishtablename}{Táboa} 


\title{Tema 8: Falsos recordos}
\date{}


\begin{document}  

\maketitle

Os falsos recordos ou memorias falsas consisten en recoñecer eventos que en realidade non ocorreron, ou en recordalos de forma diferente. Poden darse en individuos de calqueira idade (aínda que se dan máis facilmente en nenos), e engloban tanto sucesos triviais como verdadeiramente traumáticos. 

\section{Orixes da investigación sobre falsos recordos}
As primeiras teorías sobre as falsas memorias aparecen nos inicios do século XX. Destacamos as máis importantes:
\begin{itemize}
	\item \underline{Teoría de Semon (1904)}: Acentúa a importancia de diferenciar os procesos de
	\textit{engrafía} (codificación) dos de \textit{ecforia} (recuperación). Explica que en todo 
	acto de codificación o contexto activa a recuperación de información previa, o que provoca que 
	novo \textit{engrama} non sexa unha copia literal da realidade, senon unha suma de información 
	nova e recuperada. Todo o que entra na memoria vese necesariamente distorsionado, e cada vez que 
	se recupera un evento, este volve a recodificarse na memoria.
	\item \underline{Teoría de Janet (1928)}: Distingue entre <<memoria traumática>> e <<memoria
	narrativa>>. Considera que as experiencias novas son codificadas en función dos coñecementos
	previos do suxeito e incorporadas ás súas estruturas cognitivas, polo que a súa posterior
	recuperación é unha combinación de coñecementos vellos e novos. Con todo, mentres que os 
	recordos narrativos (ordinarios) tenden a distorsionarse e falsificarse mediante a 
	reconstrución, os traumáticos mantéñense intactos durante longos períodos temporais. 
	\item \underline{Traballo de Bartlett}: No seu libro \textit{Remembering} (1932), Bartlett 
	afirma que no recordo de eventos as persoas tendemos a encher as nosas lagoas perceptivas con 			información obtida en situacións previas e semellantes. A partir dos seus experimentos sobre
	percepción e recordo, Bartlett conclúe que tanto percibir como recordar son procesos 
	construtivos.
	
	Deseña experimentos nos que presenta relatos ós suxeitos para que os recorden. Observa que os
	informes diminúen a medida que aumenta o tempo transcorrido entre a fase de retención e a de 
	recordo: os participantes omiten detalles que non encaixan coas súas expectativas, cambian
	palabras ou nomes descoñecidos por outros máis familiares, alteran a orde dos acontecementos e
	distorsionan o contido da historia para adptalo ás súas experiencias culturais. Bartlett conclúe 
	que o recordo é un proceso esquemático, en tanto que o suxeito adapta o material que se lle 
	presenta para que sexa compatible cos seus modelos de coñecemento preexistentes. Así, o que se 
	retén na memoria é unha versión esquematizada e distorsionada do material orixinal.
	\item \underline{Traballo de Allport e Postman:} Nun intento de atopar o fundamento científico
	dos rumores, os investigadores presentan a imaxe dunha escea a un suxeito. O suxeito describe o
	que ve na imaxe. A continuación preséntaselle esta descrición a un segundo suxeito, que á súa 
	vez debe facer un novo informe para presentarllo ó terceiro participante. O proceso remata coa
	comparación entre o informe do derradeiro suxeito e a imaxe inicial.
	
	Os investigadores atopan que as esceas se van transformando a medida que se transmiten, e que en
	moitas delas se produce sempre un erro característico (por exemplo, que unha navalla pase dunha
	man branca a unha negra). Este tipo de distorsións indican que os prexuizos, as crenzas e unha
	infinidade máis de factores inflúen no recordo. Destes traballos extráese que a memoria humana é
	extremadamente maleable, por ser sensible ós efectos de múltiples factores tanto internos coma
	externos ó suxeito.
\end{itemize}

Unha conclusión importante destes traballos é que os procesos construtivos se levan a cabo durante a codificación, e os reconstrutivos durante a recuperación.

Máis tarde, na década dos setenta, Elisabeth Loftus (de quen xa se falou no apartado de memoria de testemuñas) inicia os seus estudos co <<paradigma da desinformación>>, nos que demostra que a información enganosa presentada despois dun evento altera sistematicamente o recordo do mesmo. Os seus experimentos constan de tres fases:
\begin{itemize}
	\item Os participantes presencian un suceso nun vídeo.
	\item Os suxeitos len un resumo do evento, que conten suxerencias enganosas sobre os detalles do
	mesmo.
	\item Pídese ós participantes que narren o que viron no vídeo da primeira fase. 
\end{itemize}

A información suxerida despois da observación do evento cambia o recordo das testemuñas, sen que se detecte a influencia desta información. Mediante o <<efecto da desinformación>>, as persoas chegamos a crer que vimos e/ou sentimos cousas que non nos sucederon. 

En relación con isto, é necesario comentar que existen unha serie de variables que parecen influir no grao de suxestión das persoas. A suxestionabilidade aumenta a medida que o fai o intervalo entre
o suceso e o recordo do mesmo. A autoridade percibida na persoa que fai as suxerencias enganosas tamén incrementa o grao de suxestión. A simple repetición das suxerencias falsas aumenta o seu efecto, e as persoas son máis suxestionables cando as suxerencias lles resultan plausibles.

\section{Monitorización e control das fontes da memoria}
Sabemos que o recordo da información sofre algún cambio cada vez que se recupera. É importante saber, ademais, que os recordos fórmanse a través de dúas fontes, unha externa (percepción) e unha interna (imaxinación).

Normalmente é fácil discernir se os recordos son reais ou se son froito dunha fantasía. Marcia Johnson fala de <<control de realidade>> para referirse ós procesos que permiten diferenciar os recordos de fontes internas dos de fontes externas. Así, Johnson e Raye (1981) propoñen un modelo que explica que toda pegada de memoria contén distintos atributos (contextuais, sensoriais, semánticos e operacións cognitivas) que caracterizan os recordos. En concreto, os recordos externos posúen máis características sensoriais, contextuais e semánticas, mentres que os internos presentan máis información sobre operacións cognitivas. 

En base a estes supostos, as autoras consideran que os procesos de control de realidade implican dous tipos de xuizos, un inconsciente e outro consciente. Para recoñecer a orixe dun recordo, primeiro iniciaríase un proceso automático e inconsciente de análise dos atributos de cada pegada de memoria, que desencadearía unha análise consciente da decisión tomada polo proceso. Cando este proceso falla (o que é relativamente frecuente), incíciase un novo proceso consciente de razoamento, que ten en conta tanto as características do recordo como as suposicións da metamemoria. 

Porén, pode ocorrer que unha pegada de memoria non presente as características propias da súa clase. Neste caso, iníciase un proceso que <<monitarización das fontes>>, que implica discernir entre fontes internas, entre fontes externas e entre fontes internas e externas (as últimas son os procesos de control de realidade).

Desta investigación pode concluirse que a maioría das distorsións de memoria (excepto quizais os erros de omisión) son resultado dos fallos nos procesos de monitorización das fontes da memoria. Isto tamén explicaría por que se observan tantos fallos deste tipo en enfermidades mentais que presentan alucinacións e obsesións (esquizofrenia, demencia, delirios, etc.).

\section{Creación de memorias falsas: paradigmas e mecanismos}
\subsection{O paradigma de Desee-Roediger-McDermott}
Nos experimentos con este paradigma preséntanse listas de palabras que inducen ós suxeitos a recordar unha palabra crítica, que non se presenta na lista pero é suxerida por ela. Tras a presentación da lista experimental realízase unha proba de recordo libre inmediato, na que os suxeitos deben recuperar o maior número posible de palabras que crean ter escoitado/visto durante a fase de adestramento. Nesta fase, moitos suxeitos recordan erroneamente a palabra crítica.

Logo disto, os suxeitos levan a cabo unha tarefa distractora (resolución de operacións aritméticas sinxelas). Unha vez finalizada a tarefa, realízase unha proba de recoñecemento de palabras. Nela preséntase unha nova lista que contén palabras da lista de estudo inicial. Os suxeitos deben emitir xuízos sobre estas palabras, distinguindo entre as que non estaban na lista experimental, as que recordan que estaban (poden recuperar o momento no que se lles presentaron) e as que saben que estaban (están seguros de que a palabra estaban a lista pero non lembran o momento da súa presentación).

Obsérvase que os acertos na proba de recordo libre melloran a execución na posterior proba de recoñecemento. Porén, ocorre algo semellante cos erros, o que fai que aumente o recordo de palabras críticas (ou falsas alarmas) na proba de recoñecemento. 

Os efectos de falsa memoria atopados nestes estudos son robustos, tanto en recordo coma en coñecemento: as respostas falsas emítense cun alto grao de confianza nas probas de recordo libre, e van acompañadas de xuízos <<recordar>> na proba de recoñecemento. 

Existen dúas posibles explicacións teóricas para a existencia de falsos recordos:
\begin{itemize}
	\item \textbf{Teoría da pegada borrosa:} A presentación da lista de palabras asociadas activaría
	unha representación ou significado xenérico das mesmas, que se recordaría nas fases posteriores.
	\item \textbf{Proposta mixta:} Combina os marcos teóricos da propagación da activación e dos
	procesos de control da fonte. Propón que os falsos recordos se producirían como resultado da 
	activación repetida da palabra crítica durante a codificación e dun fallo nos procesos de 
	control da fonte (cando/onde/como se adquiriu o recordo) no momento da recuperación.
\end{itemize}

Deste paradigma extráese que os recordos poden alterarse con relativa facilidade: a memoria dos individuos é extremadamente suxestionable.

\subsection{Creación de memorias falsas de eventos complexos}
\begin{itemize}
	\item \underline{Implantación de recordos autobiográficos falsos}: Unha investigación destacada
	en relación con este tema é a de Loftus e Pickrell (1995), quen implantaron en adultos o recordo
	falso de terse perdido, sendo nenos, nun centro comercial. Aínda que o nivel de claridade e a
	lonxitude das descricións foron superiores nas memorias verdadeiras, as falsas resultaron moi
	convincentes. 
	
	En contra da opinión de Loftus, máis tarde demostrouse que a implantación de recordos falsos é
	limitada e deben cumplirse dúas condicións para lograla: que os suxeitos almacenen na súa
	memoria información relevante sobre os falsos eventos e que ditos eventos lles resulten
	plausibles. A plausibilidade dos eventos depende das características individuais dos suxeitos
	(grao de suxestionabilidade, imaxinación, capacidade de formación e manipulación de imaxes
	mentais, etc.). Neste sentido, pode dicirse que algúns eventos son máis sinxelos de implantar
	que outros.
	\item \underline{O poder da imaxinación}: Cando se imaxina algo, aínda que se sepa de antemán
	que é ficticio, aumenta a probabilidade de creación dun recordo falso sobre ese algo, así como a
	confianza na súa veracidade. Habitualmente temos <<pensamentos contrafácticos>> (escenarios
	hipotéticos alternativos ó que podería ou debería ter ocorrido). 
	
	En ocasións, un único acto de imaxinación dun evento hipotético aumenta a confianza subxectiva
	de que dito evento si ocorreu na realidade. A este fenómeno chámaselle <<inflación da
	imaxinación>>, e tentouse explicar argumentando que imaxinar eventos aumenta tanto a viveza como
	a familiaridade dos mesmos. O aumento destas características produciría unha atribución errónea
	e unha idea de que ditas sensacións son xeradas por un recordo real. Isto é, a inflación da
	imaxinación produciría un fallo nos procesos de monitorización das fontes que levaría a atribuir
	os factores de viveza e familiaridade a unha fonte equivocada.
	\item \underline{Diferenzas individuais na implantación de recordos falsos}: Hyman e Billings
	(1998) obtiveron que a propensión á creación de recordos falsos a partir de información falsa
	podería relacionarse coa capacidade de crear imaxes mentais e de asumir e elaborar suxerencias
	falsas, así como con certa dificultade nos procesos de control de realidade. Tamén poderían
	influir, ademais dalgúns dos factores comentados anteriormente, as características da
	personalidade dos suxeitos.
\end{itemize}

\section{Base funcional da memoria construtiva}
Os recordos distorsiónanse pola infuenza de tres grupos de factores: o coñecemento e as crenzas individuais, os pensamentos e sentimentos produto da imaxinación, dos desexos e dos soños de cada un e as suxerencias e insinuacións externas.

Como mencionamos anteriormente, a memoria humana constrúese con procesos de codificación e recuperación, que se encargan de mesturar e xuntar información externa e interna ó suxeito. Isto contribúe a que se produzan erros de memoria, que ademais do valor negativo que xa comentamos, tamén posúen un gran valor positivo, en tanto que se cre que surxen de procesos adaptativos e posibilitan o importante papel da memoria na planificación do futuro.

A memoria almacena recordos de experiencias pasadas, que serán útiles (nun contexto adaptativo) sempre e cando nos permitan predicir as situacións futuras. Con todo, os sucesos futuros nunca son réplicas exactas dos acontecementos pasados; por tanto, podería considerarse que un sistema de memoria que fixera réplicas exactas dos recordos non sería verdadeiramente útil para predicir o futuro. Pola contra, unha memoria construtiva (como a nosa) si sería de gran utilidade para esta tarefa. Aínda que este sistema implicaría a existencia de erros de memoria, tería a importante vantaxe adaptativa de permitir simular ou imaxinar o futuro. 

En definitiva, a memoria humana é un sistema neurocognitivo que codifica, almacena e recupera a información a través de procesos construtivos e reconstrutivos, e como consecuencia disto é proclive ós erros; porén, grazas a esta flexibilidade permítenos tanto recordar o pasado como imaxinar o futuro.

Schacter (1999) fala da dualidade da memoria, situando, por unha banda, a súa gran influenza e utilidade nas nosas vidas, e por outro, as súas limitacións. Organiza e clasifica estas transgresións segundo sete pecados básicos:

\begin{itemize}
	\item \underline{Erros de omisión}: Implican diferentes tipos de olvido.
	\begin{itemize}
		\item \textbf{Transitoriedade:} Diminución crecente da accesibilidade á información co 
		paso do tempo. O transcurso do tempo debilita os recordos, producindo un olvido máis rápido
		nas horas ou días inmediatamente posteriores á adquisición dun recordo, e de seguido, unha
		estabilización. Isto ocorre porque, como sabemos, a información retida na MCP é transitoria
		por definición, pero unha vez transferida á MLP e consolidada faise duradeira.
		
		En relación con este erro aparece a <<nova teoría do desuso>> (1992), que postula que a
		inhibición (ou pérdida de acceso) da recuperación é unha consecuencia do desuso da
		información. Esta teoría contén varios supostos básicos:
		\begin{enumerate}
			\item[1.] Os ítems da memoria están representados por dous tipos de forzas, a \textit{
			forza de almacenamento}, que se refire a como un ítem se relaciona ou se asocia con 
			outros, e a \textit{forza de recuperación}, que se refire á facilidade momentánea de
			acceso a un ítem.
			\item[2.] A probabilidade de que un ítem se recupere depende totalmente da súa forza de
			recuperación, e é independente da súa forza de almacenamento.
			\item[3.] A recuperación dun ítem aumenta a súa forza de recuperación, e o estudio do
			mesmo aumenta a súa forza de almacenamento.
			\item[4.] O desuso da información afecta á súa forza de recuperación, pero non á súa
			forza de almacenamento.
		\end{enumerate}
		
		Así, a teoría do desuso indica que a información non recuperada farase progresivamente menos
		accesible, pero non se eliminará da memoria. Pola contra, Schacter plantexa que os recordos
		que non se recuperan poderían disiparse co paso do tempo.
		\item \textbf{Distractibilidade ou despistes:} Fallos de atención durante o proceso de 
		codificación que impiden codificar a información profundamente. Normalmente danse cando se
		intenta facer varias cousas á vez e non se presta a suficiente atención a unha delas. 
		
		As accións que poden levarse a cabo eficazmente con pouca (ou ningunha) atención denomínanse
		automáticas. Son vantaxosas en tanto que permiten levar a cabo varias tarefas á vez, pero
		provocan a <<amnesia do automático>>, que consiste na ausencia de recordo das accións
		realizadas de forma automática.
		\item \textbf{Bloqueo:} Inaccesibilidade temporal de información almacenada na memoria. O
		caso máis interesante e máis estudado en relación con este erro de memoria é o fenómeno da
		``punta da lingua'', que é a incapacidade de producir unha palabra ou un nome aínda sabendo
		con certeza que este está dispoñible na memoria. Os erros de bloqueo aumentan coa idade, o
		que explicaría a dificultade dos anciáns para lembrar, por exemplo, o nome dos seus netos.
	\end{itemize}
	\item \underline{Erros de comisión}: Implican diferentes tipos de distorsións de memoria.
	\begin{itemize}
		\item \textbf{Atribución errónea:} Asignación dun recordo a unha fonte equivocada.
		Considéranse tres formas de atribución errónea:
		\begin{enumerate}
			\item[1.] Recordo correcto da información pero equivocación na atribución da fonte. Este
			tipo de erros tamén son frecuentes nas persoas de idade avanzada, e ocorren como
			consecuencia de fallos nos procesos de monitorización da fonte.
			\item[2.] Ausencia absoluta de conciencia de recordo, que consiste na atribución errónea
			de información a un mesmo (un pensamento, unha idea) cando esta se recuperou, de forma
			inconsciente, dunha experiencia previa. Un bó exemplo deste fenómeno sería o do <<plaxio
			involuntario>>.
			\item[3.] Recordo ou recoñecemento falso dun evento que en realidade nunca se
			experimentou, do que xa falamos ampliamente ó longo deste resumo.
		\end{enumerate}
		\item \textbf{Suxestionabilidade:} Implantación de falsos recordos mediante preguntas ou 
		comentarios suxestivos de axentes externos ó individuo durante a recuperación dos eventos.
		Relaciónase coa atribución errónea no sentido de que a conversión dunha suxerencia nun
		falso recordo implica unha atribución incorrecta. Porén, a atribución errónea non precisa da
		suxestión para darse.
		\item \textbf{Propensión ou sesgo:} Influenza do coñecemento e das crenzas actuais do 
		individuo sobre o seu modo de recordar. O que as persoas saben, cren e sinten nun momento
		dado pode influir e distorsionar profundamente o seu recordo do pasado.
		\item \textbf{Persistencia:} Recordo patolóxico, isto é, incapacidade para olvidar recordos
		dolorosos, sobre todo de episodios traumáticos, que resultan indesexables para quen os
		recorda. Este fenómeno tamén inclúe a rumiación sobre eventos negativos e o recordo de medos
		e fobias, e afecta ó individuo perturbándoo física e psicoloxicamente. 
	\end{itemize}
\end{itemize}

Estes erros ocorren con frecuencia na vida cotiá, e aínda que poden resultar frustrantes para que os sofre, parecen ser útiles (e incluso necesarios) para a supervivencia. Schacter opina que non deben considerarse como defectos no sistema de memoria, senon como subproductos das características desexables da memoria humana. 


\end{document}