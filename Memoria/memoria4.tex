

\documentclass[a4paper,11pt]{article} 
\usepackage[spanish]{babel}           
\usepackage[utf8]{inputenc}           

\usepackage[T1]{fontenc}   		   % Fonte por defecto.
\usepackage{graphicx, subfigure}    		   % Engadir imaxes.
\usepackage{color}      		   % Uso de cores.
\usepackage{anysize}     		   % Modificar o tamaño dos marxes.
\usepackage{multicol, multirow}    % Escribir a doble, triple...columna.
\usepackage{bm}          		   % Letras gregas en negriña.
\usepackage{textcomp}    		   % Símbolos, poden consultarse na rede.
\usepackage{eurosym}     		   % Símbolo € (\euro).
\usepackage{amsthm}                % Paquete da AMS para escribir teoremas.
\usepackage{amsmath,amsfonts}      %Paquetes específicos de símbolos.
\usepackage{lineno}                % Numerar as liñas. 

\marginsize{1.5cm}{1.5cm}{1.5cm}{1.5cm} % MARXES: Esq, der, sup, inf.
\parindent=0mm                        % Sangría. 
\parskip=2mm                          % Espazo entre párrafos.
\renewcommand{\baselinestretch}{1}    % Interliñado.
\renewcommand{\spanishtablename}{Táboa} 


\title{Tema 4: Memoria a longo prazo II. Procesos de recuperación e olvido}
\date{} %Para que non apareza deixa o espaxo entre {} valeiro. Para que apareza a data de hoxe: \today


\begin{document}  

\maketitle 

\section{Dispoñibilidade versus accesibilidade}
Actualmente pénsase que a información almacenada na MLP nunca chega a perderse por completo, excepto por causas físicas ou orgánicas (accidentes ou enfermidades). Pártese de que a recuperación implica ter acceso á información almacenada na memoria. Un fallo na recuperación de información non significa necesariamente que esta se perdera: pode que a información fose codificada e esté almacenada sen que se teña fácil acceso a ela ou se teña só acceso a unha parte. En termos de Tulving, é posible que a información esté dispoñible pero que non sexa accesible. Isto é o que ocorre no fenómeno da ``punta da lingua'', cando se cre que se coñece certa información pero non se é capaz de recordala. Segundo Anderson, a sensación de que se coñece algo adoita ser un bó indicador de que realmente se coñece.  

Un exemplo da disociación entre dispoñibilidade e accesibilidade son as diferenzas no rendemento en probas de recordo e recoñecemento, sendo este superior nas segundas. Estas diferenzas interprétanse en base a que as probas de recoñecemento proporcionan claves ou indicios de recuperación máis eficaces que as de recordo: a presenza dos ítems proporciona potentes claves que os fan máis accesibles. Considéranse dúas interpretacións alternativas deste fenómeno:
\begin{itemize}
	\item \textbf{Teoría tradicional do umbral ou da pegada mnémica:} Existe un único proceso 
	subxacente a ambos procesos, polo que a recuperación dos ítems dependerá de que a forza da súa 
	pegada mnémica acade un umbral crítico que é maior para o recordo que para o recoñecemento. 
	Un problema importante que esta teoría non pode resolver é o da paradoxa da frecuencia, que 
	explica que as palabras que se presentan máis frecuentemente son mellor recordadas que as menos 
	frecuentes, mentres que estas son mellor recoñecidas. 
	\item \textbf{Teoría das dúas fases:} O recordo implica dous procesos, un de xeración de 
	posibles ítems e outro de recoñecemento dos apropiados. Deste xeito, os fallos na xeración dos 
	ítems impedirán que poidan recoñecerse. Para o recoñecemento só se precisaría o segundo destes 
	procesos.
	
	Esta teoría non explica os efectos que ten o contexto sobre a memoria de recoñecemento e a 
	existencia de fallos no recoñecemento de palabras que posteriormente poden recordarse.
\end{itemize}

\section{O proceso de recuperación: claves de recuperación e especificidade da codificación}
Tulving insiste en que para poder recordar ben, o suxeito debe establecer regras para a codificación dos ítems durante a fase de almacenamento e reaplicalas no momento da recuperación. Así, as claves de recuperación serán eficaces se e só se estiveron presentes tamén no momento da codificación. 

Segundo isto, a recuperación é un proceso que transcorre a partir dunha ou máis claves cara unha pegada obxectivo, ben a través dunha vía de conexións asociativas ou ben a través dun proceso de propagación da activación. As pegadas de memoria relaciónanse entre si a través de conexións estruturais, denominadas asociacións, e propagan a súa activación cara outras pegadas coas que están relacionadas. Neste sentido, propagarase máis activación canto máis forte sexa a asociación entre as pegadas. 

O proceso de propagación da activación depende de dous factores:
\begin{itemize}
	\item \underline{Nivel de activación}: É o estado interno dunha pegada de memoria, que reflexa o 
	seu nivel de excitación e determina o seu grao de accesibilidade. Aumenta cando se percibe algo 
	relacionado coa pegada no entorno ou cando se focaliza a atención directamente na pegada. 
	Persiste durante algún tempo antes de desvanecerse.
	\item \underline{Propagación da activación}: É a transmisión automática de ``enerxía'' desde 
	unha pegada a outras a través das asociacións entre elas. A cantidade de activación que se 
	propaga é proporcional á forza das conexións, e propágase en paralelo desde unha clave ata 
	tódolos asociados.
\end{itemize}

\subsection{O fenómeno do fallo no recoñecemento de palabras recordables}
Segundo a teoría dos dous procesos calquera ítem que poida recordarse debería poder recoñecerse, en tanto que o recordo implica unha fase de recoñecemento. Porén, os estudos de Tulving e Thompson (1973) demostran que pode ocorrer o recordo de ítems aínda que se falle no seu recoñecemento. 

Estes investigadores atopan que un indicio de recuperación potente ou asociado forte pode ser inefectivo en elicitar o recordo da palabra obxectivo se só se presenta no momento da proba (pero non na fase de estudo) ou se é diferente do presentado na fase de estudo. Os indicios específicos facilitan o recordo se e só se a información sobre eles e sobre a súa relación coas palabras a recordar se almacena ó mesmo tempo que as palabras correspondentes. 

Posteriormente, os investigadores modifican o seu procedemento para asegurar que as palabras obxectivo sexan sometidas ó proceso de decisión. Este novo paradigma experimental consta de catro pasos:
\begin{enumerate}
	\item Fase de estudo na que se presenta unha lista na que cada unha das palabras obxectivo é 
	acompañada por un asociado débil ou de baixa frecuencia. 
	\item Fase de libre asociación na que se lles presenta ó suxeitos unha lista que inclúe 
	asociados fortes das palabras obxectivo, e se lles pide que digan palabras nas que estes lles 
	fan pensar. Entre elas, os suxeitos inclúen unha gran porcentaxe das palabras obxectivo 
	orixinais. 
	\item Pídeselles ós suxeitos que indiquen se entre as palabras xeradas por eles mesmos hai 
	algunha pertencente á lista inicial. O recoñecemento das palabras obxectivo orixinais é moi 
	baixo.
	\item Proba de recordo na que os asociados débiles da lista de estudo inicial son presentados 
	como indicios de recuperación das palabras obxectivo iniciais. Moitas palabras que os suxeitos 
	fallan en recoñecer na fase anterior son recordados nesta. 
\end{enumerate}

\subsection{Xurdimento e desenvolvemento do principio de especificidade da codificación}
Dúas ideas fundamentais inciden no planteamento de Tulving do principio:
\begin{itemize}
	\item \textbf{Efectividade dos indicios de recuperación:} Un indicio será efectivo se e só se a 
	palabra a recordar é codificada especificamente respecto a ese indicio no momento da 
	aprendizaxe.
	\item \textbf{Dualidade conceptual entre memoria episódica e semántica:} Considéranse dous 
	sistemas separados pero interconectados. A memoria semántica fai referencia ó almacenamento e 
	utilización de coñecementos xerais sobre o mundo. A memoria episódica refírese ó almacenamento e 
	recuperación de acontecementos localizados temporal e espacialmente e experimentados
	personalmente, e ás relacións espazo-temporais entre os mesmos.
	
	Esta hipótese permite resolver unha contradicción: por un lado, o carácter circunstancial da 
	pegada mnémica (enfatízase a especificidade da codificación en cada situación particular, frente 
	ó suposto de que cada ítem ten só unha representación na memoria); e por outro, a interpretación 
	da MLP como unha complexa rede semántica de conceptos e relacións asociativas.
\end{itemize}

Neste contexto, o principio de especificidade da codificación surxe como un intento de facilitar a comprensión de tódolos fenómenos coñecidos sobre memoria episódica e recuperación. A súa formulación, segundo Tulving e Thompson:

\begin{quote}
	<<As operacións específicas de codificación realizadas sobre aquelo que percibimos determinan 
	o que se almacena e, á súa vez, o que se almacena determina que claves de recuperación son 
	eficaces para acceder ó que está almacenado>>
\end{quote}

Supoñen que almacenamos información sobre a codificación específica da palabra a memorizar no contexto e na situación na que se está memorizando. Polo tanto, a eficacia das claves dependerá das propiedades da pegada mnémica da palabra obxectivo na memoria episódica, que é independente das propiedades semánticas da palabra. Isto obsérvase claramente coas palabras homónimas: se temos que memorizar a palabra \textit{violeta} e a clave que nos dan é \textit{azul}, esta palabra codificarase máis probablemente como ``nome de color'' que como ``nome de flor'' ou ``nome de muller''. Posteriormente, ``nome de color'' será unha clave eficaz para recuperar a palabra, pero as outras dúas non.

Isto tamén é aplicable ó fenómeno do fallo no recoñecemento de palabras recordables: nos experimentos de Tulving, os suxeitos aprenden a palabra \textit{frío} asociada a \textit{chan}. Posteriormente, ó presentarlles a palabra \textit{quente} como indicio, a miúdo xeran a palabra \textit{frío} como asociada, pero non a recoñecen como pertencente á lista inicial. Isto ocorre porque as propiedades codificadas respecto á palabra \textit{frío} no contexto de \textit{chan} non se solapan coas propiedades da mesma palabra no contexto de \textit{quente}. 

En contra de Tulving, algúns investigadores atoparon que nalgunhas circunstancias indicios que non estiveron presentes durante a aprendizaxe poden ser efectivos. Con todo, segue considerándose crucial para a recuperación exitosa a interacción entre a información almacenada na pegada mnémica e a das claves, sexan estas explícitas ou non.

En canto á distinción entre recordo e recoñecemento, Tulving afirma ó principio que son esencialmente un mesmo proceso de recuperación, como consecuencia da interacción entre a información da pegada e dos indicios, aínda que estes son <<indicios copia>> (copias dos estímulos orixinais) no caso do recoñecemento. Posteriormente admite que hai diferenzas nos procesos implicados nestes fenómenos, e que se require unha maior cantidade de solapamento informacional para o éxito no recordo que para o éxito no recoñecemento.

\section{Formas de recuperación: memoria explícita e memoria implícita}
Segundo Schacter (1987) ``memoria explícita'' e ``memoria implícita'' son termos relacionados coa experiencia psicolóxica dunha persoa durante a recuperación, e non se refiren nin implican a existencia de dous sistemas de memoria diferentes. A memoria explícita reflexa a información adquirida de forma consciente, esixindo a recolección consciente de experiencias previas; a implícita reflexa a información adquirida de maneira non consciente, polo que non require recolección consciente. Estas memorias poden disociarse moi nitidamente e existen probas específicas que reflexan o seu contido:

\begin{itemize}
	\item \textbf{Probas de memoria directas ou explícitas:} Son as que esixen expresións
	conscientes de memoria para a recuperación de eventos pasados, sexan estes episodios persoais ou 
	materiais do contexto experimental. Nestas tarefas, o contexto actúa como clave. Consideramos 
	dous tipos segundo a importancia da orde na recuperación da información:
	\begin{itemize}
		\item \underline{Tarefas nas que a orde é importante}: Recordo serial (recórdase a 
		información na orde na que foi presentada) e de pares asociados. A información percíbese de 
		forma visual auditiva, e reprodúcese de forma oral ou escrita. O rendemento de memoria é 
		elevado, polo que diminúe o número de ensaios na reaprendizaxe.
		\item \underline{Tarefas nas que a orde non é importante}: Poden producirse acertos, olvidos 
		ou omisións e erros de comisión ou intrusións. Consideramos tres tipos de tarefas:
		\begin{itemize}
			\item \textbf{Recordo libre:} Os suxeitos reproducen a información que se lles presentou 
			na fase de estudo na orde que prefiren.
			\item \textbf{Recordo con claves:} É unha clase de recordo libre na que se empregan 
			indicios de recuperación, ademais do contexto.
			\item \textbf{Recoñecemento:} Os suxeitos identifican os ítems presentados na fase de 
			estudo de entre un grupo de ítems que contén estes e outros que se denominan 
			distractores. Son importantes nesta proba as falsas alarmas (ítems que se recoñecen 
			aínda que en realidade non foron presentados na fase de estudo), e poden ser 
			interesantes tamén as medidas de latencia (TR) e algúns aspectos cualitativos coma os 
			xuizos recordar-saber.
		\end{itemize}
	\end{itemize}
	\item \textbf{Probas de memoria indirectas ou implícitas:} Non requiren recuperación consciente 
	ou intencional, medindo a influencia das experiencias previas sen que o suxeito as recorde 
	directamente. Os suxeitos realizan actividades cognitivas ou motoras a partir de instrucións 
	que fan referencia á tarefa presente, non a acontecementos anteriores. O contexto non actúa 
	como clave, e a medida destas probas é relativa. Baixo certas condicións, os suxeitos poden 
	levar a cabo eficientemente estas tarefas sen ser quen de recoñecer a mesma información de forma 
	explícita. 
	
	Os tests de memoria implícita divídense en dúas grandes categorías: verbais e non verbais. Á súa 
	vez, cada unha destas subdivídese en varios grupos: 
	\begin{itemize}
		\item \underline{Probas de coñecemento factual, conceptual, léxico e perceptivo}: Miden a 
		exactitude da resposta e/ou a latencia da resposta correcta, non en termos absolutos, senon 
		avaliando o fenómeno de facilitación ou \textit{priming}, isto é, a influenza do material 
		presentado previamente sobre a tarefa.
		\begin{itemize}
			\item \textbf{Factual e conceptual:} Asociación de palabras, produción ou verificación 
			de membros dunha categoría, xuizos de fama...
			\item \textbf{Léxico:} Decisión léxica, completar de palabras a partir de fragmentos 	
			iniciais ou intermedios...
			\item \textbf{Perceptivo:} Identificación taquistoscópica de debuxos ou palabras, 
			decisión de obxectos, xuizos de preferencia estética...
		\end{itemize}		 
		\item \underline{Probas de coñecemento procedimental}: Examínanse os cambios na execución 
		dunha acción como resultado da práctica previa. Poden referirse a habilidades cognitivas 
		(solución de problemas como rompecabezas) ou perceptivo-motoras (debuxar en espello). 
		
		Nestas probas mídese primeiro, en tempo ou en número de ensaios, canto tarda o suxeito en 
		adquirir un determinado nivel de destreza, e como proba de memoria, canto tempo se aforra 
		nun intento posterior grazas a esa experiencia previa.
		\item \underline{Probas de resposta avaliativa}: Baséanse en que a presentación repetida dun 
		estímulo favorece a elaboración posterior de xuizos máis positivos sobre ese material que 
		sobre outro non presentado. Os xuizos poden ser de preferencia ou xuizo afectivo (sobre 
		debuxos ou melodías) ou cognitivos (sobre a credibilidade dunha afirmación). Estas probas 
		requiren a comparación entre os xuizos sobre o material xa presentado e o novo.
		\item \underline{Outras probas de cambio condutual}: Mídense as respostas dos suxeitos nunha 
		fase de exposición inicial e compáranse coas obtidas nunha fase posterior, ante a mesma 
		estimulación e ante estímulos diferentes. Estas probas poden ser cambios de resposta 
		fisiolóxica (condutancia eléctrica da pel, potenciais evocados...), condicionamento e aforro 
		ou reaprendizaxe.
	\end{itemize}
\end{itemize}

\section{Efectos do contexto}
\begin{itemize}
	\item \underline{Memoria dependente do contexto}: Cando se produce un  emparellamento entre o 
	contexto de codificación e o de recuperación, a memoria resulta beneficiada. Nun experimento con 
	buzos, Baddeley e Godden (1975) obtiveron que estes presentaban dificultades no recordo da 
	información aprendida cando este se producía nun contexto físico diferente, pero non cando 
	recordaban no mesmo lugar no que aprenderan. Isto ocorre só para probas de recordo, non para as 
	de recoñecemento.
	
	Máis tarde, Eric Eich (2007) suxeriu que este fenómeno está determinado non tanto pola 
	similitude entre os contextos de codificación e recuperación, senon por como os suxeitos os 
	``sintan''. Defende a <<hipótese da mediación do humor>>, segundo a cal a dependencia do 
	contexto físico é, en realidade, unha dependencia do estado de ánimo. 
	\item \underline{Memoria dependente do estado}: Algunha información recupérase mellor cando a 
	recuperación se produce no mesmo estado farmacolóxico (ou nun semellante) no que se produciu a 
	aprendizaxe. Isto ocorre porque determinados aspectos do estado poden codificarse
	incidentalmente como parte da experiencia episódica, e a reinstauración de dito estado durante a 
	recuperación actuaría como unha clave. Coma o efecto anterior, isto só se atopou en probas de 
	recordo, non de recoñecemento.
	
	O mesmo ocorre co estado de ánimo, independentemente da valencia afectiva do material a 
	recordar: o estado de ánimo presente durante a codificación parece actuar como clave de 
	recuperación durante o recordo. En xeral non é un efecto robusto, pero pode obterse evidencia 
	clara e consistente do mesmo se concurren certas condicións: que os suxeitos experimenten 
	afectos fortes, estables e sinceros, que eles mesmos xeren os eventos a recordar e que a 
	recuperación esté mediada por claves ``invisibles'' producidas polos participantes, non por 
	claves ``observables'' proporcionadas polo experimentador.
	\item \underline{Memoria dependente do contexto cognitivo}: O recordo vese facilitado cando os 
	pensamentos, ideas e conceptos particulares que ocupan a nosa atención durante a fase de 
	codificación coinciden cos da fase de recuperación.
	
	Neste sentido, Marian e Neisser (2000) levan a cabo un estudo con persoas bilingües ruso-inglés, 
	no que lles piden que narren historias sobre as súas vidas en resposta a unhas palabras 
	(claves). Realizan a metade da sesión en ruso e a outra metade en inglés. Atopan que cando a 
	entrevista se desenvolve en ruso, os suxeitos xeran recordos experimentados en fala rusa no 
	64\% das claves, e un 35\% cando a entrevista se desenvolve en inglés. Obteñen o mesmo para os 
	recordos de fala inglesa. Conclúen que o contexto lingüístico actúa como outra forma de contexto 
	incidental, facilitando a recuperación dos recordos na mesma lingua que se codificaron.
\end{itemize}

\section{Memoria reconstrutiva}
É o proceso activo e inferencial de recuperación mediante o cal se enchen os espazos en branco da memoria a partir das experiencias previas, a lóxica ou os obxectivos. Cando a recuperación se ve implicada en recordar algo que está ó límite da accesibilidade, as personas podemos ser capaces de recordar certos aspectos mentres nos vemos obrigadas a imaxinar outros. 

Os estudos sobre memoria reconstrutiva supoñen que existen procesos de recuperación autmáticos cos que a información emerxe sen razón aparente. Tamén explica que cando non ven á mente a información axeitada, empregamos os fragmentos que temos coma se foran pistas para recompoñer os nosos recordos. 

A reconstrución baséase a miúdo no coñecemento previo, que suxire inferencias plausibles. Estas inferencias poden levarnos a cren que estamos recordando algo que en realidade non sucedeu, ou que ocorreu de forma diferente. As inferencias reconstrutivas poden producir erros na memoria que son máis probables a medida que transcorre o tempo desde o evento a recordar, porque diminúe a accesibilidade ó recordo orixinal. 

Con todo, os procesos reconstrutivos son bastante útiles porque axudan a recordar información verdadeira facendo inferencias plausibles sobre os posibles sucesos.

\section{Procesos de olvido: olvido incidental e olvido motivado}
Tulving (1974) define o olvido como a incapacidade para recordar nun momento dado algo que si se recordaba nunha ocasión previa.

No estudo da memoria, a palabra olvido ten dúas acepcións: o borrado completo da pegada de memoria que non pode recordarse ou un fallo de recuperación que pode subsanarse con claves axeitadas. A primeira definición implicaría a perda dunha sinapse no cerebro, pero actualmente non pode comprobarse este fenómeno en ningún organismo. Por tanto, este tipo de olvido debe ser descartado.

A segunda acepción é a que se considera no estudo da memoria. A efectividade das claves pode variar, permitindo recordar nun momento posterior algo que parecía olvidado. Nos casos de información que nunca chega a recordarse, pode ocorrer que dita información non fora codificada de forma axeitada. A información non estaría dispoñible na memoria porque nunca chegou a ser almacenada.

Neste marco teórico, consideramos dous tipos de olvido: o olvido incidental, que se produce en ausencia de intención de olvidar, e o olvido motivado, que se produce como resultado de procesos ou condutas dirixidas a diminuir a accesibilidade de determinados contidos de memoria.

\subsection{Olvido incidental}
Explícase desde varias perspectivas:
\begin{itemize}
	\item \textbf{Teoría do decaemento da pegada:} Considera que o olvido é un proceso pasivo, 
	postulando que os recordos non empregados se desvanecen co paso do tempo. Esta teoría baséase na 
	<<curva do olvido>> de Ebbinghaus (1885), que establece unha relación logarítmica entre memoria 
	e tempo que explica por que o olvido é moi rápido despois da aprendizaxe e logo se enlentece, e 
	na <<lei do desuso>> de Thorndike, que postula que os hábitos, e por ende os recordos, que se 
	usan repetidamente se fortalecen, mentres que os que non se usan debilítanse ata desaparecer. 
	
	Máis actualmente, Bjork propuxo unha ``nova teoría do desuso'' na que considera, desde un punto 
	de vista funcional, que o olvido é un resultado necesario da actualización do sistema de 
	memoria, que inhibe o que non se usa e fortalece o que se repite. Moitos investigadores cren que 
	o decaemento está relacionado coa perda de información da MT verbal e visual, e nos efectos de 
	facilitación por repetición ou familiaridade.
	
	Os que apoian esta teoría baséanse en que tanto o nivel de activación coma os elementos 
	estruturais (asociacións) dos recordos degrádanse co tempo, aínda que o recordo subxacente 
	permaneza intacto. Estas teorías teñen sentido, entanto que somos seres biolóxicos en constante 
	cambio: as neuronas morren e as conexións sinápticas modifícanse.
	
	A pesar destas teorías, os psicólogos experimentais actuais consideran que o tempo por si só non 
	explica o olvido, senon que correlaciona cos factores que o causan. Para demostrar a existencia 
	do decaemento, habería que demostrar que o olvido medra co paso do tempo, o que implicaría 
	manter ó suxeito experimental nun estado de vacío mental, sen repaso, pensamentos ou 
	experiencias que contaminasen o estado do recordo. Polo tanto, é imposible demostrar a 
	existencia deste fenómeno.
	\item \textbf{Teoría da fluctuación contextual:} Explica a curva do olvido baseándose en que os 
	cambios do contexto aumentan co paso do tempo. As persoas experimentan constantemente estímulos, 
	situacións, pensamentos e emocións novos. Por conseguinte, o seu contexto incidental será máis 
	semellante a aquel no que estiveron hai pouco tempo, e perderá similitude co paso do tempo. Esta 
	teoría permite explicar, en parte, o fenómeno da <<amnesia infantil>> (dificultade da maioría 
	das persoas para recordar os seus primeiros anos de vida).
	\item \textbf{Teoría da interferencia:} Propón que o almacenamento de trazos semellantes 
	dificulta a recuperación. En tanto que o número de trazos parecidos se incrementa co paso do 
	tempo, esta teoría proporciona unha explicación directa da curva do olvido. 
	
	A interferencia aumenta cando a clave que permite acceder a unha pegada de memoria obxectivo se 
	asocia a recordos adicionais. Así, a posibilidade de chegar ó recordo a través dunha clave non 
	depende só da forza da asociación entre eles, senon tamén de se a clave se relaciona con outros 
	ítems. Cando unha clave se asocia a varios ítems, estes compiten entre si para acceder á 
	consciencia. Así, o recordo tende a diminuir cando aumenta o número de ítems asociados á mesma 
	clave. Esta idea é respaldada polo <<principio de sobrecarga da clave>>: cando unha clave se 
	asocia a demasiados elementos, a súa capacidade de dar acceso a cada un dos trazos diminúe.
	
	Consideramos dous tipos de interferencia:
	\begin{itemize}
		\item \underline{Retroactiva}: Tendencia a que a información adquirida recentemente impida a 
		recuperación de recordos semellantes máis antigos. Os procesos asociados ó almacenamento de 
		novas experiencias perxudican a habilidade de recuperar algunhas máis lonxanas.
		\item \underline{Proactiva}: Tendencia que teñen os recordos máis antigos a interferir coa 
		recuperación de experiencias e coñecementos máis recentes. Este fenómeno contribúe a 
		determinar a tasa de olvido e é máis pronunciado en probas de recordo que nas de 
		recoñecemento.
	\end{itemize}
	
	Existen unha serie de mecanismos subxacentes á interferencia que explican como se produce:
	\begin{itemize}
		\item \underline{Bloqueo asociativo}: Proceso teórico que explica os efectos da 
		interferencia durante a recuperación, segundo a cal as persoas olvidan os exemplares non 
		practicados das categorías practicadas porque as asociacións cos exemplares practicados 
		dominan a recuperación. É dicir, o recordo dos exemplares practicados bloquea o recordo 
		daqueles cos que non se practicou. En termos máis teóricos, dise que as claves fallan ó 
		evocar as súas pegadas obxectivo ó evocar repetidamente un competidor máis forte, o que leva 
		a que se abandonen os esforzos para recuperar o ítem obxectivo.
		\item \underline{Desaprendizaxe asociativa}: Propón que o vínculo asociativo que conecta un 
		estímulo cun trazo de memoria se debilita cando se recupera erroneamente o trazo ó buscar 
		outro diferente. É dicir, a conexión entre clave e elemento obxectivo vese ``penalizada'' 
		cada vez que o segundo se recupera de forma inapropiada. Como ocorría co olvido permanente, 
		non é posible demostrar este fenómeno.
		\item \underline{Inhibición}: Postula que o olvido surxe, en parte, pola supresión das 
		pegadas de memoria dos competidores. A perspectiva do control inhibitorio suxire un papel 
		activo e adaptativo o olvido, que pode servir para un propósito útil. Este mecanismo
		facilita a recuperación e fai que as recuperacións posteriores da mesma información sexan 
		máis sinxelas, ó reducir a competición futura.
	\end{itemize}
\end{itemize}

\subsection{Olvido motivado}
\subsubsection{Sesgo de positividade}
Atópase un sentimento de benestar xeral na maioría da poboación, pero a miúdo este non obedece ás circunstancias obxectivas da vida das persoas. En tanto que a nosa valoración de como nos foi na vida depende dos recordos, podemos dicir que a nosa memoria contribúe á percepción dese benestar: existe un forte sesgo de positividade sobre o que a maioría das persoas recordan a longo prazo, sendo os recordos positivos máis accesibles que os negativos. Este sesgo aumenta coa idade, e focalízase cada vez máis nos obxectivos emocionais e no mantemento do sentimento de benestar.


Charles, Mather e Carstensen (2003) realizaron un estudo do que concluiron que os sesgos de memoria non son accidentais. Pediron a xóvenes e anciáns que observaran trinta e dúas esceas que incluían imaxes agradables, desagradables e neutras. Quince minutos despois, pedíronlles que lembrasen tódalas imaxes que puideran. Obtiveron que se recordaban mellor as imaxes con contido emocional que as neutras, e que os anciáns recordaban menos que os xóvenes. Ademais, atoparon que o recordo de esceas positivas aumentaba conforme o facía a idade: os xóvenes recordaron esceas positivas e negativas por igual, mentres que os anciáns recordaban dúas veces máis esceas positivas que negativas. 

Os sesgos emocionais tamén se observan con palabras e caras. Nun estudo posterior, Mather e Carstensen (2005) plantean que a medida que a xente medra e ve que a vida se acurta, céntrase máis en manter unha sensación de benestar e menos en obxectivos que teñen que ver co coñecemento e co futuro. O resultado disto é o desenvolvemento de habilidades de <<regulación emocional>> que permiten controlar, en parte, os que se recorda.

\subsubsection{Terminoloxía sobre olvido motivado}
\begin{itemize}
	\item \textbf{Represión:} No marco freudiano, é un mecanismo de defensa psicolóxica que envía 
	recordos, ideas e pensamentos non desexados ó inconsciente para reducir o conflito e a dor 
	psíquica. Os contidos reprimidos non desaparecen da mente: poden seguir exercendo a súa 
	influenza na conduta de forma inconsciente, manifestándose nos soños, preferencias, temas de 
	discusión ou reaccións emocionais. Ademais, estes contidos poden emerger en ocasións futuras, no 
	que Freud chamaba \textit{regreso do reprimido}.
	\item \textbf{Distinción entre <<represión>> e <<supresión>>:} A represión é un proceso 
	inconsciente e automático de defensa, mediante o cal se exclúe un recordo da conciencia sen que 
	a persoa chegue a percibilo. A supresión é un proceso consciente, intencional e guiado por 
	obxectivos, no que se exclúen ideas e recordos da consciencia. 
	\item \textbf{Olvido motivado:} Ademais do olvido intencional e da amnesia psicóxena, recolle os 
	casos nos que o olvido non é accidental pero tampouco obedece a unha intención, e outros casos 
	máis ordinarios nos que as persoas olvidan cousas desgradables.
	\begin{itemize}
		\item \underline{Olvido intencional}: Surxe como consecuencia dunha intención consciente de 
		olvidar. Inclúe estratexias conscientes para olvidar, como a supresión e o cambio 
		intencional de contexto.
		\item \underline{Amnesia psicóxena}: Abarca calquera tipo de olvido de orixe psicolóxica, 
		non atribuible a danos ou disfuncións biolóxicas. Adoita empregarse este termo para casos de 
		olvido profundo e extraordinario de grandes bloques de información sobre a propia vida, ou 
		para o olvido de eventos específicos que deberían recordarse. 
	\end{itemize}
\end{itemize}

\subsubsection{Factores que predín o olvido intencional}
En ocasións pode producirse olvido para reducir as interrupcións da interferencia proactiva na nosa concentración. Estos casos adoitan estudarse co procedemento de <<olvido dirixido>>, no que se pide de forma explícita ós participantes que olviden un material codificado recentemente. Existen dúas variantes deste procedemento, e cada unha delas implica distintos procesos de olvido:
\begin{itemize}
	\item \textbf{Método do ítem:} Cada participante debe memorizar unha serie de ítems de un en un. 
	Cada vez que se presenta un ítem, indícase se este debe ser recordado ou olvidado. Logo 
	realízase unha proba de memoria sobre tódalas palabras, na que se observa que o recordo das 
	palabras que debían olvidarse é menor que o das que debían recordarse.
	
	Os efectos deste método obsérvanse tanto en recordo coma en recoñecemento, polo que se cre que 
	reflexa un déficit na codificación episódica. A instrución de recordar desencadea unha 
	codificación semática elaborada, pero a de olvidar permite deter o procesamento da palabra. 
	Neste sentido, as persoas exercen control sobre o que deixan pasar á memoria regulando que 
	estímulos reciben procesamento elaborativo e cales non.
	\item \underline{Método da lista}: Preséntase a instrución de olvidar cando os suxeitos xa levan 
	estudada media lista, xeralmente por sorpresa. A continuación preséntase o resto da lista e logo 
	sométese ós suxeitos a unha proba de memoria, ben sobre ambas listas ou ben só sobre a primeira. 
	Compáranse os resultados cos de un grupo que non recibiu a instrución de olvidar. Obtéñense dous 
	resultados destacables: que o grupo \textit{olvidar} tende a recordar mellor a segunda lista que 
	o grupo \textit{recordar}, é dicir, que a interferencia proactiva da primeira lista adoita 
	desaparecer cando os participantes cren que poden olvidala, o que significa que a instrución de 
	olvidar produce un beneficio; e que o grupo \textit{olvidar} recorda peor os ítems da primeira 
	lista en comparación co grupo \textit{recordar}, o que significa que a instrución ten un custo. 
	
	Crese que este método reflexa un déficit de recuperación: os participantes non saben que terán 
	que olvidar a primeira lista mentres a estudan, polo que non teñen motivos para non codificar 
	eficazmente. Os seus efectos adoitan desaparecer en probas de recoñecemento, pero as probas de 
	memoria implícita revelan a presenza dos ítems. Con isto obsérvase, como dicía Freud, que o 
	olvidado intencionalmente pode influir na conduta, aínda que non se sexa consciente.
	
	Existen dúas teorías que poden explicar o olvido dirixido mediante mediante o método da lista:
	\begin{itemize}
		\item \textbf{Hipótese da inhibición da recuperación:} A instrución de olvidar a primeira 
		lista inhibe os seus ítems reducindo o seu nivel de activación, e afectando así ó seu 
		recordo posterior. A inhibición non ten efectos permanentes, polo que os recordos son 
		dispoñibles. Esta hipótese explica por que é difícil recordar os ítems olvidados de forma 
		intencional pero é fácil reconocelos: a súa presentación restablece os seus niveis de 
		activación.
		\item \textbf{Hipótese do cambio contextual:} A instrución de olvidar separa mentalmente os 
		ítems que deben olvidarse dos que deben recordarse. Se o contexto mental cambia entre as 
		partes da lista, e o contexto da segunda parte permanece activo durante a proba final, os 
		ítems a olvidar recordaranse menos porque o novo contexto é unha clave de recuperación débil 
		para eles. 
		
		Esta hipótese non é incompatible coa anterior se se asume que o cambio de contexto mental se 
		realiza inhibindo o contexto non desexado, en lugar dos ítems individuais.
	\end{itemize}
\end{itemize}

\end{document} %Non pode haber nada escrito despois desta instrución.
