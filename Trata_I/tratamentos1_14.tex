\documentclass[a4paper,11pt]{article}
\usepackage[spanish]{babel}        
\usepackage[utf8]{inputenc}           


\usepackage[T1]{fontenc} 
\usepackage{graphicx}    
\usepackage{color}      
\usepackage{anysize}     
\usepackage{multicol}    
\usepackage{multirow}
\usepackage{bm}          
\usepackage{textcomp}   
\usepackage{eurosym}     
\usepackage{amsthm}     
\usepackage{amsmath,amsfonts} 
\usepackage{lineno} 


\marginsize{1.5cm}{1.5cm}{1.5cm}{1.5cm} 
\parindent=0mm                        
\parskip=3mm                         
\renewcommand{\baselinestretch}{1}    
\renewcommand{\spanishtablename}{Táboa}
 

\title{Tema 14: Inoculación de estrés}
\date{} 

\begin{document}   

\maketitle 

\section{Bases teóricas}
A técnica de inoculación de estrés orixínase na década dos 70, case ó mesmo tempo que a técnica de autoinstrucións. Meichenbaum, o seu creador, deseñouna inicialmente como unha técnica para o control da ansiedade, pero posteriormente descubriuse a súa utilidade nun gran número de problemas. Actualmente emprégase como un modelo xeral de tratamento no que se ensina ó paciente un conxunto de habilidades que lle permiten enfrontarse ós problemas da vida diaria. 

Esta técnica céntrase nas relacións de interdependencia existentes entre variables afectivas, fisiolóxicas, condutuais, cognitivas e socio-ambientais. Baséase en dous modelos:
\begin{itemize}
	\item \textbf{Modelo de afrontamento de Murphy (1962):} Propón tres momentos distintos e 
	consecutivos para explicar as reaccións das persoas ante situacións de posible ameaza: 
	preparación para o afrontamento, afrontamento da situación e esforzos secundarios de afrontamento 
	para lidiar coas consecuencias da situación. Este modelo permite xustificar as fases incluidas na 
	inoculación de estrés. 
	\item \textbf{Modelo de afrontamento do estrés de Lazarus e Folkman:} Entende o estrés como a 
	relación entre a persoa e un entorno que esta percibe como desbordante, e que pon en perigo o seu 
	benestar. Por tanto, os factores que determinan o estrés serían a persoa e a situación 
	(normalmente ambigua e incerta). 
\end{itemize}

\section{Procedemento de inoculación de estrés}
\begin{enumerate}
	\item \underline{Conceptualización}: Establécese a relación terapeuta-paciente e recóllense os 
	primeiros datos. O terapeuta debe explicar ó paciente as reaccións das persoas ante as situacións 
	de estrés, empregando como exemplo o problema real do paciente. Tamén debe corrixir as súas 
	falsas atribucións e crenzas sobre o problema. Nesta fase é importante prever a resistencia do 
	paciente a adherirse ó tratamento.
	\item \underline{Adquisición e ensaio de habilidades}: Ensinar técnicas de relaxación, de 
	reestruturación cognitiva, de autoinstrución, de solución de problemas, de habilidades condutuais 
	(modelado, exposición, ensaio de conduta, etc.) e de autoeficacia e autorrecompensa (fomento de 
	avaliacións realistas, distinción entre erro e fracaso, reforzo de cambios graduais e de 
	intentos, atribución do cambio a un mesmo e establecemento de metas realistas). 
	\item \underline{Aplicación e consolidación}: Ponse en práctica o aprendido na fase anterior. As 
	estratexias que se empregan son o ensaio en imaxinación, o ensaio de conduta 
	(\textit{role-paying}) e a exposición en vivo graduada. Nesta fase tamén se prepara ó paciente 
	para a prevención de recaídas. 
	\item \underline{Avaliación do tratamento}: Compróbase se o paciente emprega de forma axeitada as 
	técnicas de afrontamento aprendidas (avaliación postratamento). Pasados uns meses, compróbase se 
	segue empregando as técnicas e se se produciu unha mellora na súa calidade de vida (avaliación de 
	seguimento). 
\end{enumerate}





\end{document}