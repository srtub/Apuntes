\documentclass[a4paper,11pt]{article}
\usepackage[spanish]{babel}        
\usepackage[utf8]{inputenc}           


\usepackage[T1]{fontenc} 
\usepackage{graphicx}    
\usepackage{color}      
\usepackage{anysize}     
\usepackage{multicol}    
\usepackage{multirow}
\usepackage{bm}          
\usepackage{textcomp}   
\usepackage{eurosym}     
\usepackage{amsthm}     
\usepackage{amsmath,amsfonts} 
\usepackage{lineno} 
\usepackage{latexsym}


\marginsize{1.5cm}{1.5cm}{1.5cm}{1.5cm} 
\parindent=0mm                        
\parskip=3mm                         
\renewcommand{\baselinestretch}{1}    
\renewcommand{\spanishtablename}{Táboa}
 

\title{Tema 17: Adestramento en habilidades sociais}
\date{} 

\begin{document}   

\maketitle 

\section{Introdución}
O AHS é unha das técnicas cognitivo-conductuais, máis potentes para aumentar a eficacia interpersoal, tratar diferentes problemas psicolóxicos e mellorar a calidade de vida dos pacientes. Trátase dunha das técnicas máis difíciles e laboriosas de empregar, xa que require coñecementos de diversas áreas da psicoloxía. Tamén é unha das más empregadas no marco da saúde mental.

Unha conduta socialmente habilidosa implica tres compoñentes da habilidade social: a dimensión condutual (tipo de habilidade), a dimensión persoal (variables) e a dimensión situacional (contexto ambiental). A aprendizaxe de habilidades sociais permite iniciar e manter conversacións, falar en público, expresar amor, afecto e agrado, defender os propios dereitos, rexeitar peticións, solicitar favores e facer e aceptar cumplidos, entre outras accións. 

\section{Marco teórico}
O EHS pode entenderse como unha técnica de intervención baseada nos principios da aprendizaxe social, e podería definirse como unha terapia dirixida a ensinar estratexias e habilidades interpersonais ó sujeto para incrementar a competencia da súa actuación en situacións sociais e/ou críticas. As premisas que subxacen ó EHS son:
\begin{itemize}
	\item[-] As relacións personais son importantes para o desenvolvemento e o funcionamento 
	psicolóxico.
	\item[-] A falla de armonía interpersoal contribúe a potenciar as disfuncións psicolóxicas.
	\item[-] Certos estilos interpersonais son máis adaptativos que outros. 
	\item[-] Os estilos e estratexias interpersonais poden ensinarse.
	\item[-] A mellora na competencia persoal pode contribuir a mellorar o funcionamento psicolóxico.
\end{itemize}

As razóns que impedirían a un suxeito manifestar unha conduta socialmente habilidosa son:
\begin{itemize}
	\item[$\star$] Non existen respostas pertinentes no seu repertorio de conduta, ben por non telas 
	aprendido ou por levalas a cabo de forma errónea.
	\item[$\star$] Padece ansiedade condicionada debido a experiencias previas aversivas, e esta 
	impídelle responde axeitadamente.
	\item[$\star$] Examina de forma incorrecta a súa actuación social, autoavaliándose negativamente.
	\item[$\star$] Non está motivado para actuar apropiadamente, o que pode deberse a unha carencia 
	de reforzo nas interaccións sociais.
	\item[$\star$] Non sabe discriminar as situacións nas que unha determinada resposta é eficaz.
	\item[$\star$] Non ten seguridade sobre os seus dereitos.
	\item[$\star$] Sufriu illamento.
	\item[$\star$] Padeceu dificultades ambientais restrictivas que lle impiden expresarse
	correctamente. 
\end{itemize}

Estas razóns relaciónanse con catro modelos que permiten intervir en determinados déficits: modelo de déficit en habilidades, modelo de ansiedade condicionada, modelo cognitivo avaliativo e modelo de discriminación errónea. 

\section{Procedemento en AHS}
Antes de comezar coa intervención, deben identificarse e avaliarse as áreas nas que o paciente ten dificultades. Débese fomentar a motivación e ensinar ó paciente a relaxarse e a manter o autocontrol, así como explicarlle en que vai consistir o adestramento. Os pasos a seguir son os seguintes:
\begin{enumerate}
	\item Construir un sistema de crenzas que manteña o respeto polos propios dereitos e os dos 
	demais. 
	\item Distinguir entre respostas asertivas, pasivas e agresivas.
	\item Reestruturación cognitiva da forma de pensar incorrecta do paciente.
	\item Ensaio condutual de respostas socialmente axeitadas.
\end{enumerate}

\section{Ensaio de conduta}
Mediante a representación de papeis ensínase a forma axeitada de enfrontar situacións da vida real. Esta representación realízase con varias persoas, non debe durar máis de tres minutos e as respostas deben ser o máis breves posible. Nesta técnica, o cambio de conduta é un fin en si mesmo. Os pasos para levala a cabo son:
\begin{enumerate}
	\item Describir a situación problema.
	\item Representar o que o paciente fai normalmente nesa situación.
	\item Identificar as posibles cognicións desadaptativas e os dereitos humanos básicos implicados 
	na situación.
	\item Seleccionar unha resposta axeitada para ese caso en concreto e definir obxectivos a curto e 
	longo prazo.
	\item Suxerir respostas alternativas (o terapeuta e os demais membros do grupo) e representalas 
	mediante modelado.
	\item Práctica encuberta da conduta modelada por parte do paciente, e representación da mesma 
	ante os membros do grupo. O paciente debe tratar de integrar a resposta aprendida no seu 
	repertorio de conduta.
	\item Avaliar a eficacia da resposta, tanto o paciente como o resto de membros do grupo.
	\item Representar de novo a conduta (modelo), tendo en conta as suxerencias realizadas. En cada 
	ensaio convén non tratar de mellorar máis de dous elementos verbais ou non verbais á vez.
	\item Repetir os pasos tantas veces como sexa necesario ata que tódolos membros do grupo 
	coincidan en que a resposta esta lista para ser empregada na vida real. 
	\item Repetir a escea completa e dar ó paciente as últimas instrucións, explicándolle as 
	consecuencias positivas e negativas coas que podería atoparse. 
\end{enumerate}

As vantaxes desta técnica son que permite empregar diferentes suxeitos para a representación de papeis, inclúe retroalimentación, a aprendizaxe vicaria é máis eficaz, proporciona un contexto de apoio, emprega situacións reais (o que implica unha maior probabilidade de emisión posterior da conduta) e economiza o tempo do terapeuta.

\section{Variacións da técnica: o AHS individual}
Esta modalidad facilita a avaliación inicial das habilidades e debilidades do paciente durante o periodo de liña base, así como unha observación continua deste. Tamén permite que o terapeuta se concentre máis nos problemas particulares do individuo. Adoita empregarse cando a ansiedade excesiva do suxeito lle impide adaptarse a un grupo. Nestes casos, comézase con AHS individual e posteriormente, cando a ansiedade diminúe, intégrase ó paciente nun grupo de terapia. 

\end{document}