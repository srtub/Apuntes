\documentclass[a4paper,11pt]{article}
\usepackage[spanish]{babel}        
\usepackage[utf8]{inputenc}           


\usepackage[T1]{fontenc} 
\usepackage{graphicx}    
\usepackage{color}      
\usepackage{anysize}     
\usepackage{multicol}    
\usepackage{multirow}
\usepackage{bm}          
\usepackage{textcomp}   
\usepackage{eurosym}     
\usepackage{amsthm}     
\usepackage{amsmath,amsfonts} 
\usepackage{lineno} 
\usepackage{latexsym}


\marginsize{1.5cm}{1.5cm}{1.5cm}{1.5cm} 
\parindent=0mm                        
\parskip=3mm                         
\renewcommand{\baselinestretch}{1}    
\renewcommand{\spanishtablename}{Táboa}
 

\title{Tema 16: Técnicas de modelado}
\date{} 

\begin{document}   

\maketitle 

\section{Introdución}
O modelado é un proceso de aprendizaxe baseado na observación da conduta dunha ou de varias persoas, no que a conduta do modelo actúa como estímulo para xerar condutas, pensamentos ou actitudes semellantes no observador. Esta técnica pode empregarse para adquirir novos repertorios de conduta ou habilidades, inhibir ou desinhibir condutas xa adquiridas, facilitar condutas, incrementar a estimulación ambiental ou promover cambios na activación emocional e de valencia afectiva.

Segundo Bandura, o modelado implica os seguintes subprocesos:
\begin{itemize}
	\item[$\leadsto$] Atender ás condutas do modelo.
	\item[$\leadsto$] Reter o observado.
	\item[$\leadsto$] Repetir o observado para mellorar a calidade da propia actuación.
	\item[$\leadsto$] Estar motivado para poñer en marcha o aprendido en distintas situacións.
\end{itemize}

\section{Factores que afectan ó modelado}
\begin{itemize}
	\item \textbf{Factores que afectan á adquisición:} 
	\begin{itemize}
		\item \underline{Proceso de atención}: O modelo debe realizar a conduta de forma clara, 
		mediante presentacións curtas e dividíndoa en compoñentes sinxelos. O nivel de dificultade 	
		da conduta debe ser axeitado ó observador. Convén empregar instrucións específicas e resaltar 
		a utilidade funcional do que se está ensinando.
		\item \underline{Características do modelo}: Os modelos máis eficaces son aqueles que se 
		parecen ó suxeito (en idade, sexo, etc.), teñen prestixio para el, son eficaces e/ou teñen 
		valor afectivo para o observador (amigos, familia, etc.). 
		\item \underline{Características do observador}: Debe realizar unha valoración previa das 
		súas habilidades, controlar os niveis de ansiedade e percibir a utilidade do que observa.
		\item \underline{Mellora dos procesos de retención}: Deben empregarse instrucións precisas e 
		claras. A aprendizaxe mellora coa práctica, sexa esta cognitiva (práctica mental, imaxes, 
		autoinstrucións) ou na vida real.
		\item \underline{Métodos de presentación}: O modelado pode realizarse en vivo (cun modelo 
		real), en vídeo (modelos filmados), mediante instrucións verbais, de forma encuberta ou por 
		contraste (observación de comportamentos axeitados e inaxeitados e das súas respectivas 
		consecuencias). Pode empregarse un único modelo ou múltiples. 
	\end{itemize}
	\item \textbf{Factores que aumentan a actuación:} O adestramento en habilidades complexas debe 
	realizarse de forma gradual, comezando por compoñentes sinxelos e avanzando ata os niveis máis 
	complexos. É importante practicar o reensaio e aportar \textit{feedback}. Para que sexa máis 
	eficaz, o \textit{feedback} debe aplicarse inmediatamente despois da execución, referíndose a 
	compoñentes específicos da conduta e sinalando primeiro os correctos (reforzo positivo) e logo 
	os incorrectos. En terapias grupais, é positivo empregar tamén o \textit{feedback} dos membros do 
	grupo. 
	\item \textbf{Reforzos e incentivos para a actuación:} Consideramos dous tipos segundo a quen se 
	apliquen:
	\begin{itemize}
		\item \underline{Reforzo vicario}: Os reforzadores aplicados ó modelo poden ser extrínsecos 
		(tanxibles, sociais ou de actividade) ou intrínsecos (autorreforzo do modelo, que serve para 
		ensinar autocontrol ó observador). Se se pretende reducir unha conduta de forma vicaria, 
		aplícanse castigos ou sancións á conduta do modelo.
		
		O reforzo vicario ten funcións informativas, motivacionais, emotivas e valorativas.
		\item \underline{Reforzo directo do observador}: O seu uso aumenta as probabilidades de que 
		se repita a conduta. É máis efectivo a longo prazo que o reforzo vicario.
	\end{itemize}
	\item \underline{Transferencia e xeralización:} Algunhas estratexias que favorecen este aspecto 
	son o uso de regras xerais que gobernen a actuación correcta, a sobreaprendizaxe (automatización 
	de habilidades), o emprego de contextos de adestramento realistas, a variabilidade estimular 
	(varios modelos, contextos e graos de dificultade), o reforzo en situacións reais, as tarefas 
	para casa e o adestramento en prevención de recaídas e posibles fracasos.
\end{itemize}

\section{Procedemento xeral do modelado}
Antes da intervención debe levarse a cabo unha avaliación condutual do paciente e deben definirse as metas a curto e medio prazo. Os pasos dun adestramento son:
\begin{enumerate}
	\item Explicar a conduta ou habilidade ó paciente, que debe estar moderadamente relaxado.
	\item Modificar as crenzas erróneas que poida ter o suxeito.
	\item Aportar instrucións acerca dos aspectos máis relevantes a atender.
	\item O modelo realiza e describe verbalmente a conduta.
	\item O paciente describe a conduta observada, xunto cos seus antecedentes e consecuentes.
	\item O paciente realiza a conduta, con axuda nos primeiros ensaios e recibindo \textit{feedback}
	positivo tras cada un.
	\item Levar a cabo varios ensaios para sobreaprender a habilidade, empregando múltiples modelos.
	\item Deseñar tarefas para a casa.
	\item Establecer continxencias de reforzo no medio natural e practicar o autorreforzo.
	\item Adestrar as habilidades en orde de dificultade progresiva e definir estratexias para 
	enfrontar posibles obstáculos.
\end{enumerate}

\section{Técnicas de modelado}
\subsection{Modelado participante}
Máis rápido e eficaz na eliminación de fobias que a desensibilización sistemática. A súa efectividade obedece á extinción vicaria e á adquisición de coñecementos técnicos e habilidades motoras. Os pasos para levalo a cabo son:
\begin{enumerate}
	\item Construir unha xerarquía de situacións temidas (empregando unha escala de USAS).
	\item O modelo amosa o comportamento axeitado en cada ítem da xerarquía.
	\item O paciente repite a conduta do modelo, con axudas físicas ó principio que irán 
	desaparecendo gradualmente.
	\item Deseño de tarefas para casa de dificultade progresiva.
\end{enumerate}

\subsection{Ensaio mental simbólico ou cognitivo}
\begin{enumerate}
	\item O modelo realiza a conduta obxectivo.
	\item Instruir ó paciente en codificación simbólica, que consiste en organizar a acción mediante 
	un protocolo verbal fácil de recordar. Trátase de construir unha especie de guión coas regras que 
	se deben seguir para levar a cabo a conduta.
	\item Practicar o ensaio cognitivo, no que o paciente debe visualizarse a si mesmo realizando as 
	condutas previamente realizadas polo modelo.
\end{enumerate}

\subsection{Modelado encuberto}
Permite modificar condutas existentes ou aprender outras novas mediante a observación en imaxinación do comportamento dun modelo e das súas consecuencias. Emprégase con pacientes que non son capaces de imaxinarse a si mesmos pero si a outras persoas. Os pasos a seguir son:
\begin{enumerate}
	\item Presentar varias esceas que o paciente debe imaxinar (e incluso completar). O tempo entre a 
	presentación destas pode variar entre un e cinco minutos.
	\item Presentar condutas máis difíciles en cada sesión.
	\item Practicar as condutas na vida real.
\end{enumerate}

\subsection{Automodelado}
O paciente aprende ou modifica patróns comportamentais observando a súa propia conduta.
\begin{enumerate}
	\item Grabar en vídeo as condutas. 
	\item Amosar as grabacións ó paciente.
	\item Realizar ensaios das condutas observadas na vida real.
\end{enumerate}

O automodelado diferénciase da autoconfrontación en que nesta outra técnica amósanse exemplos negativos da conduta do paciente (o cal non é recomendable).

\subsection{Modelado autoinstrucional}
Ensínase ó paciente a darse instrucións a si mesmo que guíen as súas accións.
\begin{enumerate}
	\item O terapeuta modela unha tarefa en voz alta.
	\item O paciente executa a tarefa mentres é instruido.
	\item O paciente repite a tarefa instruíndose a si mesmo en voz alta.
	\item O paciente repite a tarefa instruíndose a si mesmo en voz baixa.
	\item O paciente repite a tarefa instruíndose a si mesmo de forma encuberta. 
\end{enumerate}

\end{document}