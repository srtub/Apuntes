\documentclass[a4paper,11pt]{article}
\usepackage[spanish]{babel}        
\usepackage[utf8]{inputenc}           


\usepackage[T1]{fontenc} 
\usepackage{graphicx}    
\usepackage{color}      
\usepackage{anysize}     
\usepackage{multicol}    
\usepackage{bm}          
\usepackage{textcomp}   
\usepackage{eurosym}     
\usepackage{amsthm}     
\usepackage{amsmath,amsfonts} 
\usepackage{lineno}     


\marginsize{1.5cm}{1.5cm}{1.5cm}{1.5cm} 
\parindent=0mm                        
\parskip=3mm                         
\renewcommand{\baselinestretch}{1}    
 

\title{Tema 1: Técnicas de modificación de conduta}
\date{} 


\begin{document}   

\maketitle 

\section{Características, fundamentos teóricos e definición das técnicas de modificación de conduta}
\begin{itemize}
	\item[-] Distintos enfoques teóricos e diferentes técnicas.
	\item[-] Obxecto de estudo: condutas específicas e os seus determinantes.
	\item[-] Enfoque centrado no momento actual (\textit{aquí e agora}).
	\item[-] Obxectivo da intervención: diminuir ou eliminar condutas desadaptativas, incrementar ou 
	instaurar condutas adaptativas. 
	\item[-] Interdependencia avaliación-tratamento.
	\item[-] Enfoque científico. Avaliación empírica dos procedementos.
	\item[-] Protocolos de intervención que respetan a individualidade.
	\item[-] Participación activa do paciente na terapia e colaboración terapeuta-paciente. 
	\item[-] Importancia das variables individuais: axustar o tratamento e os procedementos ás 
	necesidades do paciente. 
	\item[-] Progresión paso a paso: do máis sinxelo ó máis complexo, axustándose sempre ó paciente.
	\item[-] Terapia breve.
	\item[-] Combinación simultánea de tratamentos. 
	\item[-] Importancia do medio familiar, social e cultural do paciente. 
\end{itemize}

\section{Desenvolvemento histórico}
Considéranse tres xeracións históricas das técnicas de modificación de conduta:
\begin{enumerate}
	\item \underline{Primeira xeración}: Entre 1950 e 1970. Imperan o condicionamento operante e o 
	condicionamento clásico.
	\item \underline{Segunda xeración}: En torno a 1970. Imperan os enfoques condutual e cognitivo 
	(Ellis, Beck) e a técnica de modelato (Bandura). Considérase que os pensamentos determinan a 
	conduta. Por tanto, trátase de eliminar os pensamentos inaxeitados e substituilos por outros máis 
	axeitados, para provocar un cambio nas emocións. 
	\item \underline{Terceira xeración}: En torno a 1990. Destacan a Terapia de Aceptación e 
	Compromiso (Hayes et al.), consistente en aceptar o que sucede e traballar para mellorar aquelo 
	que nos fai sentir mal, e a Terapia de Activación Condutual (Jacobson et al.), que propón que a 
	depresión surxe a causa da falta de actividades gratificantes, que afectan positivamente ó 
	pensamento de quen as realiza. 
\end{enumerate}

\subsection{Antecedentes (ata 1938)}
\begin{itemize}
	\item[$\circ$] \textbf{Desenvolvemento teórico e investigación:} Desenvólvense as leis do 
	condicionamento clásico (Pavlov) e formúlase a <<Lei do efecto>> sobre a aprendizaxe por 
	recompensa (Thorndike).
	\item[$\circ$] \textbf{Aplicacións:} Investigacións sobre a neurose experimental. Primeiras 
	aplicacións do condicionamento ó comportamento humano (\textit{exemplo:} experimento do pequeno 
	Albert). 
	\item[$\circ$] \textbf{Ámbito clínico:} Insatisfacción cos tratamentos psicolóxicos imperantes 
	(\textit{exemplo:} psicanálise de Freud). 
\end{itemize}

\subsection{Aparición (1938-1958)}
\begin{itemize}
	\item \underline{Sudáfrica}: Desenvólvense tratamentos para os trastornos fóbicos e outros 
	problemas de ansiedade, entre eles a desensibilización sistemática. Algúns autores destacados son 
	Wolpe, Rachman ou Lazarus. 
	\item \underline{Reino Unido}: Investigación e aplicación de técnicas derivadas da psicoloxía da 
	aprendizaxe á neurose. Nace o <<Instituto de Psiquiatría da Universidade de Londres>> (Eysenck e 
	Shapiro). Demóstrase a ineficacia da psicoterapia tradicional. 
	\item \underline{Estados Unidos}: Investigación e aplicación de técnicas de condicionamento 
	operante a condutas psicóticas, suxeitos con retraso mental e modificación de condutas 
	infantiles. Un autor destacado é Skinner.
\end{itemize}

\subsection{Consolidación (1958-1970)}
\begin{itemize}
	\item[-] Etapa de fundamentación teórica, baseada na psicoloxía da aprendizaxe e na aprendizaxe 
	animal.
	\item[-] Enfatízanse os eventos e condutas observables, así como a demostración obxectiva da 
	eficacia do tratamento en base a estes (cambios observables).
	\item[-] Descrición dos trastornos en termos de relacións: estímulos, condutas e consecuencias.
	\item[-] Desenvólvense novos tratamentos psicolóxicos (DS, técnicas operantes e aversivas, 
	relaxación...).
	\item[-] Interrelación entre teoría, investigación e aplicacións.
	\item[-] Limitación en modelos explicativos, tipo de trastornos abordados e programas de 
	tratamento psicolóxico excesivamente sinxelos. 
\end{itemize}

\subsection{Expansión (1970-1990)}
\begin{itemize}
	\item[$\diamond$] Desenvólvense novos planteamentos teóricos (modelo da aprendizaxe social e 
	enfoque cognitivo condutual) e novas técnicas, algunhas de base teórica pouco consistente.
	\item[$\diamond$] Etapa de fundamentación metodolóxica, menor énfase na fundamentación teórica.
	\item[$\diamond$] Asignación dun papel máis activo ó paciente.
	\item[$\diamond$] Interese polo desenvolvemento da avaliación condutual.
	\item[$\diamond$] Programas de tratamento máis complexos (multicompoñentes).
	\item[$\diamond$] Importancia da relación terapeuta-paciente e das habilidades terapéuticas.
	\item[$\diamond$] Expansión a outros campos (educación, adiccións, empresas, hospitais, cárceres, 
	etc.). 
	\item[$\diamond$] Traballo interdisciplinar.
\end{itemize}

\subsection{Reconceptualización (etapa actual)}
\begin{itemize}
	\item[$\cdot$] Fundamentación das técnicas na psicoloxía cognitiva experimental.
	\item[$\cdot$] Preponderancia do enfoque cognitivo-condutual na práctica clínica.
	\item[$\cdot$] Aplicación a un amplo rango de pacientes, condutas e contextos.
	\item[$\cdot$] Énfase na autonomía persoal (autocontrol). Participación moi activa do suxeito.
	\item[$\cdot$] Preocupación pola xeralización de resultados.
	\item[$\cdot$] Interese pola prevención, especialmente a de recaídas.
	\item[$\cdot$] Importancia da colaboración con outros profesionais e implicación de persoas do 
	entorno como axentes do cambio (familia, amigos, etc.).  
\end{itemize}

\section{Enfoques históricos}
\subsection{Enfoque condutista mediacional}
\begin{itemize}
	\item Principios do condicionamento clásico no tratamento da conduta problemática.
	\item Modelo básico: traballos de Wolpe e Eysenck.
	\item Fundamentos teóricos: Pavlov, Hull, Mowrer.
	\item Variables mediacionais ou constructos hipotéticos (ansiedade, medo) na explicación das 
	condutas problemáticas.
	\item Aspectos cognitivos como as imaxes considéranse secuencias E-R rexidas polas mesmas leis de 
	aprendizaxe que as manifestas. Plantéxanse situacións supoñendo que as imaxes mentais suxeridas 
	actuarán como estímulos reais, provocando que o suxeito responda coma se se atopase na situación 
	problema.
	\item Campo de aplicación: problemas nos que a ansiedade desempeña un papel central.
	\item Técnicas: DS, inundación, terapia aversiva...
	\item Críticas:
	\begin{itemize}
		\item Utilización de medidas subxectivas (\textit{exemplo:} USAS) do nivel de ansiedade dos 
		pacientes.
		\item Os procesos mediacionais seguen as mesmas leis de aprendizaxe que as condutas 
		manifestas. Os pacientes non responden igual na situación imaxinada que na real.
		\item Xeralización do nivel encuberto ó mundo real.
		\item Insuficiencia dos modelos de condicionamento para explicar a adquisición dos trastornos 
		de ansiedade (as fobias non se adquiren sempre coma no ``pequeno Albert''). 
	\end{itemize}
\end{itemize}

\subsection{Enfoque da análise condutual aplicada}
\begin{itemize}
	\item Intervencións baseadas nos principios e procedementos do condicionamento operante.
	\item Enfoque metodolóxico, análise experimental da conduta cuxo obxectivo é demostrar que o 
	cambio se debe ó tratamento. Preferencia por deseños de investigación intrasuxeito (A-B-A, 
	intervimos temporalmente e observamos os posibles cambios na situación orixinal).
	\item A conduta está controlada por factores ambientais: relación funcional entre a conduta e os 
	seus determinantes ambientais.
	\item A conduta anómala considérase o resultado dun proceso de aprendizaxe gobernado polos 
	mesmos principios que a normal. 
	\item Áreas de aplicación:
	\begin{itemize}
		\item Suxeitos con capacidade cognitiva limitada ou deteriorada.
		\item Modificación de ambientes sociais e institucionais (hospitais, cárceres...).
	\end{itemize}
	\item Críticas:
	\begin{itemize}
		\item Reduccionismo (só condutas manifestas). Trátase de modificar a conduta molesta, sen ter 
		en conta os factores persoais.
		\item Determinismo ambiental e escaso interese por factores persoais.
		\item Problema da xeralización e mantemento dos cambios terapéuticos.
	\end{itemize}
\end{itemize}

\subsection{Enfoque baseado na aprendizaxe social}
\begin{itemize}
	\item Aportacións de Bandura.
	\item O medio, tal como o percibe, filtra e procesa o suxeito, determina a conduta e esta, á súa 
	vez, modifica o medio, polo que se fala de determinismo recíproco medio-suxeito.
	\item Na explicación das condutas e a súa modificación fai referencia a 3 sistemas reguladores da 
	conduta:
	\begin{itemize}
		\item Estímulos externos. Afectan á conduta a través de condicionamento clásico.
		\item As consecuencias da conduta. Condicionamento operante (non me enfronto á situación, e 
		sigo tendo medo).
		\item Os procesos cognitivos mediacionais (percepción, interpretación e valoración dos 
		estímulos e influencia sobre a conduta).
	\end{itemize}
	\item Emprega conceptos derivados da teoría de Autoeficacia (convicción da persoa na súa propia 
	capacidade para acadar un resultado concreto) e do modelado para esclarecer a interdependencia 
	entre os cambios cognitivos e condutuais.
	\item Importancia do Autocontrol.
	\item Procedementos de modelado, técnicas de autocontrol...
	\item Críticas: variables mediacionais difíciles de precisar e de verse reflexadas nas condutas; 
	aportación escasa de técnicas.
\end{itemize}

\subsection{Enfoque cognitivo e/ou cognitivo condutual}
\begin{itemize}
	\item Procesos cognitivos desencadeantes do desenvolvemento, mantemento e modificación da 
	conduta. En función do que pense o suxeito, así se vai sentir e actuará.
	\item Supostos compartidos:
	\begin{itemize}
		\item A actividade cognitiva afecta ás emocións e á conduta (se penso moito no mal que me 
		estou atopando, atopareime cada vez peor).
		\item A actividade cognitiva pode ser monitorizada e alterada.
		\item O cambio condutual pódese conseguir a través da manipulación das formas de pensamento 
		disfuncional. Traballamos sobre os pensamentos e crenzas que nos fan dano.
		\item Integración dos procedementos condutuais e cognitivos. Ademais de traballar sobre os 
		pensamentos e crenzas, propóñense accións que permitan modificar paulatinamente a conduta. A 
		consecución efectiva deste tipo de accións pode resultar gratificante para os pacientes e 
		contribuir, á súa vez, a modificación de pensamentos.
	\end{itemize}
	\item Catro planteamentos terapéuticos:
	\begin{itemize}
		\item \textbf{Psicoterapias racionais:} Identificar e reestruturar as cognicións 
		desadaptadas (Ellis, Beck).
		\item \textbf{Habilidades de afrontamento:} Adquirir habilidades para afrontar situacións 
		(Meichenbaum).
		\item \textbf{Tecnicas de solución de problemas:} Ensinar a resolver problemas cunha 
		metodoloxía sistemática (D'Zurilla, Nezu).
		\item \textbf{Técnicas de condicionamento encuberto:} Baseadas no condicionamento clásico e 
		operante, aplicadas ás condutas cognitivas (Cautela).
	\end{itemize}
	\item Críticas:
	\begin{itemize}
		\item Non queda claro cales son as variables relevantes, a súa forma de actuación ou a 
		dirección en que se produce a influencia cando se afirma a relación entre pensamentos, 
		emocións e condutas (un cambio no pensamento conleva un cambio nas emocións e condutas).
		\item Escasa fundamentación teórica.
		\item Falta unha avaliación da eficacia dalgunhas técnicas. 
	\end{itemize}
\end{itemize}

\section{Distintos modelos de intervención}
\begin{itemize}
	\item[-] Psicodinámicos (Freud e outros).
	\item[-] Humanístico/existenciais (Escola da Gestalt, Rogers).
	\item[-] Sistémicos (Minuchin, Watzlawick, Haley).
	\item[-] Condutual e cognitivos.
\end{itemize}

\end{document}