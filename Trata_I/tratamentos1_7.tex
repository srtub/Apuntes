\documentclass[a4paper,11pt]{article}
\usepackage[spanish]{babel}        
\usepackage[utf8]{inputenc}           


\usepackage[T1]{fontenc} 
\usepackage{graphicx}    
\usepackage{color}      
\usepackage{anysize}     
\usepackage{multicol}    
\usepackage{multirow}
\usepackage{bm}          
\usepackage{textcomp}   
\usepackage{eurosym}     
\usepackage{amsthm}     
\usepackage{amsmath,amsfonts} 
\usepackage{lineno} 


\marginsize{1.5cm}{1.5cm}{1.5cm}{1.5cm} 
\parindent=0mm                        
\parskip=3mm                         
\renewcommand{\baselinestretch}{1}    
\renewcommand{\spanishtablename}{Táboa}
 

\title{Tema 7: Técnicas operantes para reducir e eliminar condutas}
\date{} 

\begin{document}   

\maketitle 

O control de estímulos reforzadores emprégase con frecuencia para a redución de condutas. Estas técnicas causan efectos menos intensos e inmediatos que as técnicas aversivas, pero tamén menos respostas emocionais negativas. É moi importante alternar a súa aplicación co desenvolvemento de condutas alternativas máis axeitadas. 

\section{Extinción}
Redúcese ou elimínase unha conduta mediante a non presentación dun reforzador positivo ante unha resposta previamente reforzada.
\begin{enumerate}
	\item Definir a conduta que se quere eliminar.
	\item Identificar os reforzadores que manteñen dita conduta.
	\item Controlar que o suxeito non obteña ningún destes reforzadores cando emita a conduta.
	\item Manter as condicións o tempo suficiente, implicando a tódalas persoas próximas ó paciente.
	\item Combinar con reforzo de respostas alternativas.
\end{enumerate}

A eficacia na redución da conduta depende da historia previa do paciente e dos reforzos que manteñen a conduta problema (o reforzo intermitente e os reforzadores moi potentes son máis difíciles de extinguir). 

Algúns fenómenos relacionados coa extinción son:
\begin{itemize}
	\item \textbf{Estalido de extinción:} Aumento da frecuencia e da intensidade da conduta problema 
	nas primeiras fases do procedemento.
	\item \textbf{Recuperación espontánea:} Reaparición da conduta problema ó deter o procedemento de
	extinción.
	\item \textbf{Agresión inducida polo tratamento:} Aparición de condutas agresivas durante a 
	aplicación do mesmo (insultos, rabietas, etc.).
\end{itemize}

Con esta técnica obtense unha redución progresiva e duradeira da conduta. Porén, só convén empregala se se pode permitir un aumento inmediato da conduta problema, e se esta non é perigosa para o paciente e para outras persoas. 

\section{Procedementos de reforzo diferencial (RD)}
Refórzanse unha ou máis condutas distintas da que se quere eliminar. Espérase que ó aumentar a emisión destas, se reduza a emisión da conduta problema. Temos tres tipos:
\begin{itemize}
	\item[$\odot$] \underline{Reforzo diferencial doutras condutas (RDO)}: A vantaxe é que non é 
	necesario identificar ou controlar os reforzadores que manteñen a conduta problema; as 
	desvantaxes, que o efecto pode ser lento e que pode seguir aparecendo a conduta problema.
	\begin{itemize}
		\item[1.] Definir a conduta a eliminar.
		\item[2.] Seleccionar os reforzadores.
		\item[3.] Establecer un criterio para obter reforzo: (a) emitir calquera conduta, (b) emitir 
		condutas específicas ou (c) non emitir a conduta problema durante un determinado periodo de 
		tempo.
		\item[4.] Nos casos b e c, establecer un intervalo temporal de aplicación.
	\end{itemize}
	\item[$\odot$] \underline{Reforzo diferencial de condutas incompatibles (RDI)}: As vantaxes e 
	desvantaxes son as mesmas que as do RDO.
	\begin{itemize}
		\item[1.] Definir a conduta problema.
		\item[2.] Seleccionar condutas incompatibles propias do paciente, ou instauralas.
		\item[3.] Seleccionar os reforzadores.
		\item[4.] Aplicar o RDI de forma continua inicialmente, e de forma intermitente máis tarde.
	\end{itemize}
	\item[$\odot$] \underline{Reforzo diferencial de taxas baixas (RDTB)}: Trátase de reducir a 
	emisión dunha conduta que é axeitada pero cuxa frecuencia é incorrecta. 
	\begin{itemize}
		\item[1.] Definir a conduta a reducir.
		\item[2.] Seleccionar os reforzadores.
		\item[3.] Posibilidade de empregar estímulos discriminativos.
		\item[4.] Aplicar o RDTB de forma continua inicialmente, e de forma intermitente máis tarde. 
		A aplicación pode levarse a cabo de dous xeitos:
		\begin{itemize}
			\item[(a)] Reforzar a conduta tras ter transcorrido un tempo determinado desde a última 
			emisión.
			\item[(b)] Reforzar a conduta se o tempo de emisión da mesma non supera un intervalo 
			establecido.
		\end{itemize}
	\end{itemize}
\end{itemize}

\section{Custo de resposta}
Redúcese ou elimínase unha conduta mediante a desaparición continxente dun estímulo agradable (reforzador positivo).
\begin{enumerate}
	\item Definir a conduta a eliminar.
	\item Seleccionar os reforzadores que se van eliminar (antes hai que permitir que o suxeito 
	acumule unha reserva de reforzadores, se non a ten).
	\item Establecer o custo que terá a emisión da conduta problema.
	\item Retirar o reforzador positivo continxente a dita conduta.
	\item Combinar con reforzo positivo de condutas alternativas.
\end{enumerate}

A eficacia do tratamento dependerá da importancia que o paciente lle de ó reforzador retirado. Deben probarse primeiro as cantidades de reforzador a eliminar, para evitar a habituación. Se aínda así se produce, vólvese á liña base e selecciónase un reforzador diferente. Ademais, non se debe permitir que o paciente se quede sen ningún reforzador, xa que isto diminuiría a motivación, e en consecuencia, a eficacia do tratamento.

As vantaxes desta técnica son a redución rápida e duradeira da conduta problema e que non é necesario identificar os reforzadores que manteñen dita conduta. As desvantaxes, a posible aparición de respostas emocionais negativas ou de agresión, a necesidade de acumular reforzadores e a posible realización doutras condutas non axeitadas. 

\section{Tempo fóra (TF)}
Retíranse as condicións do ambiente que resultan reforzantes ou alónxase ó suxeito do ambiente reforzante cando emite a conduta problema.
\begin{enumerate}
	\item Definir a conduta a eliminar.
	\item Identificar os reforzadores que manteñen dita conduta, sen necesidade de controlalos.
	\item Dispoñer dunha área de illamento próxima (non aversiva).
	\item Avisar ó paciente antes de iniciar a aplicación do tratamento.
	\item Cando emite a conduta, levar ó paciente á área de illamento sen enfadarse con el, e deixalo 
	alí durante un tempo suficiente (de 5 a 20 minutos). Pode levarse a cabo de tres xeitos:
	\begin{enumerate}
		\item Alonxando á persoa da situación para que non reciba reforzo.
		\item Manténdoa no mesmo lugar, pero en ausencia doutras persoas e de reforzo.
		\item Manténdoa no mesmo lugar mentres observa como outras persoas si obteñen reforzo. 
	\end{enumerate}
	\item Non finalizar o TF se o suxeito continúa emitindo a conduta problema; o TF comeza cando se 
	detén a emisión.
	\item Aplicar condutas restitutivas se o suxeito altera a área de illamento. 
\end{enumerate}

Mentras dura o TF, débese modificar dentro do posible o ambiente que reforza a conduta problema. Non se pode aplicar esta técnica cando o paciente realiza condutas autorreforzantes nin para escapar de situacións aversivas. 

As vantaxes desta técnica son a eliminación progresiva e relativamente rápida da conduta problema e que non é necesario controlar os reforzadores que a manteñen. As desvantaxes, que a redución adoita ser temporal (posible recuperación da conduta problema ó volver ó medio habitual) e que poden aparecer respostas emocionais negativas e agresión. 

\section{Saciación}
O reforzador que mantén a conduta problema perde o seu valor o presentalo de forma moi continuada, ou unha conduta en si mesma reforzante perde o seu valor ó realizala moi rapidamente. Os efectos desta técnica son moi inmediatos pero pouco permanentes. Non pode aplicarse a condutas perigosas.
\begin{enumerate}
	\item Identificar a conduta reforzante ou o reforzador da conduta problema.
	\item Programas sesións masivas (sen pausas nin descansos).
	\item Deseñar tarefas para a casa.
\end{enumerate}

As dúas variantes desta técnica son:
\begin{itemize}
	\item[•] \textbf{Saciación de reforzador (ou de estímulo):} Útil cando o reforzador é barato e 
	sinxelo de administrar, pero non cando a conduta se mantén grazas a múltiples reforzadores ou 
	se estes son sociais.
	\begin{itemize}
		\item[1.] Identificar o reforzador que mantén a conduta problema.
		\item[2.] Proporcionar o reforzador en cantidades excesivas.
	\end{itemize}
	\item[•] \textbf{Saciación de resposta (práctica negativa ou masiva):} Emitir a conduta problema 
	de forma masiva. Se existe algún reforzador relacionado con ela, non é necesario identificalo.
\end{itemize}

\section{Sobrecorrección}
Tras a realización dunha conduta inaxeitada, realízase outra que remedia en exceso a consecuencias desta.
\begin{enumerate}
	\item Identificar a consecuencia a restaurar. 
	\item Definir as condutas e aplicalas o tempo suficiente para producir os efectos desexados.
	\item Avisar verbalmente antes de iniciar a técnica. Esta debe aplicarse de forma consistente e 
	inmedianta, e debe suprimirse o reforzo durante a aplicación.
\end{enumerate}

Existen dous tipos de sobrecorrección:
\begin{itemize}
	\item[$\diamond$] \underline{Sobrecorrección restitutiva}: Restaurar o dano ocasionado por unha 
	conduta mellorando en exceso o estado orixinal.
	\item[$\diamond$] \underline{Sobrecorrección de práctica positiva}: Realizar de forma repetitiva 
	unha conduta positiva, incompatible coa conduta problema.
\end{itemize}

As vantaxes desta técnica son que ten menos consecuencias negativas que o castigo positivo, a aprendizaxe de condutas axeitadas e a redución rápida e duradeira da conduta problema. As desvantaxes, a posibilidade de unha redución máis lenta e pouco duradeira, e a aparición de posibles resistencias (rabietas, choros, etc.).

\end{document}