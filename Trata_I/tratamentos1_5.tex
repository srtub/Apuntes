\documentclass[a4paper,11pt]{article}
\usepackage[spanish]{babel}        
\usepackage[utf8]{inputenc}           


\usepackage[T1]{fontenc} 
\usepackage{graphicx}    
\usepackage{color}      
\usepackage{anysize}     
\usepackage{multicol}    
\usepackage{multirow}
\usepackage{bm}          
\usepackage{textcomp}   
\usepackage{eurosym}     
\usepackage{amsthm}     
\usepackage{amsmath,amsfonts} 
\usepackage{lineno} 


\marginsize{1.5cm}{1.5cm}{1.5cm}{1.5cm} 
\parindent=0mm                        
\parskip=3mm                         
\renewcommand{\baselinestretch}{1}    
\renewcommand{\spanishtablename}{Táboa}
 

\title{Tema 5: Técnicas de exposición}
\date{} 

\begin{document}   

\maketitle 

Estas técnicas consisten en exposicións repetidas e prolongadas a estímulos ansióxenos, no marco dun programa estruturado, coa fin de lograr que as respostas de evitación e escape non sexan interpretadas polo paciente como sinais de seguridade.

En relación con elas aparecen varios modelos teóricos:
\begin{itemize}
	\item \textbf{Modelo de habituación dual (Watts, 1979):} Propón que na redución das respostas de 
	ansiedade interveñen dous procesos distintos: a sensibilización (a exposición inicial incrementa 
	as respostas de ansiedade) e a habituacion (a exposición repetida reduce as respostas de 
	ansiedade). 
	\item \textbf{Modelo de estímulos evocadores e respostas evocadas (Marks, 1989):} Considera que 
	os estímulos fóbicos provocan unha serie de respostas que incrementan a ansiedade anticipatoria, 
	e esta á súa vez provoca as respostas de tipo evitativo.
	\item \textbf{Mecanismos explicativos da reducción do medo durante a exposición (Mathews, Gelder 
	e Johnston, 1985):} Serían a habituación (perspectiva fisioloxica), a extinción (perspectiva 
	condutual) e o cambio de expectativas (perspectiva cognitiva).
\end{itemize}

\section{Condicións da exposición}
Condicións previas á intervención son a confianza do paciente no terapeuta, a implicación activa pola súa parte e a posibilidade de implicación dun coterapeuta. Outros factores a ter en conta son:
\begin{itemize}
	\item[$\diamond$] \underline{Duración da exposición e intervalos entre sesións}: As sesións 
	longas facilitan a habituación, e o intervalo entre sesións debe ser curto.
	\item[$\diamond$] \underline{Gradiente da exposición, grao de activación e nivel atencional}: O 
	cambio debe ser tan rápido como o paciente poida tolerar. É necesario un grao mínimo de 
	activación durante a intervención, polo que debe suprimirse o consumo de ansiolíticos e alcohol. 
	O paciente debe comprometerse a prestar atención ó estímulo ansióxeno.
	\item[$\diamond$] \underline{Condutas de evitación}: As condutas de escape breves, se van 
	seguidas dunha reexposición inmediata, non afectan á eficacia da técnica.
	\item[$\diamond$] \underline{Potenciadores da exposición}: Poden ser a aprendizaxe modelada, a 
	respiración abdominal, informar dos progresos realizados, o emprego de autoinstrucións e a 
	implicación dun coterapeuta.
	\item[$\diamond$] \underline{Especificidade}: A exposición permite reducir unicamente as condutas 
	tratadas, e é pouco xeralizable. Esta intervención responde máis a un proceso de habituación que 
	a un de adquisición de habilidades globais. 
	\item[$\diamond$] \underline{Preditores do éxito terapéutico}:
	\begin{itemize}
		\item \textbf{Ó inicio do tratamento:} Condutas de evitación definidas, implicación, estado 
		de ánimo normal.
		\item \textbf{Durante o tratamento:} Progreso nas primeiras sesións, implicación nas tarefas.
		\item \textbf{Despois do tratamento:} Ter traballado a prevención de recaidas. 
	\end{itemize}
	\item[$\diamond$] \underline{Preditores de fracaso}: Abandono das prácticas regulares de 
	exposición, illamento social.
\end{itemize}

\section{Modalidades da exposición}
\subsection{Exposición en imaxinación}
\begin{enumerate}
	\item Informar sobre a natureza do problema, o procedemento terapéutico e o fundamento teórico da 
	exposición.
	\item Comprobar a capacidade de imaxinación do paciente mediante adestramento con esceas neutras.
	\item Confeccionar unha xerarquía de situacións que o paciente evita ou que lle provocan 
	ansiedade, segundo o grao de dificultade que lle supón enfrontarse a elas. Tódolos ítems da 
	xerarquía deben provocar, como mínimo, niveis moderados de ansiedade (aumento da distancia entre 
	ítems con respecto á desensibilización). 
	\item Presentar os distintos ítems, narrando a escea e avaliando o nivel de ansiedade cada poucos 
	segundos. Pódese repetir toda a descrición ou só partes dela, centrándose nos aspectos máis 
	relevantes desta.
	\item Solicitar ó suxeito que describa e reviva a experiencia na súa mente. 
	\item Reducir a ansiedade dun ítem polo menos ó 50\% antes de pasar ó seguinte. 
	\item Reforzar cada paso exitoso do paciente.
	\item Mandar practicar na casa cos primeiros ítems traballados. Pode axudar unha grabación da 
	sesión.
	\item Expoñerse en vivo a ítems inferiores.
\end{enumerate}

\subsection{Exposición en vivo}
\begin{itemize}
	\item[-] Antes, durante e ó final da sesión deben darse instrucións de respiración pausada, pero 
	sen desviar a atención do estímulo ansióxeno.
	\item[-] Reforzar cada paso exitoso do suxeito e recordarlle que a ansiedade irá diminuindo a 
	medida que permaneza na situación.
	\item[-] Porporcionar información específica sobre os progresos realizados, mediante 
	retroalimentación ou autorrexistros cubertos polo propio paciente.
	\item[-] Se se empregan coterapeutas, debe instruirselle sobre a natureza do problema, a técnica 
	de exposición e o manexo de intentos de evitación do paciente, mediante axudas verbais e físicas 
	(respiración abdominal, verbalizacións tranquilizadoras, etc.). Debe aprender tamén a estruturar 
	e afrontar as sesións de exposición e a estimular a independencia do paciente no afrontamento das 
	situacións. 
\end{itemize}

\subsection{Exposición en grupo}
Os resultados son comparables ós da exposición individual. Está indicada para pacientes que viven sós, carecen de habilidades sociais ou manteñen unha relacion de parella conflitiva. Os mellores resultados obtéñense en grupos cohesionados.

Aínda que a exposición en imaxinación pode levarse a cabo en grupo, a exposición en vivo debe executarse de forma individual (aínda que pode planificarse e avaliarse en grupo). En persoas con fobia social, esta técnica pode ser útil se se desempeñan diferentes roles.

\subsection{Autoexposición}
Presenta algúns requisitos: identificación de condutas problemáticas, establecemento de metas realistas, práctica regular, avaliación da redución de ansiedade, planificación do manexo de contratempos e implicación dun coterapeuta nas primeiras sesións.

\subsection{Exposición a través de novas tecnoloxías}
Empreganse programas asistidos por ordenador e de realidade virtual.


\end{document}