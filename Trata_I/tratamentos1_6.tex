\documentclass[a4paper,11pt]{article}
\usepackage[spanish]{babel}        
\usepackage[utf8]{inputenc}           


\usepackage[T1]{fontenc} 
\usepackage{graphicx}    
\usepackage{color}      
\usepackage{anysize}     
\usepackage{multicol}    
\usepackage{multirow}
\usepackage{bm}          
\usepackage{textcomp}   
\usepackage{eurosym}     
\usepackage{amsthm}     
\usepackage{amsmath,amsfonts} 
\usepackage{lineno} 


\marginsize{1.5cm}{1.5cm}{1.5cm}{1.5cm} 
\parindent=0mm                        
\parskip=3mm                         
\renewcommand{\baselinestretch}{1}    
\renewcommand{\spanishtablename}{Táboa}
 

\title{Tema 6: Técnicas operantes para o establecemento ou aumento de condutas}
\date{} 

\begin{document}   

\maketitle 

\section{Conceptos básicos relacionados coas técnicas operantes}
\begin{itemize}
	\item[•] \textbf{Resposta operante:} Resposta emitida libremente polo individuo que xera cambios 
	no ambiente, e cuxa probabilidade de emisión posterior ven determinada por ditos cambios.
	\item[•] \textbf{Continxencia:} Relación entre as condutas e os cambios ambientais ou 
	consecuencias xeradas por estas. 
	\item[•] \textbf{Reforzo positivo:} Increméntase unha conduta mediante a presentación continxente 
	dun estímulo agradable asociado a ela (reforzador positivo).
	\item[•] \textbf{Reforzador negativo:} Increméntase unha conduta mediante a eliminación 
	continxente dun estímulo desagradable ou aversivo asociado a ela (reforzador negativo).
	\item[•] \textbf{Castigo positivo:} Redúcese ou elimínase unha conduta mediante a presentación 
	continxente dun estímulo aversivo.
	\item[•] \textbf{Castigo negativo:} Redúcese ou elimínase unha conduta mediante a eliminación 
	continxente dun estímulo agradable.
	\item[•] \textbf{Extinción:} Redúcese ou elimínase unha conduta mediante a discontinuidade na 
	presentación dun reforzador positivo.
	\item[•] \textbf{Recuperación:} Icreméntase unha conduta mediante a eliminación dun estímulo 
	aversivo.
\end{itemize}

As condutas non están controladas só polas súas consecuencias, senon tamén polos estímulos que as preceden (antecedentes). En relación con esto, consideramos dous conceptos:
\begin{itemize}
	\item[$\circ$] \underline{Estímulos discriminativos}: Favorecen a emisión da conduta en 
	situacións semellantes a aquelas en que foi reforzada previamente.
	\item[$\circ$] \underline{Estímulos delta}: Dificultan a emisión da conduta cando esta non foi 
	reforzada en situacións semellantes.
\end{itemize}

\section{Programas de reforzo}
\begin{enumerate}
	\item \underline{Definición da conduta}: En termos operativos, observables e medibles. Débese 
	establecer a liña base (conduta inicial), os obxectivos (conduta obxectivo) e as condutas 
	intermedias (dependen da liña base, da dificultade da conduta obxectivo e das habilidades do 
	paciente).
	\item \underline{Búsqueda e selección de reforzadores}: Convén empregar varios reforzadores para 
	evitar a saciación. Estes deben ser agradables para o paciente, suficientemente potentes e 
	dispoñibles no medio habitual, pero só para o administrador do programa. Os reforzadores poden 
	ser comestibles, tanxibles, cambiables, actividades e sociais. A súa administración debe ser 
	continxente á conduta, e en cantidades altas inicialmente, que se irán reducindo co tempo.
	\item \underline{Selección do tipo de programa}: Existen diferentes tipos de programas.
	\begin{itemize}
		\item \textbf{Continuos:} Refórzanse tódalas aparicións da conduta desexada. Axeitados para 
		a fase de adquisición da conduta.
		\item \textbf{Intermitentes:} Refórzanse só algunhas emisións da conduta. Axeitados unha vez 
		que se aprendeu a conduta. Este tipo de programas favorecen a resistencia á extinción, a 
		evitación da saciación (ou o retraso na súa presentación), o rendemento estable e o 
		mantemento, xeralización e retirada do programa. Dentro destes, atopamos:
		\begin{itemize}
			\item \underline{Programas de razón}: Razón fixa (refórzase a conduta despois dun número 
			fixo de emisións) ou razón variable (refórzase a conduta despois dun número variable de 
			emisións, en torno a un promedio).
			\item \underline{Programas de intervalo}: Intervalo fixo (refórzase a conduta despois dun 
			determinado periodo de tempo) ou intervalo variable (refórzase a conduta despois dun 
			periodo temporal que varía en torno a un promedio). 
			\item \underline{Programas de duración}: Duración fixa (refórzase a conduta despois de 
			ter persistido esta durante un tempo determinado) ou duración variable (refórzase a 
			conduta despois de ter persistido esta durante un tempo variable en torno a un promedio).
		\end{itemize}
	\end{itemize}
	\item \underline{Aspectos contextuais}: Convén que o programa se leve a cabo no medio habitual do 
	paciente. Os reforzadores materiais deben ir acompañados de reforzo social. 
\end{enumerate}

\section{Modelado ou aprendizaxe por aproximacións sucesivas}
Refórzanse as aproximacións sucesivas a unha conduta final obxectivo, así como a extinción de condutas previas.
\begin{enumerate}
	\item Definir a conduta final, tendo en conta as súas características e as circunstancias en 
	que debe e non debe realizarse. 
	\item Definir a conduta inicial, que debe ser o suficientemente frecuente para poder ser 
	reforzada.
	\item Definir as condutas intermedias. O número de condutas intermedias dependerá da dificultade 
	das condutas inicial e final e das habilidades e recursos do paciente.
	\item Determinar o tamaño de cada paso e o tempo de permanencia nel.
	\item Selección de reforzadores.
\end{enumerate}

\section{Encadeamento}
Descompóñense condutas complexas noutras máis sinxelas, para traballalas por separado. 
\begin{enumerate}
	\item Definir a conduta final.
	\item Definir a conduta inicial.
	\item Dividir a cadea en diferentes unidades, cuxo tamaño dependerá do caso concreto.
	\item Elixir o método de encadeamento.
	\begin{itemize}
		\item \textbf{Encadeamento de tarefa completa:} Repítense tódolos pasos da cadea en cada 
		ensaio ata chegar a aprendelos.
		\item \textbf{Encadeamento cara adiante:} Execútase o primeiro paso e refórzase, logo o 
		primeiro e o segundo e refórzanse, e así ata completar a cadea.
		\item \textbf{Encadeamento cara atrás:} Execútase o último paso e refórzase, logo o 
		penúltimo e o último e refórzanse, e así ata completar a cadea.
	\end{itemize}
	\item Seleccionar os reforzadores.
	\item Implementar o procedemento e continuar co reforzo ata que o paciente aprenda a tarefa. 
\end{enumerate}

\section{Esvaecemento}
Permite que as condutas se manteñan a longo prazo, en ausencia dun apoio externo ou instigador. 
\begin{enumerate}
	\item Fase aditiva, na que se introduce unha axuda para que o paciente aprenda a conduta.
	\item Fase sustractiva, na que se retira progresivamente a axuda unha vez que o paciente vai 
	aprendendo e consolidando a conduta. 
\end{enumerate}

As axudas ou instigadores que se empregan poden ser de varios tipos:
\begin{itemize}
	\item[$\cdot$] \underline{Verbais}: Instrucións verbais sobre como realizar a conduta.
	\item[$\cdot$] \underline{Xestuais ou condutuais}: Movementos do terapeuta que axudan a emitir a 
	conduta, en ausencia de contacto co suxeito.
	\item[$\cdot$] \underline{Ambientais}: Cambiar as condicións do contexto para favorecer a emisión 
	da conduta.
	\item[$\cdot$] \underline{Físicas}: Guiar físicamente (por contacto) a conduta do paciente. 
\end{itemize}

Aínda que o emprego do reforzo negativo non é moi común na clínica, pode empregarse nalgúns casos facendo que a emisión da conduta obxectivo elimine algún estímulo desagradable presente no ambiente do paciente. Tanto o condicionamento de escape coma o evitativo están relacionados co incremento de condutas por reforzo negativo. Tamén pode empregarse integrado en técnicas de redución de condutas. 


\end{document}