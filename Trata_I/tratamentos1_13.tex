\documentclass[a4paper,11pt]{article}
\usepackage[spanish]{babel}        
\usepackage[utf8]{inputenc}           


\usepackage[T1]{fontenc} 
\usepackage{graphicx}    
\usepackage{color}      
\usepackage{anysize}     
\usepackage{multicol}    
\usepackage{multirow}
\usepackage{bm}          
\usepackage{textcomp}   
\usepackage{eurosym}     
\usepackage{amsthm}     
\usepackage{amsmath,amsfonts} 
\usepackage{lineno} 


\marginsize{1.5cm}{1.5cm}{1.5cm}{1.5cm} 
\parindent=0mm                        
\parskip=3mm                         
\renewcommand{\baselinestretch}{1}    
\renewcommand{\spanishtablename}{Táboa}
 

\title{Tema 13: Autoinstrucións}
\date{} 

\begin{document}   

\maketitle 

\section{Bases teóricas}
O adestramento en autoinstrucións de Meichenbaum desenvólvese dentro dun modelo cognitivo-condutual. É unha técnica cognitiva coa que se pretende modificar as respostas emocionais e condutuais mediante o control das autoverbalizacións existentes, e o cambio destas por outras máis útiles.

Consiste nun conxunto de ordes que o suxeito se da a si mesmo para manexar a súa conduta ante determinadas situacións. O obxectivo é modificar o diálogo interno da persoa para facilitar o afrontamento de situacións difíciles. 

Meichenbaum e Goodman obtiveron bos resultados empregando esta técnica en pacientes esquizofrénicos e hiperactivos.

Os antecedentes desta técnica son os traballos de Luria e Vygotski sobre o papel da linguaxe como proceso autorregulador do comportamento infantil. Estes autores describían tres etapas para explicar a iniciación e inhibición da conduta motora dos nenos a través da linguaxe. 
\begin{enumerate}
	\item A conduta do neno está dirixida polas instrucións dos adultos.
	\item O neno guía, en gran parte, a súa propia conduta, a través de verbalizacións en voz alta 
	mentres actúa sobre como levar a cabo as accións para acadar os seus obxectivos.
	\item O neno guía a súa propia conduta a través da linguaxe encuberta. 
\end{enumerate}

\section{Adestramento en autoinstrucións}
\begin{enumerate}
	\item[\textbf{1.}] \textbf{Modelado cognitivo:} O terapeuta executa a tarefa mentres se da 
	instrucións en voz alta. O paciente aprende por observación.
	\item[\textbf{2.}] \textbf{Guía externa en voz alta:} O terapeuta guía a conduta do paciente 
	mediante instrucións en voz alta.
	\item[\textbf{3.}] \textbf{Autoinstrucións en voz alta:} O paciente repite a acción practicada 
	guiándose polas súas propias instrucións en voz alta, mentres o terapeuta o orienta, corrixe e 
	reforza.
	\item[\textbf{4.}] \textbf{Autoinstrucións enmascaradas:} O paciente repite a acción guiándose 
	polas mesmas instrucións, pero esta vez en voz baixa.
	\item[\textbf{5.}] \textbf{Autoinstrucións encubertas:} O paciente guía a súa conduta mediante 
	autoinstrucións internas, pensándoas sen expresalas verbalmente. 
\end{enumerate}

\section{Secuencia de autoinstrucións}
\begin{enumerate}
	\item Definir o problema (\textit{que teño que facer?}).
	\item Guía de resposta e planificación dunha estratexia de execución (\textit{como o fago?}).
	\item Focalización da atención nas directrices que guían a execución.
	\item Autorreforzo.
	\item Autocorrección e posibles alternativas para solucionar os erros cometidos.
\end{enumerate}

\section{Consideracións prácticas}
O adestramento debe ser axeitado para as características do suxeito, a súa psicopatoloxía, a conduta específica a modificar e a dificultade da tarefa. É importante coñecer cales son as autoverbalizacións do suxeito cando realiza unha tarefa, para preservar as útiles, eliminar as inaxeitadas e instaurar outras máis axeitadas. O terapeuta debe promover a práctica continuada en diferentes situacións, para fomentar a xeralización. Requírese a implicación activa do suxeito.

Cando se traballa con nenos, convén aplicar a técnica nas primeiras horas do día para evitar as distraccións por fatiga e aumentar a práctica ó longo da xornada. É mellor traballar con grupos reducidos de nenos (entre 3 e 5). Empregar grabacións dos propios nenos ou doutros facilita o adestramento, ó proporcionar \textit{feedback} e reforzo. É importante respetar o ritmo de cada neno e asegurarse de que interioricen as autoinstrucións, en ausencia de memorización mecánica. Recoméndase apoiarse en técnicas de imaxinación, aproximacións sucesivas, etc.

\end{document}