\documentclass[a4paper,11pt]{article}
\usepackage[spanish]{babel}        
\usepackage[utf8]{inputenc}           


\usepackage[T1]{fontenc} 
\usepackage{graphicx}    
\usepackage{color}      
\usepackage{anysize}     
\usepackage{multicol}    
\usepackage{bm}          
\usepackage{textcomp}   
\usepackage{eurosym}     
\usepackage{amsthm}     
\usepackage{amsmath,amsfonts} 
\usepackage{lineno} 


\marginsize{1.5cm}{1.5cm}{1.5cm}{1.5cm} 
\parindent=0mm                        
\parskip=3mm                         
\renewcommand{\baselinestretch}{1}    
 

\title{Tema 3: Técnicas de control da activación}
\date{} 

\begin{document}   

\maketitle 

As técnicas de relaxación son procedementos para ensinar ó suxeito a controlar o nivel de activación a través da modificación directa das condicións fisiolóxicas. Estas técnicas permiten diminuir a tensión muscular, a frecuencia e intensidade do ritmo cardíaco, a frecuencia respiratoria, os niveis de secreción de adrenalina e noradrenalina e o metabolismo basal, e aumentar a vasodilatación arterial. 

\section{Técnicas de control da respiración}
Instruccións:
\begin{itemize}
	\item[-] Realización en condicións óptimas: roupa, posición e asento cómodos, temperatura 
	agradable, ollos pechados, ausencia de ruido, etc.
	\item[-] Concentrarse nas sensacións corporais antes de iniciar o exercicio, visualizando unha 
	escea agradable.
	\item[-] A duración de cada exercicio é de entre 2 e 4 minutos, e a do periodo de descanso de 
	entre 1 e 2 minutos. 
	\item[-] Cada ciclo repítese 3 ou 4 veces.
	\item[-] Debe levarse a cabo polo menos dúas veces ó día, durante polo menos 5 minutos cada vez.
	\item[-] Se ocorre un episodio de hiperventilación, detense o exercicio e respírase nunha bolsa 
	de plástico, para reducir a cantidade de osíxeno inspirado. 
	\item[-] Convén realizar os exercicios en lugares distintos para aprender a poñelos en práctica 
	en calquera sitio, de ser necesario.
\end{itemize}

Procedemento:
\begin{enumerate}
	\item Un tempo. Dirixir o aire inspirado cara a parte inferior dos pulmóns. Colocando unha man 
	enriba do ventre e outra enriba do estómago, só debe moverse a primeira.
	\item Dous tempos. Dirixir o aire inspirado cara a parte inferior e media dos pulmóns. Nótase 
	primeiro o movemento da man do ventre e logo o da man do estómago.
	\item Tres tempos. Inspiración completa enchendo de aire o ventre, o estómago e o peito, 
	respectivamente.
	\item Tres tempos con expiración. Realízase unha inspiración segundo o exercicio tres e expírase 
	levantando os hombros e pechando un pouco os labios, como resoplando.
	\item Inspiración e expiración. Inspiración continua, sen marcar os tempos, e expiración cada vez 
	máis silenciosa. Alternancia respiratoria. Prácticar en diferentes posicións.
	\item Xeralización. Convén repetir algunha palabra ou frase que facilite a realización dos 
	exercicios. 
\end{enumerate}

\section{Visualización}
\begin{enumerate}
	\item Sentarse ou recostarse tranquilamente, mantendo unha respiración pausada.
	\item Visualizar unha situación que transmita calma, tratando de experimentala con tódolos 
	sentidos. 
\end{enumerate}

\section{Relaxación diferencial}
Algunhas consideracións prácticas:
\begin{itemize}
	\item \textbf{Condicións ambientais:} Habitación tranquila, ausencia de ruido e interrupcións, 
	temperatura agradable, luz tenue.
	\item \textbf{Atuendo:} Roupa cómoda, quitar reloxos, xoias e zapatos.
	\item \textbf{Postura:} Relaxada, estando sentado ou tumbado. 
	\item \textbf{Abandono activo:} Suprimir o control de pensamentos.
	\item \textbf{Focalización da atención:} Nas propias sensacións e na voz do terapeuta, que debe 
	ser suave, tranquila e monótona. En todo momento debe indicar o que se vai facer.
	\item \textbf{Diario de autorrexistro:} Inclúe a hora de inicio e fin de cada sesión e os 
	exercicios realizados. 
	\item \textbf{Práctica regular:} A aprendizaxe realízase de forma constante. Pode ser útil o uso 
	de instruccións grabadas para apoiar as primeiras fases do adestramento na casa.
\end{itemize}

\section{Relaxación progresiva}
Baséase en dúas instruccións ou ordes, que se realizan dúas veces con cada parte do corpo.
\begin{itemize}
	\item[$\diamond$] \underline{Instrucción de tensión}: Facer forza nunha parte do corpo 
	determinada e manter a tensión entre 5 e 10 segundos.
	\item[$\diamond$] \underline{Instrucción de relaxación}: Relaxar dita parte do corpo e manter ese 
	estado entre 20 e 30 segundos.
\end{itemize}

Algúns problemas que poden surxir durante esta práctica son: 
\begin{itemize}
	\item[•] \textbf{Calambres:} Xerar menos tensión.
	\item[•] \textbf{Movementos:} Ignoralos se non son moi frecuentes ou resolvelos falando co 
	paciente.
	\item[•] \textbf{Charla e risa:} Ignoralas, comunicarse só a través de sinais manuais. Se 
	persisten, analizar o ocorrido co paciente ó final da sesión.
	\item[•] \textbf{Ruidos no exterior:} Ignoralos.
	\item[•] \textbf{Espasmos e tics:} Repetir que a relaxación vai ben, tratando de que cesen.
	\item[•] \textbf{Pensamentos perturbadores:} Aumentar a parte falada do terapeuta para que o 
	paciente se concentre na súa voz, e non nestes pensamentos.
	\item[•] \textbf{Que o paciente se durma.}
\end{itemize}

A relaxación progresiva pode realizarse con diferentes grupos musculares, que son:

$$\textbf{Relaxación muscular}\left\lbrace\begin{array}{l}
	\text{12 grupos musculares}\left\lbrace\begin{array}{l}
		\text{Extremidades superiores}\left\lbrace\begin{array}{l}
			\text{Man e antebrazo dominantes}\\
			\text{Bíceps dominante}\\
			\text{Man e antebrazo non dominante}\\
			\text{Bíceps non dominante}
		\end{array}\right.\\
		\text{Cabeza e pescozo}\left\lbrace\begin{array}{l}
			\text{Fronte}\\
			\text{Parte central da cara}\\
			\text{Mandíbula}\\
			\text{Pescozo}
		\end{array}\right.\\
		\text{Tronco}\left\lbrace\begin{array}{l}
			\text{Hombros, peito e espalda}\\
			\text{Estómago}
		\end{array}\right.\\
		\text{Extremidades inferiores}\left\lbrace\begin{array}{l}
			\text{Perna dereita}\\
			\text{Perna esquerda}
		\end{array}\right.
	\end{array}\right.\\
	\text{7 grupos musculares}\left\lbrace\begin{array}{l}
		\text{Extremidades superiores}\left\lbrace\begin{array}{l}
			\text{Brazo dereito}\\
			\text{Brazo esquerdo}
		\end{array}\right.\\
		\text{Cara}\\
		\text{Pescozo}\\
		\text{Tronco}\\
		\text{Extremidades inferiores}\left\lbrace\begin{array}{l}
			\text{Perna dereita}\\
			\text{Perna esquerda}
		\end{array}\right.
	\end{array}\right.\\
	\text{4 grupos musculares}\left\lbrace\begin{array}{l}
		\text{Extremidades superiores}\\
		\text{Cabeza e pescozo}\\
		\text{Tronco}\\
		\text{Extremidades inferiores}
	\end{array}\right.
\end{array}\right.$$

\section{Adestramento autóxeno}
Debe realizarse nun entorno físico axeitado e adoptando as medidas necesarias para que o paciente se sinta cómodo (roupa cómoda, ollos pechados, etc.). O paciente debe estar tumbado, preferiblemente. A súa actitude debe ser de concentración pasiva (deixarse levar polas sensacións), e debe manter as fórmulas verbais de relaxación na mente. As fórmulas repetiranse lentamente. O terapeuta terá que axustarse ó ritmo do paciente. Se surxe algún problema nun exercicio, continúase co procedemento e logo vólvese a el. Os pasos a seguir son:
\begin{enumerate}
	\item  \underline{Inicio de sesión}: Inducir a relaxación, procurando que o paciente se sinta 
	tranquilo, manteña os ollos pechados, manteña unha respiración regular e calmada e se concentre 
	nas súas sensacións corporais, relaxando o corpo de pés a cabeza.
	\item \underline{Introducir fórmulas de suxestión}: Repítese cada fórmula de suxestión seis veces 
	para cada parte do corpo, deixando tempo entre cada instrución para que o paciente se concentre 
	na sensación. Tras cada repetición, o paciente dise a si mesmo ``estou moi tranquilo''. 
	\item \underline{Exercicios}: Relaxación muscular a través da sensación de pesadez, relaxación 
	vascular a través da sensación de calor, normalización da actividade cardíaca, regulación do 
	aparato respiratorio, regulación dos órganos da rexión abdominal e redución do fluxo sanguíneo na 
	cabeza. 
	\item \underline{Finalización da sesión}: Conta atrás de catro a un. No catro aprétanse varias 
	veces os puños, no tres flexiónanse varias veces os brazos, no dous realízanse varias 
	inspiracións profundas e sóltase o aire, e no un ábrense os ollos. 
\end{enumerate}

Algunhas precaucións a ter en conta cando se emprega este tipo de relaxación son a facilitación de efectos de perda de contacto coa realidade (que poden dexenerar en estados disociativos, alucinacións e parentesias), o incremento do efecto de certas drogas e fármacos, a indución dunha desactivación excesiva e a evocación de pensamentos ou emocións que poden alterar ó paciente. 

\section{\textit{Mindfulness} ou atención}
Trátase dunha meditación budista de orixe monástico, trasladada a occidente por Jon Kabat-Zim. Fai referencia a un estado de atención e conciencia plenas, referidas ó momento presente; a unha actitude activa, amable e reflexiva de carácter non valorativo; a experiencias contemplativas, que implican aceptar a realidade tal e como se presenta; e á apertura á experiencia sensorial sen prexuizos. 

Os compoñentes principais do \textit{mindfulness} son:
\begin{itemize}
	\item[-] Centrarse no momento presente, sen tratar de cambialo nin de contemplar realidades 
	alternativas, aceptando as emocións e pensamentos (tanto positivos coma negativos) tal e como se 
	presentan.
	\item[-] Apertura á experiencia e ós feitos.
	\item[-] Aceptación radical das experiencias vitais, o cal implica aceptar tamén os 
	acontecementos desagradables como parte da historia da persoa.
	\item[-] Elección da experiencia, decidindo de forma activa a que atender segundo os propios 
	intereses.
	\item[-] Renunciar ó control directo das emocións e sentimentos, experimentándoos tal e como 
	surxen.
\end{itemize}

Este é un procedemento terapéutico promove o uso da meditación como base de diversas técnicas de relaxación fisiolóxica e emocional. Encádrase entre as terapias de terceira xeración, en tanto que emprega o contexto como elemento principal de explicación e intervención, promove estratexias de cambio de conduta de carácter indirecto e experencial, destaca a importancia da función da conduta sobre a súa forma e relaciónase coa observación.

Os efectos do \textit{mindfulness} son:
\begin{itemize}
	\item[$\circ$] \underline{Autorregulación}: Regúlanse as respostas emocionais e fisiolóxicas de 
	forma natural, permitindo a activación dos mecanismos de \textit{feedback} propios do organismo.
	\item[$\circ$] \underline{Aprendizaxe de novas respostas}: Extínguense o bloqueo e o control de 
	emocións, pensamentos e sensacións, e promóvense respostas alternativas.
	\item[$\circ$] \underline{Regulación emocional}: Obsérvanse e descríbense as propias emocións 
	para modificar a resposta automática ante a súa aparición.
	\item[$\circ$] \underline{Redución das crenzas}: Rexéitase a conduta gobernada por regras e 
	actúase en función de continxencias relacionadas cos sucesos.
	\item[$\circ$] \underline{Control da atención.}
\end{itemize}



\end{document}