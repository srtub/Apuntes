\documentclass[a4paper,11pt]{article}
\usepackage[spanish]{babel}        
\usepackage[utf8]{inputenc}           


\usepackage[T1]{fontenc} 
\usepackage{graphicx}    
\usepackage{color}      
\usepackage{anysize}     
\usepackage{multicol}    
\usepackage{multirow}
\usepackage{bm}          
\usepackage{textcomp}   
\usepackage{eurosym}     
\usepackage{amsthm}     
\usepackage{amsmath,amsfonts} 
\usepackage{lineno} 


\marginsize{1.5cm}{1.5cm}{1.5cm}{1.5cm} 
\parindent=0mm                        
\parskip=3mm                         
\renewcommand{\baselinestretch}{1}    
\renewcommand{\spanishtablename}{Táboa}
 

\title{Tema 11: Terapia condutual racional emotiva e reestruturación racional sistemática}
\date{} 

\begin{document}   

\maketitle 

Os antecedentes desta terapia son as terapias de reestruturación cognitiva da década dos 60. A TREC aparece en 1994, influenciada pola psicanálise, o pensamento estoico (a alteración emocional débese á interpretación que fai o suxeito da situación, e non á situación en si mesma), e o enfoque activo-directivo do condutismo. 

\section{Bases teóricas}
A TREC baséase na idea de que tanto as emocións como as condutas son produto das crenzas do individuo, isto é, da interpretación que este fai da realidade. Segundo isto, a causa dos problemas psicolóxicos atoparíase nun sistema de crenzas caracterizado por patróns de pensamento disfuncionais. Por tanto, o obxectivo desta terapia será identificar os pensamentos irracionais do paciente e substituilos por outros máis racionais, que lle permitan acadar as súas metas. 

Para explicar a xénese e o mantemento de problemas, Ellis propón o Modelo A-B-C, onde A é o acontecemento activador (suceso real e extremo da vida do suxeito), B é a cadea de pensamentos que surxe como consecuencia de A (valoración do acontecido) e C é o conxunto de consecuencias emocionais e condutuais que experimenta o paciente. Grazas á terapia, esta secuencia completarase co debate sobre as ideas irracionais (D), que dará lugar a novas emocións e a unha nova conduta (E). 

Un aspecto básico e crucial desta terapia é a distinción entre:
\begin{itemize}
	\item \textbf{Crenzas racionais:} Son probabilísticas, preferenciais ou relativas, e exprésanse 
	en termos de desexo ou gusto. 
	\item \textbf{Crenzas irracionais:} Son dogmáticas ou absolutas, e exprésanse en termos de 
	necesidade imperiosa e esixente. Destas crenzas derívanse tres inferencias ou procesos 
	secundarios irracionais nucleares:
	\begin{itemize}
		\item \underline{Tremendismo}: Tendencia a resaltar en exceso o negativo dos acontecementos.
		\item \underline{Baixa tolerancia á frustración}: Tendencia a esaxerar o insoportable dunha 
		situación, e a calificala como insufrible.
		\item \underline{Condenación ou avaliación global da valía do ser humano}: Tendencia a 
		condenar ás persoas, ou ó mundo en xeral, se non lle proporcionan ó individuo o que cre que 
		merece.
	\end{itemize}
	
	Do mesmo xeito, as crenzas irracionais dan lugar a dúas perturbacións psicolóxicas:
	\begin{itemize}
		\item \underline{Ansiedade do eu}: Crenzas relacionadas coa incompetencia ou descalificación 
		persoal.
		\item \underline{Ansiedade perturbadora}: Crenzas relacionadas con lograr os obxectivos 
		levando unha vida cómoda, sen moito esforzo. As demandas dirixidas ós demais.
	\end{itemize}
\end{itemize}

\section{Procedemento da TREC}
\begin{enumerate}
	\item \underline{Avaliación dos problemas e explicación do esquema A-B-C}: Mediante a entrevista, 
	comezan a determinarse os problemas internos, externos e secundarios, e identifícanse os 
	pensamentos irracionais a través dun diálogo didáctico. Nesta fase, preséntanselle ó paciente os 
	supostos que explican a xénese e o mantemento do seu problema, destacando o papel dos pensamentos 
	irracionais no malestar subxectivo e na conduta desadaptativa. Convén empregar exemplos.
	
	É importante deixarlle claro que aínda que as ideas irracionais foran aprendidas en experiencias 
	previas desagradables, a causa dos seus problemas non está nestas experiencias, senon no 
	mantemento das mesmas ideas á hora de interpretar os acontecementos. Para superar os problemas, 
	deberá aceptarse a si mesmo e adoptar un papel activo no cuestionamento das súas crenzas 
	irracionais.
	\item \underline{Detección das ideas irracionais}: Para facer conscientes as crenzas do paciente, 
	solicítaselle que rexistre o que se di a si mesmo cando ocorren os seus problemas, empregando 
	unha folla de autorrexistro.
	\item \underline{Debate e cambio das ideas irracionais}: O terapeuta cuestiona a veracidade dos 
	pensamentos irracionais do paciente, seguindo o método hipotético-deductivo a través de cinco 
	pasos:
	\begin{enumerate}
		\item[\textbf{a.}] \textbf{Empirismo:} Centrarse no que se pode comprobar.
		\item[\textbf{b.}] \textbf{Lóxica:} Partir de premisas verdadeiras.
		\item[\textbf{c.}] \textbf{Flexibilidade:} Capacidade para ver as cousas doutra maneira cando 
		os datos son diferentes.
		\item[\textbf{d.}] \textbf{Ausencia de valoración moral:} Estas non son científicas.
		\item[\textbf{e.}] \textbf{Probabilismo:} En ocasións, as cousas non son tan certas como se 
		cre.
	\end{enumerate}
	
	As preguntas empregadas para modificar as ideas irracionais relaciónanse coa análise da 
	utilidade, da validez e doutros puntos de referencia.
	\item \underline{Uso doutras técnicas durante o proceso terapéutico}: 
	\begin{itemize}
		\item \textbf{Redución ó absurdo:} Asumir unha crenza irracional como verdadeira, e levala ó 
		extremo de tal xeito que quede patente o ilóxico desta.
		\item \textbf{Reacción incrédula do terapeuta.}
		\item \textbf{Imaxinación racional emotiva:} Permite determinar cales son os pensamentos 
		racionais do paciente.
	\end{itemize}
	
	Algúns dos exercicios que se empregan durante a terapia son: exercicios para atacar a vergoña, 
	adestramento en autoinstrucións, ensaio de conduta e adestramento en habilidades sociais, 
	exposición ás situacións temidas, adestramento en solución de problemas, técnicas humorísticas e 
	tarefas para casa, compostas por tarefas cognitivas e condutuais (exposición a situacións 
	problema, escoitar e debatir grabacións, biblioterapia, ensinar a TREC a algún coterapeuta, 
	autorrexistros, etc.).
	
	Tamén é importante detectar se no paciente aparecen medos ou resistencias para irracionalmente. 
	Algúns medos comúns son o medo a volverse frío emocionalmente, o medo a perder a personalidade ó 
	cambiar as ideas, o medo a converterse en mediocre por abandonar crenzas perfeccionistas ou o 
	medo a perder gratificacións que obtén pola ``enfermidade''.
	\item \underline{Aprendizaxe dunha nova filosofía de vida}: O obxectivo é estabilizar as novas 
	crenzas racionais, poñendo en práctica o aprendido na vida real para fortalecer o hábito. É útil 
	seguir completando os autorrexistros para poder facer un seguimento.
\end{enumerate}




\end{document}