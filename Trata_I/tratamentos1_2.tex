\documentclass[a4paper,11pt]{article}
\usepackage[spanish]{babel}        
\usepackage[utf8]{inputenc}           


\usepackage[T1]{fontenc} 
\usepackage{graphicx}    
\usepackage{color}      
\usepackage{anysize}     
\usepackage{multicol}    
\usepackage{bm}          
\usepackage{textcomp}   
\usepackage{eurosym}     
\usepackage{amsthm}     
\usepackage{amsmath,amsfonts} 
\usepackage{lineno}     


\marginsize{1.5cm}{1.5cm}{1.5cm}{1.5cm} 
\parindent=0mm                        
\parskip=3mm                         
\renewcommand{\baselinestretch}{1}    
 

\title{Tema 2: Avaliación condutual e proceso de intervención terapéutica}
\date{} 

\begin{document}   

\maketitle 

\section{Análise da demanda}
Primeira descrición dos problemas e toma de primeiras decisións en función das variables observables (idade, nivel cultural, problemas...). Empréganse termos específicos e precisos e unha linguaxe descritiva para referirse ós problemas. Débense concretar as queixas a un ou dous problemas principais, e establecer prioridades entre eles:
\begin{enumerate}
	\item O que mellor indique a razón que levou ó paciente a slicitar axuda.
	\item Aquel cuxa resolución é máis importante para o paciente.
	\item Aquel cuxa probabilidade de ser resolto con éxito e pouco esforzo é maior.
	\item O que debe ser resolto antes de pasar a outros. 
\end{enumerate}

\section{Análise descritiva-dimensional ou topográfica}
\begin{itemize}
	\item[•] \textbf{Análise de secuencias e desenvolvemento do problema:} Como e cando apareceu e 
	aparece cada vez, variables que inflúen na aparición e en que orde, a vez máis grave, a 
	evolución.
	\item[•] \textbf{Descrición dos parametros da conduta problema:} Frecuencia, intensidade e 
	duración, a distintos niveis:
	\begin{itemize}
		\item[$\circ$] \underline{Condutual}: Condutas verbais, non verbais e motoras.
		\item[$\circ$] \underline{Cognitivo}: Actividade cognitiva (pensamentos, crenzas e diálogo 
		interno).
		\item[$\circ$] \underline{Afectivo}: Sentimentos e estado de ánimo.
		\item[$\circ$] \underline{Somático}: Sensacións corporais.
	\end{itemize}
	\item[•] \textbf{Variables do contexto:} Apoio social e recursos económicos e ambientais. 
\end{itemize}

\section{Análise da causalidade}
Consiste nunha análise funcional das variables que controlan a conduta.
\begin{itemize}
	\item[$\diamond$] \underline{Antecedentes}: Descrición das circunstancias que ocorren 
	inmediatamente antes da aparición da conduta. 
	\begin{itemize}
		\item[•] \textbf{Internos:} Condutuais, cognitivos, afectivos e somáticos.
		\item[•] \textbf{Externos:} Relacionais (presencia ou ausencia doutras persoas) e contextuais 
		(lugar, tempo e acontecementos concorrentes). 
	\end{itemize}
	\item[$\diamond$] \underline{Consecuentes}: Descrición das circunstancias que ocorren 
	inmediatamente despois da aparición da conduta. 
	\begin{itemize}
		\item[•] \textbf{Consecuencias positivas:} Reforzo positivo (o aumento dunha consecuencia 
		agradable que segue á conduta problema incrementa a probabilidade de aparición da mesma) e 
		negativo (o decremento dunha consecuencia desgradable que segue á conduta problema incrementa 
		a probabilidade de aparición da mesma).
		\item[•] \textbf{Consecuencias negativas:} Castigo positivo (a conduta vai seguida dun 
		estímulo desagradable) e negativo (desaparece un reforzador que aparecía tras a emisión da 
		conduta).
		\item[•] \textbf{Consecuencias a curto e longo prazo:} Baixo estado de ánimo (curto) e perda 
		de apoio social (longo).
	\end{itemize}
	\item[$\diamond$] \underline{Variables persoais}: Descricións de aspectos relevantes do organismo 
	que poden estar relacionados coa conduta problema. 
	\begin{itemize}
		\item[•] \textbf{Variables biolóxicas:} Previas (herdables, prenatais e perinatais) e actuais 
		(enfermidades, efectos de fármacos e substancias e estados transitorios que poden alterar o 
		equilibrio orgánico do paciente, como a fatiga). 
		\item[•] \textbf{Variables psicolóxicas:} Estilo comunicativo, atribucional e de 
		afrontamento, esquemas cognitivos, manexo de emocións, personalidade, cociente intelectual...
		\item[•] \textbf{Variables sociodemográficas e psicosociais.}
	\end{itemize}
\end{itemize}

\section{Análise descritiva-categorial}
Faise un diagnóstico, decidindo a categoría diagnóstica máis axeitada segundo os cinco eixos ou categorías de clasificación multicanal do DSM-IV-TR.

\section{Análise histórica do problema}
\begin{itemize}
	\item \textbf{Predispoñentes:} Factores previos, biolóxicos e sociais, que incrementan o risco de 
	presentar o problema.
	\item \textbf{Orixe do problema:} Momento de inicio e factores desencadeantes.
	\item \textbf{Evolución do problema:} Duración, episodios previos, recaídas, tratamentos previos 
	e fracasos e éxitos terapéuticos previos. 
\end{itemize}

\section{Análise de recursos}
Apoio social (amigos, familia, etc.), recursos económicos e recursos ambientais (onde vive, onde traballa, en que ambientes se move, etc.). 

\section{Formulación clínica do caso}
\begin{itemize}
	\item[-] Etioloxía (factores de inicio, desenvolvemento do problema, predisposición do paciente a 
	padecelo, vulnerabilidade, desencadeantes...).
	\item[-] Curso do problema (como se adquire e se mantén).
	\item[-] Mantemento (modelos).
	\item[-] Pronóstico con e sen tratamento.
\end{itemize}

\section{Deseño e planificación do plan de intervención}
\begin{itemize}
	\item[$\circ$] \underline{Determinación de obxectivos terapéuticos}: A curto e longo prazo. 
	Priorizar os obxectivos segundo a inmediatez dos efectos terapéuticos, o custo, as posibilidades 
	de xeralización, etc.
	\item[$\circ$] \underline{Selección das técnicas de tratamento}: Delimitar os pasos e a orde a 
	seguir.
\end{itemize}

\section{Avaliación dos recursos obtidos e seguimento}
\begin{itemize}
	\item[$\rightarrow$] Valoración subxectiva (lográronse os obxectivos previstos?)
	\item[$\rightarrow$] Significación clínica dos cambios.
	\item[$\rightarrow$] Criterio experimental: comparación normativa, avaliación subxectiva e 
	impacto social (continúa a melloría do paciente?).
	\item[$\rightarrow$] Seguimento (na consulta, por teléfono ou por correo). 
\end{itemize}

\section{Técnicas de recollida de información}
\subsection{Entrevista}
O procedemento da entrevista condutual pode resumirse en dous grandes bloques:
\begin{itemize}
	\item[•] \textbf{Preámbulo:} Explícaselle ó paciente a utilidade do tratamento, solicítase 
	formalmente a súa colaboración e establécense os compromisos de veracidade, confidencialidade e 
	respecto da intimidade. 
	\item[•] \textbf{Pasos:} Análise da demanda (interacción inicial, na que se observan as 
	características físicas e modo de actuar do paciente, delimitación do problema e importancia do 
	problema), análise descritiva-dimensional, análise da causalidade (antecedentes, consecuentes e 
	variables persoais), análise descritiva-categorial, análise histórica do problema e análise de 
	recursos.
\end{itemize}

Os aspectos formais da entrevista son:
\begin{itemize}
	\item[$\odot$] \underline{Condución semiestruturada e directiva}: Inicialmente máis aberta e 
	pouco directiva (recóllese información global), e progresivamente máis pechada e directiva 
	(preguntas máis concretas). 
	\item[$\odot$] \underline{Relación terapeuta-paciente}: Cordial pero non amistosa, empática, 
	cálida e de colaboración.
	\item[$\odot$] \underline{Uso de reforzo}: Só en condutas específicas e momentos específicos.
	\item[$\odot$] \underline{Linguaxe}: Axeitada ó interlocutor, isto é, comprensible para el.
	\item[$\odot$] \underline{Duración}: Arredor dunha hora, aínda que a entrevista inicial adoita 
	ser máis longa.
	\item[$\odot$] \underline{Rexistro da información}: Preferentemente de forma automática 
	(grabacións), aínda que pode tomarse algunha nota. Escribir de forma continua durante a sesión 
	dominúe a confianza do paciente.
\end{itemize}

Unha vez se recollen tódolos datos na entrevista condutual, debe realizarse unha entrevista de devolución. Esta consta dos seguintes pasos:
\begin{enumerate}
	\item Resúmense os problemas detectados.
	\item Explícanse os mecanismos de adquisición e mantemento do problema, ilustrando como o causan.
	\item Enfatízase o feito de que os problemas son maioritariamente aprendidos, e por tanto poden 
	desaprenderse.
	\item Infórmase das posibles estratexias de tratamento, mencionando os pros e contras de cada 
	unha.
	\item Confírmase que o paciente comprendeu tódolos aspectos expostos e indágase sobre a 
	necesidade de cambio que este ten.
\end{enumerate}

\subsection{Observación}
\begin{enumerate}
	\item Definición da conduta obxectivo, de forma clara, precisa e mensurable.
	\item Selección dos parámetros de conduta que deben avaliarse e elección dun método de medición, 
	que pode ser de:
	\begin{itemize}
		\item \textbf{Frecuencia:} Número de veces que aparece unha conduta nun determinado periodo 
		de tempo. Axeitado para condutas discretas, de duración semellante en cada emisión e non moi 
		emitidas (\textit{exemplos:} cigarros fumados, noites que molla a cama, ausencias laborais). 
		\item \textbf{Duración:} Lapso de tempo no que se emite unha conduta. Axeitado para condutas 
		discretas e de duración variable (\textit{exemplos:} tempo de estudo, horas de sono). 
		\item \textbf{Intervalo:} Pode ser de tres tipos:
		\begin{itemize}
			\item \underline{Completo}: Presencia ou ausencia da conduta durante todo o intervalo. 
			Axeitado para condutas que persisten durante un tempo e cuxa non interrupción é 
			importante (\textit{exemplo:} atender en clase).
			\item \underline{Parcial}: Presencia da conduta polo menos unha vez ó longo do intervalo. 
			Axeitado para condutas breves e frecuentes (\textit{exemplo:} tics nerviosos).
			\item \underline{Momentáneo}: A emisión da conduta pon fin ó intervalo. Axeitado para 
			condutas que persisten durante longos periodos temporais (\textit{exemplo:} chupar o 
			dedo). 
		\end{itemize}
		\item \textbf{Proporción:} Porcentaxe de veces que aparece unha conduta dentro do número 
		total de ocasións en que sería esperable a súa aparición. Permite obter medidas relativas de 
		rendemento, actuación, etc. (\textit{exemplos:} tarefas realizadas das encomendadas, ordes 
		cumplidas). 
		\item \textbf{Produtos permanentes:} Número de resultados duradeiros da conduta. Produce 
		pouca relatividade, pero só pode aplicarse a un limitado número de condutas 
		(\textit{exemplos:} camas feitas, libros colocados). 
		\item \textbf{Secuencia:} Valora a conduta, os seus antecedentes e os seus consecuentes. 
		Proporciona información funcional sobre a conduta. Require observacións continuas 
		(\textit{exemplo:} medo ós cans).
	\end{itemize}
	\item Confección da folla de rexistro.
	\item Especificación dos aspectos contextuais (onde, cando e con quen). 
	\item Adestramento dos observadores.
	\item Observación e rexistro.
	\item Avaliación da fiabilidade do rexistro (cálculo e interpretación do índice de fiabilidade 
	pertinente para cada caso). 
	\item Análise da información recollida. 
\end{enumerate}

As fontes de erro da observación son:
\begin{itemize}
	\item[$\circ$] \underline{O suxeito observado}: Debido á reactividade (cambio de comportamento do 
	suxeito cando sabe que o están observando).
	\item[$\circ$] \underline{O observador}: Debido ó adestramento, ó efecto halo (como me cae o 
	suxeito observado) ou ó efecto deriva (familiaridade co suxeito observado). 
	\item[$\circ$] \underline{O sistema de observación}: Debido ó tipo de rexistro, á perda de 
	información ou á representatividade dos datos. 
\end{itemize}

\subsection{Auto-observación}
\begin{enumerate}
	\item Definición da conduta obxectivo, de forma clara, precisa e mensurable.
	\item Elección dun método de medición (frecuencia, duración, intervalo, proporción, produtos 
	permanentes ou secuencias).
	\item Confección da folla de rexistro. 
	\item Adestramento do suxeito para observar as condutas obxectivo e cubrir a folla de rexistro, 
	proporcionándolle instruccións sobre como, cando e onde observar. 
	\item Representación gráfica dos datos do autorrexistro. 
\end{enumerate}

\subsection{Habilidades terapéuticas}
\begin{itemize}
	\item[•] \textbf{Actitudes que favorecen a relación terapéutica:} Calidez, autenticidade e 
	empatía.
	\item[•] \textbf{Habilidades non verbais:} Voz modulada, suave e firme, animación da expresión 
	facial, mirada segura e directa (pero non fixa), asentir de vez en cando, sorrisos intermitentes, 
	gestos ocasionais coas mans, velocidade da fala moderada, emprego intermitente de expresións 
	indicativas de atención, postura relaxada e dirixida ó paciente, etc.
	\item[•] \textbf{Habilidades verbais de escoita:} 
	\begin{itemize}
		\item \underline{Clarificación}: Solicitar ó paciente que especifique ou aclare o significado 
		dalgún aspecto. 
		\item \underline{Paráfrase}: Repetir con outras palabras unha idea expresada polo paciente.
		\item \underline{Reflexo}: Verbalizar de forma empática que comprendemos o que lle ocorre ó 
		paciente.
		\item \underline{Síntese}: Recapitulación da información proporcionada polo paciente.
	\end{itemize}
	\item[•] \textbf{Habilidades verbais de acción:} 
	\begin{itemize}
		\item \underline{Preguntas}: Permiten intercambiar información.
		\item \underline{Información}: Proporcionada polo terapeuta para aclarar aspectos que o 
		paciente descoñece, como as posibles opcións que se lle presentan.
		\item \underline{Interpretación}: O terapeuta extrae conclusións dos datos proporcionados 
		polo paciente. 
		\item \underline{Confrontación}: O terapeuta amosa ó paciente as posibles discrepancias entre 
		o que pensa e sinte, ou entre o que di e fai. 
	\end{itemize}
	\item[•] \textbf{Actitudes que dificultan o proceso terapéutico:} Amosarse inseguro; actitudes 
	frías e distantes, xuizos de valor, amosarse autoritarios, criticar e regañar; desexar ser 
	apreciado polo paciente; querer ir máis rápido do que o paciente é capaz; tomar unha dirección 
	equivocada, alonxándose da preocupación principal do paciente; non tocar temas conflitivos por 
	medo á crítica; non saber por onde seguir; facer demasiadas preguntas; interrumpir prematuramente 
	os silencios; tolerar mal as demostracións emotivas e afectivas do paciente. 
\end{itemize}

%Correxir os símbolos dos ítems nesta lista, que son todos círculos.


\end{document}