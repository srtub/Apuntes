\documentclass[a4paper,11pt]{article}
\usepackage[spanish]{babel}        
\usepackage[utf8]{inputenc}           


\usepackage[T1]{fontenc} 
\usepackage{graphicx}    
\usepackage{color}      
\usepackage{anysize}     
\usepackage{multicol}    
\usepackage{multirow}
\usepackage{bm}          
\usepackage{textcomp}   
\usepackage{eurosym}     
\usepackage{amsthm}     
\usepackage{amsmath,amsfonts} 
\usepackage{lineno} 


\marginsize{1.5cm}{1.5cm}{1.5cm}{1.5cm} 
\parindent=0mm                        
\parskip=3mm                         
\renewcommand{\baselinestretch}{1}    
\renewcommand{\spanishtablename}{Táboa}
 

\title{Tema 15: Terapia de solución de problemas}
\date{} 

\begin{document}   

\maketitle 

\section{Bases teóricas}
A TSP surxe nos anos 70. Os seus creadores enténdena como un proceso cognitivo-afectivo-comportamental mediante o cal a persoa comprende os seus problemas e se esforza por alterar a natureza problemática da súa situación, poñendo en marcha respostas eficaces para manexala.

A terapia fai referencia a tres conceptos:
\begin{itemize}
	\item[•] \textbf{O problema:} Situación, actividade ou tarefa, presente ou anticipada polo 
	individuo, que demanda unha resposta efectiva para o seu funcionamento adaptativo da que non se 
	dispón. 
	\item[•] \textbf{A solución:} Resposta dirixida a alterar a natureza do problema, as reaccións 
	emocionais negativas que este provoca ou ambas.
	\item[•] \textbf{A solución de problemas sociais:} Proceso cognitivo-condutual dirixido a 
	identificar solucións adaptativas un problema. É unha actividade consciente, racional e dirixida 
	a un fin, que implica esforzos por parte do individuo. 
\end{itemize}

Nunha situación problemática existen tres fontes de activación emocional: 
\begin{itemize}
	\item[-] O problema obxectivo (enfermidade, paro, etc.).
	\item[-] A orientación cara o problema, que será máis ou menos favorable en función de se o 
	individuo percible o problema como solucionable ou non, e de se considera que posúe habilidades 
	para resolvelo ou non.
	\item[-] O estilo de solución de problemas, que pode ser racional-construtivo, 
	impulsivo-descoidado ou evitativo. 
\end{itemize}

\section{Procedemento da TSP}
\begin{enumerate}
	\item \underline{Establecer unha axeitada relación terapéutica}: Unha relación terapéutica 
	positiva favorece a detección eficaz de problemas e a resolución dos mesmos de forma adaptativa. 
	\item \underline{Avaliación condutual e formulación do caso}: Mediante a entrevista recóllense 
	tódolos datos do paciente, entre os que se inclúen o estilo de solución de problemas e a 
	orientación cara o problema concreto. Para detectar o estilo, pódese aplicar o \textit{Inventario 
	de solución de problemas sociais}. 
	\item \underline{Adestramento en solución de problemas}: Pódese aplicar como tratamento único ou 
	como técnica complementaria a outras dun tratamento. Pode aplicarse de forma individual ou 
	grupal. 
	\item \underline{Pasos principais da terapia de solución de problemas}:
	\begin{itemize}
		\item[\textbf{a.}] \textbf{Orientación cara o problema:} Os obxectivos son ensinar ó paciente 
		a detectar os problemas, a centrar a atención en condutas positivas de solución de problemas 
		e a alonxala de preocupacións improdutivas, a maximizar o esforzo para superar o estrés 
		emocional e obter estados emocionais positivos e a minimizar a angustia emocional. As 
		variables implicadas son a percepción do problema, as atribucións causais sobre este, o 
		control persoal e a compromiso tempo-esforzo. 
		\item[\textbf{b.}] \textbf{Definición e formulación do problema:} Buscar información sobre o 
		problema a partir de feitos e identificar tódolos factores relacionados co problema, 
		reavaliar o significado do problema para aumentar o benestar do paciente e establecer metas 
		realistas para solucionar o conflito. 
		\item[\textbf{c.}] \textbf{Xeración de solucións alternativas:} Consiste en xerar moitas 
		solucións diferentes para aumentar a probabilidade de aparición da máis axeitada. Para xerar 
		as solucións séguense tres principios: principio de cantidade, principio de aprazamento de 
		xuizo e principio de variedade.
		\item[\textbf{d.}] \textbf{Toma de decisións:} Avalíanse os pros e contras das distintas 
		solucións e selecciónase a mellor de todas, tendo en conta o benestar físico e emocional do 
		paciente, o benestar das persoas do entorno, o tempo e esforzo requeridos a curto e longo 
		prazo e a adecuación ós propios valores e ó medio. 
		\item[\textbf{e.}] \textbf{Posta en práctica e verificación da solución:} Primeiro ponse en 
		práctica en imaxinación, e logo trasládase á situación real. Tras a posta en práctica, o 
		paciente deberá observar a súa conduta e o resultado obtido, comparar este resultado co que 
		esperaba e reforzarse polos seus esforzos. Se o resultado é satisfactorio, finalizará o 
		proceso de solución de problemas. En caso contrario, deberá retroceder a etapas anteriores 
		para detectar posibles erros.
	\end{itemize}
	\item \underline{Tarefas para casa e novos exercicios de solución de problemas}.
	\item \underline{Mantemento de resultados e prevención de recaídas}.
	\item \underline{Detección de problemas máis frecuentes e das súas solucións mediante a TSP}. 
\end{enumerate}









\end{document}