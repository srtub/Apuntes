\documentclass[a4paper,11pt]{article}
\usepackage[spanish]{babel}        
\usepackage[utf8]{inputenc}           


\usepackage[T1]{fontenc} 
\usepackage{graphicx}    
\usepackage{color}      
\usepackage{anysize}     
\usepackage{multicol}    
\usepackage{multirow}
\usepackage{bm}          
\usepackage{textcomp}   
\usepackage{eurosym}     
\usepackage{amsthm}     
\usepackage{amsmath,amsfonts} 
\usepackage{lineno} 


\marginsize{1.5cm}{1.5cm}{1.5cm}{1.5cm} 
\parindent=0mm                        
\parskip=3mm                         
\renewcommand{\baselinestretch}{1}    
\renewcommand{\spanishtablename}{Táboa}
 

\title{Tema 4: Desensibilización sistemática}
\date{} 

\begin{document}   

\maketitle 

\section{Definición e antecedentes}
A técnica de desensibilización sistemática consiste na presentación gradual de estímulos ansióxenos en imaxinación, combinada co adestramento dunha resposta alternativa e antagónica á ansiedade, como é a relaxación muscular. 

Os autores propoñen que as respostas de ansiedade emitidas ante certas situacións nacen a raíz dun proceso de condicionamento: unha situación que inicialmente non provocaba estrés pode converterse en estresora se se asocian a ela consecuencias negativas. Por exemplo, se a primeira vez que falamos ante un público escoitamos risas ou barullo, podemos interpretar que estamos facendo algo mal e que se están burlando de nós. Isto provócanos ansiedade, e a seguinte vez que nos teñamos que enfrontar a dita situación, evocaremos esta sensación e outras relacionadas, como estrés e agobio.

Esta técnica favorece o desenvolvemento de pensamentos realistas, o cambio de expectativas e o incremento da autoeficacia. 

\section{Fundamentos teóricos e experimentais da DS}
\begin{itemize}
	\item \textbf{Contracondicionamento (Wolpe):} Asóciase un estímulo ansióxeno cunha resposta 
	incompatible, para provocar a substitución da resposta inicial (ansiedade) pola incompatible 
	(relaxación).
	\item \textbf{Inhibición recíproca (Wolpe):} Baséase nun concepto fisiolóxico. Se unha resposta 
	inhibidora de ansiedade ocorre en presencia de estímulos ansióxenos debilitará os vínculos 
	existentes entre estes estímulos e a ansiedade.
	\item \textbf{Extinción (Lomont):} Mediante a exposición repetida ó estímulo fóbico, en ausencia 
	de consecuencias adversas, extínguese o medo ante este estímulo.
	\item \textbf{Habituación (Lader):} O suxeito habitúase ó estímulo fóbico debido á exposición 
	repetida a este. Este fenómeno vese facilitado pola relaxación.
\end{itemize}

\begin{table}[h!]
\centering
\begin{tabular}{|c|c|c|c|c|}
\hline
\multirow{4}{*}{PROCESOS} & \multirow{2}{*}{\textit{Psicofisiolóxicos}} & \textbf{Non antagónico} & Habituación & \multirow{2}{*}{Efectos a curto prazo} \\ \cline{3-4}
 &  & \textbf{Antagónico} & Inhibición recíproca &  \\ \cline{2-5} 
 & \multirow{2}{*}{\textit{De aprendizaxe}} & \textbf{Non antagónico} & Extinción & \multirow{2}{*}{Efectos a longo prazo} \\ \cline{3-4}
 &  & \textbf{Antagónico} & Contracondicionamento &  \\ \hline
\end{tabular}
\caption{Modelo de Van Egeren dos mecanismos explicativos da DS}
\end{table}

\section{Desensibilización sistemática}
\begin{enumerate}
	\item[$\circ$] \underline{Avaliación da capacidade de visualización}: 
	\begin{itemize}
		\item[•] \textbf{Escea neutra:} Pídese ó suxeito que peche os ollos e descríbeselle unha 
		escea neutra. Mantense a visualización entre 12 e 30s e pregúntaselle por detalles da escea.
		\item[•] \textbf{Escea ansióxena:} Mesmo procedemento que a anterior, pero esta escea debe 
		ser provocadora de ansiedade e non relacionada co medo do paciente. Tras a visualización, 
		pregúntase polo nivel de ansiedade (de 0 a 100) e polos detalles da escea. 
	\end{itemize}
	\item[$\circ$] \underline{Adestramento na resposta incompatible coa ansiedade}: Relaxación 
	progresiva e control da respiración.
	\item[$\circ$] \underline{Elaboración dunha xerarquía de ítems de desensibilización}: De 10 a 20 
	ítems relacionados co medo do paciente, ordeados xerarquicamente de 0 a 100 USAS. Os ítems deben 
	ser concretos e realistas, e o intervalo entre cada un debe estar entre 10 e 15 USAS. Os ítems 
	gradúanse segundo a temática, a distancia ó obxecto temido, a cercanía temporal e a combinación 
	de criterios temáticos e espacio-temporais (mixtos). 
	\item[$\circ$] \underline{Procedemento}:
	\begin{itemize}
		\item[1.] Inducir relaxación. O suxeito indica cando está relaxado mediante un sinal acordado 
		(por exemplo, levantando a man).
		\item[2.] Presentación da primeira escea. Unha vez visualizada, debe indicalo cun sinal. 
		Mantense a visualización entre 7 e 10s. Se se observan respostas de ansiedade ou o suxeito 
		indica ansiedade elevada, detense a visualización e vólvese ó estado de relaxación, e logo 
		vólvese a intentalo.
		\item[3.] Tras a visualización, pregúntase polo nivel de ansiedade de 0 a 100 e indúcese de 
		novo o estado de relaxación durante 20 ou 30s, ata que esté totalmente relaxado.
		\item[4.] Preséntase o mesmo ítem aumentando os tempos de visualización e relaxación, ata que 
		o suxeito poda imaxinalo sen presentar ningunha ansiedade polo menos dúas veces consecutivas. 
		\item[5.] Débense iniciar e finalizar as sesións cunha escea que xa non provoque ansiedade e 
		con relaxación.
	\end{itemize}
\end{enumerate}

\subsection{Variantes da DS}
\begin{itemize}
	\item \textbf{Desensibilización en vivo:} Mesmos pasos que a DS en imaxinación, con problemas 
	adicionais en obter unha xerarquización axeitada e desenvolver a resposta de relaxación.
	\item \textbf{Desensibilización sistemática en grupo:} Realízase en grupos de catro a seis 
	persoas co mesmo medo, ás que se adestra conxuntamente en relaxación e se lles presenta a mesma 
	xerarquía de ítems. A administración dos ítems irá determinada polo progreso do máis lento do 
	grupo.
	\item \textbf{Desensibilización enriquecida:} Empréganse instrumentos que facilitan a 
	visualización das esceas (fotos, diapositivos, olores, ruidos, etc.). Útil para persoas ás que 
	lles costa visualizar as situacións de forma realista.
	\item \textbf{Imaxinación emotiva:} Útil sobre todo en medos infantís. Elabórase unha xerarquía 
	en base ás persoaxes favoritas do neno e ás emocións que estas evocan. Mándaselle pechar os ollos 
	e imaxinar esceas da súa vida diaria, acompañado do seu heroe. Cando surxe unha emoción positiva, 
	introdúcese un ítem.
	\item \textbf{Desensibilización mediante realidade virtual.}
	\item \textbf{Desensibilización por contacto:} Tamén chamada modelado participativo.
	\item \textbf{Desensibilización por medio de movementos sacádicos:} Empréganse os movementos 
	sacádicos dos ollos como resposta incompatible. O paciente visualiza o ítem, e cos ollos abertos 
	e a cabeza inmóbil segue os movementos horizontais dun lápiz ou do dedo do terapeuta, situado a 
	uns 30cm del. Os movementos deben ser rápidos, dous movementos dun lado a outro por segundo, e 
	deben desprazarse entre 40 e 50cm. Realízanse entre 10 e 40 desprazamentos, e ó finalizar a 
	secuencia pregúntase polo nivel de ansiedade e procédese ó periodo de descanso. Cando a resposta 
	de ansiedade se reduce a cero, asócianse pensamentos positivos a estas series de movementos 
	oculares.
\end{itemize}

\end{document}