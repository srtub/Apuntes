\documentclass[a4paper,11pt]{article}
\usepackage[spanish]{babel}        
\usepackage[utf8]{inputenc}           


\usepackage[T1]{fontenc} 
\usepackage{graphicx}    
\usepackage{color}      
\usepackage{anysize}     
\usepackage{multicol}    
\usepackage{multirow}
\usepackage{bm}          
\usepackage{textcomp}   
\usepackage{eurosym}     
\usepackage{amsthm}     
\usepackage{amsmath,amsfonts} 
\usepackage{lineno} 


\marginsize{1.5cm}{1.5cm}{1.5cm}{1.5cm} 
\parindent=0mm                        
\parskip=3mm                         
\renewcommand{\baselinestretch}{1}    
\renewcommand{\spanishtablename}{Táboa}
 

\title{Tema 10: Técnicas de autocontrol}
\date{} 

\begin{document}   

\maketitle 

\section{Bases teóricas}
\subsection{O concepto de autocontrol de Kanfer (1970)}
Conxunto de estratexias que pon en marcha unha persoa para incrementar la probabilidade de emisión dunha resposta, e para reducir a emisión doutra inicialmente máis probable. Nestes casos dase unha
situación conflitiva interna na persoa.

En función da duración da estratexia empregada, este autor considera a existencia de dous tipos de autocontrol:
\begin{itemize}
	\item[•] \textbf{Autocontrol puntual ou decisional:} O suxeito elixe entre dúas condutas, de 
	xeito que a emisión dunha bloquea a emisión da outra.
	\item[•] \textbf{Autocontrol prolongado:} Aínda que o suxeito elixe unha das condutas, a outra 
	segue presente. 
\end{itemize}

O proceso de autocontrol levaríase a cabo en tres fases:
\begin{enumerate}
	\item \underline{Autoobservación}: A persoa analiza tódolos elementos da situación, tanto 
	externos coma internos (o que sinte, o que pensa, o que fai, os elementos do contexto, etc.).
	\item \underline{Autoavaliación}: A persoa valora a súa actuación en relación ó criterio de
	comportamento establecido.
	\item \underline{Autorreforzo}: A persoa prémiase por acadar un obxectivo con reforzadores
	encubertos ou externos, o que posibilita que o proceso de autocontrol se consolide.
\end{enumerate}

\subsection{Modelo de autocontrol de Thoresen e Mahoney (1974)}
Propón que o autocontrol se desenvolve cando a persoa modifica o seu medio externo e interno para promover un cambio significativo na súa conduta. Pode falarse de autocontrol se existen dúas ou máis respostas alternativas, existen consecuencias diferentes e conflitivas para cada unha delas e/ou se manteñen os patróns de autocontrol por consecuencias externas a longo prazo (aprázase a obtención de beneficios). 

\subsection{Aportación de Bandura}
Este autor introduce o concepto de expectativa de autoeficacia, o cal fai referencia á confianza do suxeito nas súas capacidades para acadar determinados logros. A expectativa sobre a eficacia persoal é un proceso cambiante e puede influir tanto nos sentimentos coma nos pensamentos e accións. Neste sentido, unha axeitada percepción de autoeficacia conleva a aparición de pensamentos motivadores da acción e o planteamento de metas acadables. 

Segundo Bandura, as crenzas sobre as propias capacidades constrúense a partir de catro tipos de experiencias:
\begin{itemize}
	\item[-] As aprendizaxes propias e as súas consecuencias.
	\item[-] A estimación da propia capacidade en base á observación dos outros.
	\item[-] As estimacións dos demais acerca das súas capacidades para levar a cabo unha acción.
	\item[-] Os estados fisiolóxicos e afectivos, que poden repercutir sesgando os xuizos de eficacia
	personal do suxeito.
\end{itemize}

\section{Adestramento en autocontrol}
\begin{enumerate}
	\item Favorecer o compromiso ó cambio:
	\begin{itemize}
		\item[$\circ$] Ser consciente da existencia dun problema.
		\item[$\circ$] Motivación e compromiso para establecer novas pautas de funcionamento, e 
		poñelas en práctica de forma activa.
		\item[$\circ$] Focalizar a atención nos beneficios da nova conduta e nos perxuizos da 
		anterior.
		\item[$\circ$] Potenciar expectativas de autoeficacia.
		\item[$\circ$] Comunicar a outros o proxecto de cambio.
	\end{itemize}
	\item Especificar e avaliar o problema.
	\begin{itemize}
		\item[$\circ$] Observar e rexistrar a conduta que se quere cambiar.
		\item[$\circ$] Identificar as variables que inflúen na conducta que é obxecto de cambio.
	\end{itemize}
	\item Planificar os obxectivos de cambio.
	\begin{itemize}
		\item[$\circ$] Definilos de forma clara e realista.
		\item[$\circ$] Establecer metas parciais se o cambio de conduta implica elevados niveis de 
		dificultade.
	\end{itemize}
	\item Deseñar e aplicar as estratexias de cambio.
	\begin{itemize}
		\item[$\circ$] Selección das estratexias axeitadas posta en práctica das mesmas en diferentes 
		situacións.
		\item[$\circ$] Seguimento sistemático mediante autoobservación e rexistro.
	\end{itemize}
	\item Potenciar o mantemento e previr recaídas.
	\begin{itemize}
		\item[$\circ$] Identificar situacións de alto risco de recaída. Coñecer a diferenza entre 
		``recaída'' e fallo ou erro.
		\item[$\circ$] Manter o cambio poñendo en marcha diariamente as estratexias aprendidas.
		\item[$\circ$] Aprender a identificar que situaciones, pensamentos e/ou estados emocionais 
		anteceden a un proceso de recaída.
		\item[$\circ$] Preparación para manexar as posibles recaídas. Aprender a enfrontarse a elas 
		poñendo en práctica tódalas estratexias aprendidas.
	\end{itemize}
\end{enumerate}

\section{Técnicas de autocontrol}
\begin{itemize}
	\item \textbf{Centradas nas condutas:} Adestramento en resposta alternativa (condutas que impiden 
	ou interfiren coa conduta problema, tales como reacción de competencia, relaxación, romper algún 
	eslabón dunha cadea de comportamentos, etc.), contrato condutual, autoobservación e rexistro.
	\item \textbf{Centradas nos antecedentes das condutas:} Técnicas de control estimular. Altéranse 
	os factores que preceden á conduta e que inciden sobre ela, facilitando ou inhibindo a súa  
	aparición. Para Avia, o control estimular pode implicar restricción física (encamiñada a impedir 
	respostas), presentación estímulos discriminativos que incrementen a conduta obxectivo, 
	eliminación de estímulos discriminativos que eliciten a conduta problema, presentación de 
	estímulos que dificulten a emisión da conduta problema, restricción de estímulos discriminativos 
	a contextos establecidos, cambios no medio social e alteración das condicións físicas. 
	\item \textbf{Centradas nas consecuencias das condutas (técnicas de programación condutual):}
	\begin{itemize}
		\item[$\diamond$] \underline{Técnicas de autocastigo}: Se o paciente leva a cabo a conduta 
		problema, isto terá consecuencias aversivas. A técnica máis empregada é a restrición de 
		actividades agradables.
		\item[$\diamond$] \underline{Técnicas de autorreforzo}: Adéstrase ó paciente para que se 
		administre certas continxencias positivas tras a realización da conduta obxectivo. A técnica 
		máis empregada é o reforzo positivo.
	\end{itemize}
	\item \textbf{Técnicas cognitivas:} Dan importancia ó papel dos pensamentos como determinantes de 
	condutas e emocións. Unha vez identificados os procesos de pensamento que executa o paciente en 
	relación co seu problema, trátanse de establecer novas autoverbalizacións que guíen a acción 
	deste cara os obxectivos acordados. As estratexias máis empregadas son a parada de pensamento e o 
	adestramento en autoinstrucións.
\end{itemize}

\end{document}