\documentclass[a4paper,11pt]{article}
\usepackage[spanish]{babel}        
\usepackage[utf8]{inputenc}           


\usepackage[T1]{fontenc} 
\usepackage{graphicx}    
\usepackage{color}      
\usepackage{anysize}     
\usepackage{multicol}    
\usepackage{multirow}
\usepackage{bm}          
\usepackage{textcomp}   
\usepackage{eurosym}     
\usepackage{amsthm}     
\usepackage{amsmath,amsfonts} 
\usepackage{lineno} 


\marginsize{1.5cm}{1.5cm}{1.5cm}{1.5cm} 
\parindent=0mm                        
\parskip=3mm                         
\renewcommand{\baselinestretch}{1}    
\renewcommand{\spanishtablename}{Táboa}
 

\title{Tema 9: Técnicas de condicionamento encuberto}
\date{} 

\begin{document}   

\maketitle 

Permiten alterar a frecuencia de emisión dunha resposta mediante a manipulación das consecuencias, empregando a imaxinación. Céntranse en eliminar condutas desadaptativas, tanto de evitación como
de aproximación.

O procedemento é o seguinte:
\begin{enumerate}
	\item Xustificar de forma razoada o emprego da técnica.
	\item Avaliar a capacidade imaxinativa do paciente mediante adestramento en imaxinación.
	\item Establecer pautas de comunicación entre o paciente e o terapeuta durante a sesión.
	\item Alternar as imaxes das condutas a modificar coas consecuencias ou estímulos relacionados 
	con elas. 
	item Ensinar ó paciente a realizar a secuencia sen axuda.
	\item En cada sesión presentaranse 20 esceas, 10 guiadas polo terapeuta e outras 10 imaxinadas 
	polo paciente, sen axuda.
	\item Programación de tarefas entre sesións.
\end{enumerate}

\section{Técnicas baseadas no condicionamento operante}
\subsection{Técnicas que diminúen a frecuencia de emisión dunha conduta}
\subsubsection{Sensibilización encuberta}
É análoga ó castigo positivo, en tanto que permite diminuir a probabilidade de emisión dunha conduta mediante a presentación dun estímulo aversivo imaxinado, continxente á recreación mental de dita conduta. Esta técnica emprégase en condutas de aproximación. Os pasos son:
\begin{enumerate}
	\item Adestrar ó paciente en relaxación.
	\item Explicar ó paciente que a conduta problema imaxinada irá seguida dun estímulo desagradable.
	\item Solicitar ó paciente que se visualice a si mesmo realizando a conduta problema, a 
	continuación do cal se introduce o estímulo aversivo.
	\item O paciente repite o exercicio sen axuda.
	\item En cada sesión preséntanse 20 esceas, 10 guiadas polo terapeuta e outras 10 imaxinadas 
	polo paciente, sen axuda.
	\item Solicitar o paciente que cando vaia realizar a conduta problema na vida real, trate de 
	recordar a situación adestrada en imaxinación.
\end{enumerate}

\subsubsection{Extinción encuberta}
Baseada na extinción. Permite diminuir a probabilidade de emisión dunha conduta mediante a imaxinación da mesma en ausencia do estímulo reforzador que adoita acompañala. Porén, a súa eficacia non está moi avalada. 

Esta técnica emprégase para eliminar condutas desadaptativas de aproximación ou evitación. Os pasos son:
\begin{enumerate}
	\item Analizar os estímulos reforzadores relacionados coa conduta problema.
	\item Explicar ó paciente como a súa conduta se mantén grazas ós reforzadores externos.
	\item Visualizar a conduta problema en ausencia de ditos reforzadores (20 veces, 10 guiadas e 10 
	individuais). 
\end{enumerate}

\subsubsection{Custo de resposta encuberto}
Baseada no castigo negativo. Permite diminuir a frecuencia de emisión dunha conduta desadaptativa (de aproximación ou evitación) ó asociar dita emisión á perda dun reforzador positivo. Neste caso, para evitar a habituación, altérnanse distintos reforzadores en cada sesión. Estes determínanse mediante o \textit{Cuestionario de custo de resposta encuberto} (Upper e Cautela), que inclúe 20 ítems que avalían de 1 a 5 a molestia provocada por diferentes situacións. 

A eficacia desta técnica non está ampliamente demostrada, pero observáronse resultados positivos en estudos controlados. Os pasos son:
\begin{enumerate}
	\item Visualizar a conduta problema.
	\item Empregar unha clave (cambio, por exemplo) que leve a substuír a visualización da conduta 
	problema pola visualización dunha consecuencia desagradable (perda do reforzador positivo)
	\item Introducir a imaxe de perda. 
	\item Repítese a secuencia 20 veces, 10 guiadas e 10 individuais. 
\end{enumerate}

\subsection{Técnicas que aumentan a frecuencia de emisión dunha conduta}
\subsubsection{Reforzo positivo encuberto}
Permite aumentar a frecuencia de emisión dunha conduta positiva mediante o reforzo positivo en imaxes, aplicando os principios do condicionamento operante. Os pasos son:
\begin{enumerate}
	\item Establecer pautas de comunicación entre terapeuta e paciente.
	\item Determinar os estímulos reforzadores.
	\item Solicitar ó paciente que visualice a conduta obxectivo, e a continuación a imaxe 
	reforzante.
\end{enumerate}

\subsubsection{Reforzo negativo encuberto}
Baseada no reforzo negativo. Permite aumentar a probabilidade de emisión dunha conduta positiva ó asociar esta á desaparición dun estímulo aversivo. Emprégase só cando fracasan outro tipo de procedementos encubertos, por ser máis agresivo. Os pasos son:
\begin{enumerate}
	\item Seleccionar os estímulos aversivos, tendo coidado de que non dificulten o cambio rápido de 
	imaxes (non se recomenda empregar náuseas ou vómitos, por exemplo).
	\item Solicitar ó paciente que imaxine a escea desagradable.
	\item Solicitar ó paciente que imaxine rapidamente a conduta obxectivo. O cambio de imaxes debe 
	ser inmediato ante o sinal do terapeuta, para evitar o condicionamento cara atrás. 
\end{enumerate}

\section{Técnicas baseadas na aprendizaxe social: modelado encuberto}
Permite a aprendizaxe de novas condutas ou a modificación doutras existentes mediante a observación en imaxinación do comportamento dun modelo e das súas consecuencias. Esta técnica é útil para pacientes que non son capaces de imaxinarse si mesmos realizando as condutas, pero si a outras persoas. 

O procedemento consiste en presentar varias esceas que o paciente debe imaxinar. O tempo transcorrido entre a presentación das mesmas pode variar entre un e cinco minutos.

É recomendable que o modelo presente características semellantes ás do suxeito. Cantos máis modelos se empreguen, maior será a xeralización. Os modelos ``inexpertos'' adoitan ser máis efectivos que os ``expertos''. É importante ter en conta os procesos atencionais, a capacidade de retención e o nivel de activación do paciente.

\section{Técnicas baseadas no autocontrol}
\begin{itemize}
	\item \textbf{Técnica de parada de pensamento:}
	\begin{itemize}
		\item[1.] Pedir ó paciente, sentado e cos ollos pechados, que pense naquelo que lle preocupa, 
		e que unha vez visualice estes pensamentos o indique cun xesto.
		\item[2.] Dar unha orde (para, por exemplo), que pode ir acompañada dalgún ruido. Isto 
		provoca unha alarma no paciente. Explicarlle como o procedemento interrumpiu a súa secuencia 
		de pensamentos.
		\item[3.] Ensinar ó paciente a darse mentalmente esta orde, acompañándoa dunha imaxe positiva 
	\end{itemize}
	\item \textbf{Triada do autocontrol:} 
	\begin{itemize}
		\item[1.] Deter o pensamento subvocal.
		\item[2.] Indicar ó paciente que respire profundamente e se relaxe.
		\item[3.] Inducir a visualización dunha escea agradable.
	\end{itemize}
\end{itemize}

\end{document}