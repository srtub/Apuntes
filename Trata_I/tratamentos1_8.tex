\documentclass[a4paper,11pt]{article}
\usepackage[spanish]{babel}        
\usepackage[utf8]{inputenc}           


\usepackage[T1]{fontenc} 
\usepackage{graphicx}    
\usepackage{color}      
\usepackage{anysize}     
\usepackage{multicol}    
\usepackage{multirow}
\usepackage{bm}          
\usepackage{textcomp}   
\usepackage{eurosym}     
\usepackage{amsthm}     
\usepackage{amsmath,amsfonts} 
\usepackage{lineno} 


\marginsize{1.5cm}{1.5cm}{1.5cm}{1.5cm} 
\parindent=0mm                        
\parskip=3mm                         
\renewcommand{\baselinestretch}{1}    
\renewcommand{\spanishtablename}{Táboa}
 

\title{Tema 8: Técnicas operantes con sistemas de organización de continxencias}
\date{} 

\begin{document}   

\maketitle 

Este tipo de procedementos baséanse na organización e sistematización de continxencias. Son tratamentos que explicitan a relación entre as condutas e as súas consecuencias, sexan estas positivas ou negativas. Permiten traballar con varias condutas e persoas de forma simultánea. Empregan reforzadores xeralizados, e esixen un amplo control dos estímulos reforzantes do ambiente.

\section{Economía de fichas}
Sistema que motiva ós pacientes a realizar condutas desexables e diminuir condutas indesexables. Os seus elementos básicos son as condutas a traballar, as fichas (reforzadores artificiais con forma física, que son intercambiables), os reforzadores de apoio (intercambiables polas fichas), os valores das fichas e as regras de cambio. O procedemento desenvólvese en tres fases:
\begin{itemize}
	\item \textbf{Fase de implantación:} O suxeito obtén fichas que intercambia por reforzadores de 
	apoio.
	\item \textbf{Fase de implantación:}
	\begin{itemize}
		\item[1.] Definir as condutas que van ser reforzadas, en termos claros e comprensibles e de 
		maneira que poidan ser observables e rexistrables.
		\item[2.] Elixir o tipo de fichas máis axeitadas para os suxeitos. Estas deben presentar unha 
		serie de características:
		\begin{itemize}
			\item[-] Non ser fáciles de falsear ou obter por outros métodos.
			\item[-] Ser facilmente manipulables e acumulables.
			\item[-] Estar feitas dun material resistente, debido á constante manipulación.
			\item[-] Poden ser de colores para que sexa máis sinxelo asignarlles valores distintos.
		\end{itemize}
		\item[3.] Seleccionar os reforzadores de apoio. Estes tamén deben presentar unha serie de 
		características:
		\begin{itemize}
			\item[-] Non empregar reforzadores que poidan obterse fóra do programa.
			\item[-] Determinar o valor de cada un.
			\item[-] Facilitar ós pacientes unha copia dos reforzadores co seu valor en fichas.
			\item[-] Axustalos a cada suxeito e ó momento do programa (máis valor a reforzadores máis 
			solicitados, esixir máis condutas positivas a medida que avanza o programa, etc.).
		\end{itemize}
		\item[4.] Establecer as condicións de cambio de fichas (nº de fichas por cada conduta, custo 
		de cada reforzador e momentos, lugares e persoas que realizarán a entrega de fichas e os 
		intercambios).
		\item[5.] Incluir sistemas de bonificación e penalización (pago de fichas tras unha 
		infracción, tempo fóra de ganacia de fichas, tempo fóra de intercambio de fichas, etc.). 
		\item[1.] Elaborar un sistema de rexistro no que aparezan as condutas positivas, as fichas 
		que se gañan con cada unha, as fichas gañadas, canxeadas, perdidas e acumuladas e os 
		reforzadores de apoio. 
	\end{itemize}
	\item \textbf{Fase de desvanecemento:} Aumentar os tempos de entrega e intercambio de fichas, 
	reducir o número de fichas gañadas por cada conduta positiva e aumentar o número de fichas 
	requerido para obter os reforzadores, todo isto de forma progresiva. 
\end{itemize}

As vantaxes e desvantaxes do emprego deste procedemento son:
\begin{multicols}{2}
VANTAXES:
	\begin{itemize}
		\item[-] Permite cuantificar a entrega dos reforzadores, a emisión de condutax axeitadas e a 
		selección dos reforzadores de apoio por parte do paciente.
		\item[-] Redúcese a demora no reforzo.
		\item[-] Evítanse posibles problemas de saciación de reforzadores.
		\item[-] Programa individual e flexible.
		\item[-] Permite a súa aplicación en diferentes ámbitos, con varias condutas e con varias 
		persoas.
		\item[]
		\item[]
	\end{itemize}
	
DESVANTAXES:
	\begin{itemize}
		\item[-] As condutas non se xeralizan a outros ambientes e poden non emitirse unha vez 
		finalizado o programa.
		\item[-] Necesítase adestramento e capacitación do persoal a cargo do programa.
		\item[-] Pode resultar caro.
		\item[-] Hai que ter en conta diversas restriccións legais e éticas (algúns reforzadores non 
		poden empregarse como privilexios, se constitúen dereitos básicos).
		\item[-] Existe a posibilidade de obter reforzos externos ó programa.
	\end{itemize}
\end{multicols}

\section{Contratos condutuais}
Son acordos escritos ou verbais entre dúas ou máis persoas, que especifican a relación entre as condutas obxectivo e as súas consecuencias. Os seus termos negócianse entre as partes implicadas. Constitúen unha alternativa útil, rápida e económica á economía de fichas, en especial se as persoas non presentan limitacións intelectuais e non se precisa un control exhaustivo do medio ambiente.

Os pasos a seguir son:
\begin{enumerate}
	\item Delimitar as condutas obxectivo para cada unha das persoas implicadas, especificando a 
	frecuencia e a duración das mesmas. 
	\item Establecer as consecuencias positivas pola realización de cada conduta obxectivo, e as 
	consecuencias negativas pola realización de condutas inaxeitadas. Pode incluirse unha cláusula de 
	bonificación, que outorgue privilexios adicionais cando se excedan as demandas mínimas do 
	contrato.
	\item Establecer criterios de avaliación das condutas.
	\item Establecer o momento de inicio, a duración e as revisións do contrato.
	\item Establecer o momento no que se obteñen as consecuencias (positivas e negativas).
\end{enumerate}

Inicialmente as consecuencias deben ser continxentes á conduta (tanto positiva como negativa), e posteriormente convén alongar a demora para aproximarse ás condicións do medio habitual. Os criterios para obter recompensas serán máis laxos ó principio e máis estrictos a medida que dure o contrato. Debe facerse especial fincapé nas condutas positivas fronte ás negativas. 

Os contratos deben plasmarse de forma física, e o terapeuta ten que proporcionar unha copia a cada un dos implicados. Unha vez corrixidas as condutas inaxeitadas cesará o contrato, xa que o obxectivo é que o paciente se adapte ás condicións habituais do seu medio. 

Existen distintos tipos de contrato:
\begin{itemize}
	\item[$\clubsuit$] \underline{Contratos unilaterais}: O acordo implica só a unha persoa. 
	Establécense entre o terapeuta e o seu paciente.
	\item[$\clubsuit$] \underline{Contratos múltiples}: Implican a varias persoas. Son útiles porque 
	as partes implicadas actúan á vez como axentes de reforzo e de control.
	\item[$\clubsuit$] \underline{\textit{Quid pro quo}}: Os reforzadores que un suxeito obtén pola 
	súa conduta dependen da modificación da conduta do outro.
	\item[$\clubsuit$] \underline{Contratos paralelos}: Os cambios de conduta dun paciente non se 
	empregan para reforzar o cambio de conduta noutro. As recompensas son individuais. 
\end{itemize}

\end{document}